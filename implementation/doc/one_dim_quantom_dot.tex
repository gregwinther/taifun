
\begin{tcolorbox}
    {\fontfamily{cmss}\selectfont
    \textbf{class} quantum\_systems.\textbf{OneDimensionalHarmonicOscillator}

    \hspace{1em}(\emph{n}, \emph{l}, \emph{grid\_length}, \emph{num\_grid\_points}, 
    \emph{omega=$0.25$}, \emph{mass=1}, \emph{a=$0.25$}, \emph{alpha=$1.0$})

    \vspace{1em}
    Create One-Dimensional Quantum Dot basis set.
    \vspace{1em}

    \textbf{Parameters}

    \hspace{2em}\textbf{n}(\emph{int}) Number of electrons
    
    \hspace{2em}\textbf{l}(\emph{int}) Number of spinorbitals
    
    \hspace{2em}\textbf{grid\_length}(\emph{int or float}) Space over which to 
        construct wavefunction.
    
    \hspace{2em}\textbf{num\_grid\_points}(\emph{int of float}) Number of 
        points for construction of wavefunction.

    \hspace{2em}\textbf{omega}(\emph{float, default $0.25$}) Angular frequency of
        harmonic oscillator potential.
    
    \hspace{2em}\textbf{mass}(\emph{int or float, default $1.0$}) Mass of electrons.
        Atomic units is used as default.
    
    \hspace{2em}\textbf{a}(\emph{float, default $0.25$}) Necessary in Coulomb 
        integral computation.
    
    \hspace{2em}\textbf{alpha}(\emph{float, default $1.0$}) Necessary in Coulomb 
        integral computation.

    \vspace{1em}
    \textbf{Attributes}

    \hspace{2em} \textbf{h}
    One-body matrix 
    \textbf{Type} np.array
    
    \hspace{2em} \textbf{f}
    Fock matrix
    \textbf{Type} np.array

    \hspace{2em} \textbf{u}
    Two-body matrix
    \textbf{Type} np.array

    \vspace{1em}
    \textbf{Methods}

    \hspace{2em} \textbf{setup\_system}(\emph{Potential=None})
        \begin{adjustwidth}{4em}{}
        Must be called in order to compute basis functions. The method will 
        revert to regular harmonic oscillator potential if no potential has been 
        provided. Optional potentials include one-dimensional double well potentials.           
        \end{adjustwidth}
   
    \hspace{2em} \textbf{contruct\_dipole\_moment}()
        \begin{adjustwidth}{4em}{}
        Constructs dipole moment. This method is called by
        \textbf{setup\_system}(). Necessary when constructing custom systems with 
        time development.
        \end{adjustwidth}
    }
\end{tcolorbox}
