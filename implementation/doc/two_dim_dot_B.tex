
\begin{tcolorbox}
    {\fontfamily{cmss}\selectfont
    \textbf{class} quantum\_systems.\textbf{TwoDimHarmonicOscB}

    \hspace{1em}(\emph{n}, \emph{l}, \emph{radius\_length}, \emph{num\_grid\_points}, 
    \emph{omega\_0=$1.0$}, \emph{mass=$1$}, \emph{omega\_c=$0$})

    \vspace{1em}
    Create Two-Dimensional Quantum Dot with constant homogenous magnetic field.
    This class inherits from \textbf{TwoDimensionalHarmonicOscillator}.
    \vspace{1em}

    \textbf{Parameters}

    \hspace{2em}\textbf{n}(\emph{int}) Number of electrons
    
    \hspace{2em}\textbf{l}(\emph{int}) Number of spinorbitals
    
    \hspace{2em}\textbf{grid\_length}(\emph{int or float}) Space over which to 
        construct wavefunction.
    
    \hspace{2em}\textbf{num\_grid\_points}(\emph{int of float}) Number of 
        points for wavefunction.

    \hspace{2em}\textbf{omega\_0}(\emph{float, default $1.0$}) Part of harmonic 
        osc. not dep. on magnetic field. 
    
    \hspace{2em}\textbf{mass}(\emph{int or float, default $1.0$}) Mass of electrons.
        Atomic units is used as default.
    
    \hspace{2em}\textbf{omega\_c}(\emph{float, default $0$}) Larmor frequency.

    \vspace{1em}
    \textbf{Attributes}

    \hspace{2em} \textbf{h}
    One-body matrix 
    \textbf{Type} np.array
    
    \hspace{2em} \textbf{f}
    Fock matrix
    \textbf{Type} np.array

    \hspace{2em} \textbf{u}
    Two-body matrix
    \textbf{Type} np.array

    \vspace{1em}
    \textbf{Methods}

    \hspace{2em} \textbf{setup\_system}()
        \begin{adjustwidth}{4em}{}
        Must be called in order to compute basis functions.
        \end{adjustwidth}

    \hspace{2em} \textbf{construct\_dipole\_moment}()
        \begin{adjustwidth}{4em}{}
        Constucts dipole moment. This method is called by setup\_system().
        \end{adjustwidth}
    }
\end{tcolorbox}
