
\begin{tcolorbox}
    {\fontfamily{cmss}\selectfont
    \textbf{class} quantum\_systems.\textbf{TwoDimensionalDoubleWell}

    \hspace{1em}(\emph{n}, \emph{l}, \emph{radius\_length}, \emph{num\_grid\_points}, 
    \emph{barrier\_strength=}$1.0$, \emph{l\_ho\_factor=$1.0$}, 
    \emph{omega=$0.25$}, \emph{mass=$1$})

    \vspace{1em}
    Create Two-Dimensional Quantum Dot with double well potential, i.e. the Double Dot.
    This class inherits from \textbf{TwoDimensionalHarmonicOscillator}.
    \vspace{1em}

    \textbf{Parameters}

    \hspace{2em}\textbf{n}(\emph{int}) Number of electrons
    
    \hspace{2em}\textbf{l}(\emph{int}) Number of spinorbitals
    
    \hspace{2em}\textbf{grid\_length}(\emph{int or float}) Space over which to 
        construct wavefunction.
    
    \hspace{2em}\textbf{num\_grid\_points}(\emph{int of float}) Number of 
        points for wavefunction.

    \hspace{2em}\textbf{barrier\_strength}(\emph{float, default $1.0$}) Barrier strength 
        in double well potential.
    
    \hspace{2em}\textbf{l\_ho\_factor}(\emph{float, default $1.0$}) Normal HO vs double
        well basis function.

    \hspace{2em}\textbf{omega}(\emph{float, default $0.25$}) Angular frequency of
        harmonic oscillator potential.
    
    \hspace{2em}\textbf{mass}(\emph{int or float, default $1.0$}) Mass of electrons.
        Atomic units is used as default.

    \vspace{1em}
    \textbf{Attributes}

    \hspace{2em} \textbf{h}
    One-body matrix 
    \textbf{Type} np.array
    
    \hspace{2em} \textbf{f}
    Fock matrix
    \textbf{Type} np.array

    \hspace{2em} \textbf{u}
    Two-body matrix
    \textbf{Type} np.array

    \vspace{1em}
    \textbf{Methods}

    \hspace{2em} \textbf{setup\_system}(\emph{axis=$0$})
        \begin{adjustwidth}{4em}{}
        Must be called in order to compute basis functions. Parameter \emph{axis}
        decices to which axis the well barrier is aligned. $(0,1) = (x,y)$.           
        \end{adjustwidth}
    }
\end{tcolorbox}
