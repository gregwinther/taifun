\begin{tcolorbox}
    {\fontfamily{cmss}\selectfont
    \textbf{class} quantum\_systems.\textbf{QuantumSystem}
    (\emph{n}, \emph{l}, \emph{n\_up=None}, \emph{np=None})

    \vspace{1em}
    Abstract base class defining the common methods used by all different quantum systems.

    \vspace{1em}
    \textbf{Parameters:} 

    \hspace{2em} \textbf{n} (\emph{int}) Number of electrons
    
    \hspace{2em} \textbf{l} (\emph{int}) Number of spinorbitals

    \hspace{2em} \textbf{n\_up} (\emph{int, default=None}) Number of spin-up spinorbitals

    \hspace{2em} \textbf{np} (\emph{module}) Matrix library, i.e. numpy, cupy etc.

    \vspace{1em}
    \textbf{Attributes:}

    \hspace{2em} \textbf{h}
    One-body matrix 
    \textbf{Type} \emph{np.array}
    
    \hspace{2em} \textbf{f}
    Fock matrix
    \textbf{Type} \emph{np.array}

    \hspace{2em} \textbf{u}
    Two-body matrix
    \textbf{Type} \emph{np.array}

    \hspace{2em} \textbf{s}
    Overlap matrix of spinorbitals
    \textbf{Type} \emph{np.array}
    
    \hspace{2em} \textbf{spf}
    Single-particle functions
    \textbf{Type} \emph{np.array}

    \hspace{2em} \textbf{spf\_bra}
    Conjugated single-particle functions
    \textbf{Type} \emph{np.array}

    % \hspace{2em} \textbf{o}
    % Occupied orbital indices
    % \textbf{Type} \emph{Slice} object

    % \hspace{2em} \textbf{v}
    % Virtual orbital indices
    % \textbf{Type} \emph{Slice} object

    \vspace{1em}
    \textbf{Methods:}

    \hspace{2em} \textbf{setup\_system}()
    \begin{adjustwidth}{4em}{}
        Method must be implemented by subclasses.
    \end{adjustwidth}

    \hspace{2em} \textbf{change\_basis}(\emph{c}, \emph{c\_tilde=None})
    \begin{adjustwidth}{4em}{}
        Changes basis of system according to coefficient matrices $\vb{C}$
        and $\tilde{\vb{C}}$.
    \end{adjustwidth}

    \hspace{2em} \textbf{change\_to\_hf\_basis}(\emph{*args}, \emph{verbose=False} \emph{**kwargs})
    \begin{adjustwidth}{4em}{}
        Changes basis of system to Hartree-Fock basis.
    \end{adjustwidth}

    \hspace{2em} \textbf{set\_time\_evolution\_operator}(\emph{time\_evolution\_operator})
    \begin{adjustwidth}{4em}{}
        Setter for time-evolution operator.

        \textbf{Parameters:} 
        
        \hspace{1.5em} \textbf{time\_evolution\_operator} (\emph{TimeEvolutionOperator}) 

    \end{adjustwidth}

    }
\end{tcolorbox}