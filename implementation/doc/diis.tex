\begin{tcolorbox}
    {\fontfamily{cmss}\selectfont
    \textbf{class} coupled\_cluster.mix.\textbf{DIIS}
    (\emph{num\_vecs=$10$}, \emph{np=None})

    \vspace{1em}
    General vector mixing class to accelerate quasi-Newton method using the 
    direct inversion of iterative space (DIIS) scheme. Inherits from 
    \emph{AlphaMixer}.
        
    \vspace{1em}
    \textbf{Parameters}

    \hspace{2em}\textbf{num\_vecs}(\emph{float, default 10}) 
        Number of vectors to keep in memory.

    \hspace{2em}\textbf{np}(\emph{Module})
        Matrix library to be used, e.g. numpy, cupy.

    \vspace{1em} 
    \textbf{Methods}

    \hspace{2em} \textbf{compute\_new\_vector}
        (\emph{trial\_vector}, \emph{direction\_vector} \emph{error\_vector})

        \begin{adjustwidth}{4em}{}
        Computes new trial vector for mixing with full right hand side of amplitude 
        equation.

        \textbf{Parameters:} 

            \hspace{1.5em}\textbf{trial\_vector} (\emph{np.array}) 
            Initial vector for mixing

            \hspace{1.5em}\textbf{direction\_vector} (\emph{np.array})
            Vector to be added to \emph{trial\_vector}.

            \hspace{1.5em}\textbf{error\_vector} (\emph{np.array})
            Error vector associated with QN DIIS. 

        \textbf{Returns:} New mixed vector

        \textbf{Return type:} \emph{np.array}
        \end{adjustwidth}

    \hspace{2em} \textbf{clear\_vectors}()
        \begin{adjustwidth}{4em}{}
        Delete all stored vectors.
        \end{adjustwidth}
    } 
\end{tcolorbox}