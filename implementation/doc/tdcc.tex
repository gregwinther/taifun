\begin{tcolorbox}
    {\fontfamily{cmss}\selectfont
    \textbf{class} coupled\_cluster.cc.\textbf{TimeDependentCoupledCluster}

    \hspace{1em}(\emph{cc}, \emph{np=None}, \emph{integator=None} \emph{**cc\_kwargs})

    \vspace{1em}
    Abstract base class defining the basic structure for a time-dependent coupled cluster 
    solver.

    \vspace{1em}
    \textbf{Parameters}

    \hspace{2em}\textbf{cc}(\emph{CoupledCluster})
        Class instance defining the ground state solver.

    \hspace{2em}\textbf{system}(\emph{QuantumSystem}) 
        Class instance defining the system to be solved.

    \hspace{2em}\textbf{np}(\emph{module})
        Matrix/linear algebra library to be used, e.g. Numpy, Cupy
    
    \hspace{2em}\textbf{integrator}(\emph{Integrator})
        Integrator class instance, e.g. RK4, GaussIntegrator

    \vspace{1em}
    \textbf{Methods}

    \hspace{2em} \textbf{compute\_ground\_state}
        (\emph{t\_args=[], \emph{t\_kwargs=\{\}},
        \emph{l\_args=[]}, \emph{l\_kwargs}})
        \begin{adjustwidth}{4em}{}
        Call on method from \emph{CoupledCLuster} class to compute ground
        state of system.           
        \end{adjustwidth}


    \hspace{2em} \textbf{compute\_particle\_density}()
        \begin{adjustwidth}{4em}{}
        Computes one-body density at time $t$.

        \textbf{Returns:} Particle density 

        \textbf{Return type:} \emph{np.array} 
        \end{adjustwidth}     

    \hspace{2em} \textbf{rhs\_l\_amplitudes}()
        \begin{adjustwidth}{4em}{}
        Function that needs to be implemented as a generator. The generator 
        should return the $\lambda$-amplitudes right-hand sides, in order of 
        increasing excitation.           
        \end{adjustwidth}

    \hspace{2em} \textbf{rhs\_t\_amplitudes}()
        \begin{adjustwidth}{4em}{}
        Function that needs to be implemented as agenerator. The generator 
        should return the $\tau$-amplitudes right-hand sides, in order of 
        increasing excitation.
        \end{adjustwidth}

    \hspace{2em} \textbf{set\_initial\_conditions}(\emph{amplitudes=None})
        \begin{adjustwidth}{4em}{}
        Set initial condition of system. It is necessary to make a call to 
        this system before computing time-development. Can be called without 
        argument. Will in that case revert to amplitudes of ground state solver.

        \textbf{Parameters: }

            \hspace{1.5em} \textbf{amplitudes}(\emph{AmplitudeContainer})
                Amplitudes for initial system configuration.
        \end{adjustwidth}

    \hspace{2em} \textbf{solve} (\emph{time\_points}, \emph{timestep\_tol=$1e^{-8}$})
        \begin{adjustwidth}{4em}{}
        Develop given system in time, specified by an array of \emph{time\_points}.
        Integrates equations of motion repeatedly, over all time points.
        
        \textbf{Parameters:}

            \hspace{1.5em} \textbf{time\_points} (\emph{list, np.array})
                Time points over which to integrate EOM.

            \hspace{1.5em} \textbf{timestep\_tol} (\emph{float, default $1e^{-8}$})
                Tolerance in size of steps \lstinline{dt}.

        \textbf{Returns:} Amplitudes
        \textbf{Return type:} \emph{AmplitudeContainer}
        \end{adjustwidth}
   
    } 
\end{tcolorbox}