\chapter{Quantum Systems}

For a quantum system to be studied on the computer it is necessary to 
make a distinction for what defines the system. One must therefore undergo the
mathematical procedure of defining a finite basis sets that defines the 
quantum system under scrutiny when dealing with the electronic problem. 

Blah blah blah

Here we present the \lstinline{quantum_system} python module, designed to 
provide basis sets for one- and two-dimensional quantum dots. The two-dimensional 
quantum dot can also be modelled with a constant, homogeneous magnetic field. 
Moreover, the module includes an option for constructing a custom system which can 
be interfaced with popular quantum chemistry packages \lstinline{PySCF} and 
\lstinline{Psi4}\cite{PYSCF, parrish2017psi4}.

\section{Quantum Dots}

\subsection{One Dimension}

\subsection{Two Dimensions}

\subsection{Magnets and Miracles}

\section{Constructing a Custom System}

\section{Time Evolution}

