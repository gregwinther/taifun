\chapter{Implementation: Quantum Systems}

For a quantum system to be studied on the computer it is necessary to 
make a distinction for what defines the system. One must therefore undergo the
mathematical procedure of defining a finite basis sets that defines the 
quantum system under scrutiny when dealing with the electronic problem. 

Here we present the \lstinline{quantum_system} python module, designed to 
provide basis sets for one- and two-dimensional quantum dots. The two-dimensional 
quantum dot can also be modelled with a constant, homogeneous magnetic field; and 
as a double quantum dot. 
Moreover, the module includes an option for constructing a custom system which can 
be interfaced with popular quantum chemistry packages \lstinline{PySCF}\cite{PYSCF} 
and \lstinline{Psi4}\cite{parrish2017psi4}.

The \lstinline{quantum_system} module can installed from github with \lstinline{pip}
by the following command,
\begin{lstlisting}[language=bash]
pip install git+https://github.com/Schoyen/quantum-systems.git
\end{lstlisting}
The same task can of course be accomplished by more commands,
\begin{lstlisting}[language=bash]
git clone https://github.com/Schoyen/quantum-systems.git
cd quantum-systems
pip install .
\end{lstlisting}

\noindent
--------------\\
NOTE: not sure the part below here is necessary. Maybe just link to web page 
documentation? If it is done, that is.. \\
--------------

It can be useful to install the module to a separate environment. We have made this 
possible through \lstinline{conda},
\begin{lstlisting}[language=bash]
conda environment create -f environment.yml
conda activate quantum-systems    
\end{lstlisting}

\section{Quantum Dots}

In reality, quantum dots are nanometre-sized structures made of semiconductor materials.
Theoretically, quantum dots are easy to model by harmonic oscillator potential and in practice
they are relativelty easy to manufacture in a laboratory. This doubly
theoretical-experimental benefit has made quantum dots a popular area of study. Moreover so 
because of their wide area of apllications.

The possible applications of quantum dots are many. Coupled single-electron quantum dots 
could potentially be used as harware elements in quantum computers, i.e.
qubits\cite{loss1998quantum}; quantum dots also promise to increase the efficiency of 
photvoltaic solar cells; and they have already found use in cellular imaging in biology.
Reimann and Manninen\cite{reimann2002electronic} has written an outstandingly
thorough review on quantum dots, covering their varied types of fabrication, theoretical
methods common in their study and vast ocean of applications.

The usefulness and relative theoretical ease of modelling warrants the study of quantum dots.
Herein, several classes have been implemented in order to construct basis sets modelling 
quantum dots in both one and two dimensions. These basis sets models \emph{bound} systems
as the common harmonic oscillator-type potentials that are used have the characteristics of 
infinite quantum wells. 

\subsection{One Dimension}

\input{implementation/doc/one_dim_quantom_dot.tex}

\subsection{Two Dimensions}

\subsubsection{Double well}

\subsubsection{Magnetic field}

\section{Constructing a Custom System}

\section{Time Evolution}

