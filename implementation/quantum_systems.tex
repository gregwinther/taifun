\chapter{Implementation: Quantum Systems}

For a quantum system to be studied on the computer it is necessary to 
make a distinction for what defines the system. One must therefore undergo the
mathematical procedure of defining a finite basis sets that defines the 
quantum system under scrutiny when dealing with the electronic problem. 

Here we present the \lstinline{quantum_system} python module, designed to 
provide basis sets for one- and two-dimensional quantum dots. The two-dimensional 
quantum dot can also be modelled with a constant, homogeneous magnetic field; and 
as a double quantum dot. 
Moreover, the module includes an option for constructing a custom system which can 
be interfaced with popular quantum chemistry packages \lstinline{PySCF}\cite{PYSCF} 
and \lstinline{Psi4}\cite{parrish2017psi4}.

The \lstinline{quantum_system} module can installed from github with \lstinline{pip}
by the following command,
\begin{lstlisting}[language=bash]
pip install git+https://github.com/Schoyen/quantum-systems.git
\end{lstlisting}
The same task can of course be accomplished by more commands,
\begin{lstlisting}[language=bash]
git clone https://github.com/Schoyen/quantum-systems.git
cd quantum-systems
pip install .
\end{lstlisting}

\noindent
--------------\\
NOTE: not sure the part below here is necessary. Maybe just link to web page 
documentation? If it is done, that is.. \\
--------------

It can be useful to install the module to a separate environment. We have made this 
possible through \lstinline{conda},
\begin{lstlisting}[language=bash]
conda environment create -f environment.yml
conda activate quantum-systems    
\end{lstlisting}

\section{Quantum Dots}

\subsection{One Dimension}

\input{implementation/doc/one_dim_quantom_dot.tex}

\subsection{Two Dimensions}

\subsection{Magnets and Miracles}

\section{Constructing a Custom System}

\section{Time Evolution}

