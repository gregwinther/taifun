\chapter{Summary Remarks}

The aim of this thesis was to develop and test numerical methods for 
solving the time-dependent Schrödinger equation. In particular, we 
wanted to develop the Orbital Adaptive Time-Dependent Coupled Cluster 
(OATDCC) method introduced by \citeauthor{kvaal2012ab}\cite{kvaal2012ab}.

We can with confidence say that this endevour has been succesful.
We have implemented both 
a time-depedent Coupled Cluster Singles Doubles (TDCCSD) solver with 
static orbitals and an \emph{Orbital Adaptive} Time-Dependent Coupled Cluster 
Doubles (OATDCCD) solver, as well as several quantum systems and an interface 
to quantum chemistry software which enables extraction of further basis sets.
The resulting product is two modules for python; \lstinline{quantum_systems} 
and \lstinline{coupled_cluster}.

With these methods we have have managed to produce 
many results found in the existing literature on time-dependent ab initio 
many-body solvers. This includes a reproduction of the instantaneous dipole 
of $H_2$ from \citeauthor{li2005time}\cite{li2005time}, the time-dependent 
ground state probability of a one-dimensional quantum dot from 
\citeauthor{Zanghellini04}\cite{Zanghellini04}, the dipole spectrum of 
helium from \citeauthor{pedersen2019symplectic}\cite{pedersen2019symplectic}
and the ionisation of a one-dimensional model of beryllium from 
\citeauthor{miyagi2013time}\cite{miyagi2013time}.

We have been able to compute the time-development 
of a relatively high number of particles, up to $n=12$ electrons in one- and
two-dimensional quantum dots. This stands as a testimony to the robustment 
of our implementations. For ever-increasing basis set size we have produced 
time-dependent results that converge, adding to the confidence in the results.
The results relating to the quantum dots are in accordance with the 
results one would expect from theory. Specifically, in both one-dimensional 
and two-dimensional quantum dots we have shown that the harmonic potential 
theorem\cite{kohn1961cyclotron} holds - the dipole spectrum of quantum 
dots consists of only one line consistent with the oscillator frequency in 
ordinary harmonic quantum dots. For the two-dimensional double dot system,
we also see only one spectral line, but a shift in frequency. By adding a homogenous,
static magnetic field to the quantum dot, we see a split in the dipole spectrum 
resulting in two spectral lines at a distance equal to the Larmor frequency of 
the applied magnetic field, also in agreement with the harmonic potential theorem.

In assesment of the time-dependent coupled cluster method with static orbitals,
we found that the solver struggles to correctly represent the current state if 
the system progresses too far away from the inital reference state. As such the 
method is not a method that one can trust blindly, as it returns
inaccurate results in these situations. This phenomenon manisfests most clearly as 
non-sensical ground state overlap values, but also as exceedingly high amplitude 
values. The orbital-adaptive solver will remedy this problem, but has a drawback 
in that it cannot feasibly produce inner products of two states in time, and therefore 
is unable to compute time-dependent ground state probabilities.

This study, with the addition of the study by \citeauthor{islandwind2019}\cite{islandwind2019},
have only scratched the surface with regards to the cornucopia of results that our 
code base is able to produce. The reason for this is for the most part attributed to 
the time spent in development and the amount in wall clock time necessary to 
produce the results herein. A challenge with time-dependent studies is the 
impossibility of true parallelisation, as time step $t_n$ will always depend 
on $t_{n-1}$.

\section{Further Studies}

By study of the results we have produced, we think that a comparison study 
of the method with static orbitals and the one with adaptive orbitals is warranted.
One would assume that the two methods yields identical results within a reasonable 
upper limit in basis set size, but under what conditions do we get close to this
limit? We can for instance imagine that the orbital-adaptive scheme yields equivalent
results compared to the static-orbital method for a lower number of spin-orbitals,
because of the automatic adaption to the current state that an orbital adaption 
provides. A study is in preparation for publication which seeks to stress-test
the orbital-adaptive method\cite{islandwind2019numerically}.

As of now, the apparatus we have manufactured can easily be used to model 
the time-dependent behaviour of additional systems. In fact, one could even 
easily argue that the complete investigation of the systems we have implemented 
ourselves in \lstinline{quantum_dots} is far from over. We have already begun
the implementation of a smoother two-dimensional double well potential, as well as 
more interesting well potentials such as the double double dot 
from \citeauthor{nielsen2012configuration}\cite{nielsen2012configuration}. Another 
idea is the construction of potential that are not circular-symmetric.
That said, we regrattably only had time to study \emph{one} of the
\emph{five} potentials we have implemented for the one-dimensional quantum dot.
The addition of a three-dimensional quantum dot system would also be essential 
in any further studies. Here, the integral elements provided by
\citeauthor{vorrath2003electronic}\cite{vorrath2003electronic} would be useful.

We would also like to see the addition of more exotic terms to the Hamiltian in the 
quantum dot systems. We think it would not be too difficult to add a spin-operator 
terms $\hat{S}_x$, $\hat{S}_y$, $\hat{S}_z$, spin-spin coupling terms
$\hat{S}\cdot\hat{S}$, and spin-orbit
coupling terms $\hat{J}^2$, $\hat{L}\cdot\hat{S}$. We belive that these would provide 
very interesting results and enables us to model more resplendent physical effects. We are 
confident that the 
mere addition of a $\hat{S}_z$ operator would enable us to see Zeeman splitting in 
the dipole spectrum of quantum dots, for instance. Eventually, one could hope to 
implement much more complicated operators, such as quantum logic gates, with the 
hope of consequent simulation of quantum circuitry.

Richer physics can also be modelled by the implementation of higher order 
multi-pole terms for the oscillating laser field. Moving beyond the allowed transitions
dictated by the dipole approximation could yield interesting results.

Currently, an article is in preparation for publication that rely on the 
software we have developed and the results we have arrived at in this thesis;
\citeauthor{kristiansen2019time}\cite{kristiansen2019time}.
