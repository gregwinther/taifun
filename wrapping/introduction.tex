\chapter{Introduction}

    The aim and raison d'étre of this thesis is to implement methods that enables the 
    study of many-body quantum systems in time. The word `ìmplement'' entails construction 
    of numerical solvers on a computer. Such computational modelling have since its 
    inception around the middle of the last century become present in all of the natural 
    sciences, and have since made its foray in the social sciences as well. In the 
    natural sciences, one could argue that computation plays as big a role as the two 
    conventional congregations of theory and experiment. In physics, computations are 
    a central component in a vast area of fields so diverse as
    quantum chromodynamics\cite{pederiva2018computing,vege2019solving},
    molecular dynamics\cite{stende2017constructing,treider2017speeding} and 
    astrophysics\cite{sand2016massive}.

    Quantum mechanics is the description of the behaviour of matter and light in all 
    its details, and in particular of the happenings on an atomic scale. A state of 
    a quantum system is described by a wavefunction $\Psi(\vb{r}, t)$, which provides 
    us with all there is to know about a particular system. We can determine
    $\Psi(\vb{r}, t)$
    at any point in the future by solving the Schrödinger equation, given the inital 
    state of the system $\Psi(\vb{r}, t_0)$,
    \begin{equation}
        i\hbar \frac{\partial}{\partial t} \Psi(\vb{r}, t) = \hat{H} \Psi(\vb{r}, t).
    \end{equation}
    Our ability to solve the Schrödinger equation analytically is constrained to only 
    a few quantum systems, and vanishes rapidly as the number of constituent particles in 
    such a system exceed just a few. It has therefore been necessary to extend the chalk and 
    the blackboard with numerics and a computer. The underlying theory for all of chemistry 
    is well-known and has been known for more than half a century, but the only element 
    we are able to solve on the blackboard is the the very simplest element, hydrogen.
    Adding an electron to the quantum hydrogren atom results in an analytically 
    unsolvable Schrödinger equation. As such, in order to \emph{solve chemistry} we 
    need numerical approximations.


\section{The Quantum Many-Body Problem}

    We have established that solving the Schrödinger equations exactly by hand is 
    impossible in the overwhelming majority of interesting cases. For this reason, 
    a plethora of computational, approximative methods have been developed aiming 
    to solve the many-body Schrödinger equation. Starting from first principles,
    or \emph{ab initio}, the goal of such algorithms is to procure some information
    about a quantum system in a reasonable amount of time. In order to accomplish 
    this, some sacrifices must be made in the form of simplifications. 

    The Hartree-Fock
    method\cite{hartree1928wave,fock1930naherungsmethode,szabo2012modern} which 
    has seen extensive use since its inception in 1930
    employs a mean-field appriximation, which provides an efficiently computed result, 
    but it is not very accurate. The most popular approximative method is without doubt 
    the density functional theory (DFT), developed by Kohn and Sham in
    1965\cite{kohn1965self}, which simplifies the quantum many-body problem by reformulating 
    it in terms of eletron number density. However, DFT is also insufficient if one requires a 
    high degree of accuracy.

    It is possible to solve a quantum many-body problem  ``correctly'' by direct diagonalisation 
    of the Hamiltionian, often called Full Configuration Interaction (FCI). Such a method 
    will yield an absolutely accurate result in the limit of an infinite orbital basis set,
    but it suffers from exponential complexity scaling\cite{helgaker2014molecular}.
    A sophisticated Monte-Carlo scheme, like Diffusion Monte-Carlo (DMC) can in principle 
    provide the exact solution to the Schrödinger equation by imaginary time-evolution 
    of an inital wave function ansatz\cite{hammond1994monte}. The high dependence of the 
    initial guess is problematic. What is more, the method often require the input of
    another less accurate method as a starting point. Another 
    example of a similar method is the Variational Monte-Carlo (VMC) method, which is
    simpler and faster than DMC, but not as accurate.

    Since the Hartree-Fock method is the earliest of methods designed to solve the quantum 
    many-body problem, it has become the purpose of all other many-body methods to describe 
    \emph{electron correlation}, defined as representing the difference between the 
    Hartree-Fock description of the electronic wavefunction and the exact solution of 
    the Schrödinger eqution. The simplest approach to treat the electron correlation
    is by performing a Configurations Interaction (CI) expansion,
    \begin{equation}
        \Psi = \Phi_\text{HF} 
            + \sum_{ia} C^i_a\Phi^a_i 
            + \sum_{\substack{i < j \\ a < b}} C^{ij}_{ab} \Phi^{ab}_{ij} + \dots,
    \end{equation}
    where $\Phi^a_i$ is a \emph{singly excited} configuration. We obtain the exact,
    Full Configuration Interaction, by including all terms in the CI expansion.
    Truncating the configurations interactions at a decided excitation may at first 
    seem like a good idea. It is common to perform such a truncation after the 
    double-excitation level, producing Configuration Interaction Singles Doubles (CISD).

    Two important properties that a many-body method must incorporate is \emph{size-extensivity}
    and \emph{size-consistency}. A quantum-mechanical model is said to be \emph{extensive}
    if the energy of the system computed with this model scales with the size of the 
    system\cite{bartlett1978many}. For systems of non-interacting helium atoms, for instance,
    we must have that $E(N\text{He}) = NE(\text{He})$. Moreover, a model is \emph{size-consistent}
    if the energies of two systems $A$ and $B$ and of the combined system $AB$, with 
    $A$ and $B$ very far apart, computed in equivalent ways, satify
    $E(AB) = E(A) + E(B)$\cite{pople1976theoretical}. It can be shown that 
    truncated CI does \emph{not} comply with the concept of
    extensivity\cite{shavitt2009many}.

    A class of methods of which most constituents provide proper \emph{extensive} model
    descriptions is Many Body Perturbation Theory (MBPT)\cite{brueckner1955approximate}.
    MBPT can also be truncated at a certain 
    excitation level, in the same manner as truncated Configuration Interaction. 
    The linked-diagram theorem\cite{goldstone1957derivation}, states that a 
    particular perturbation expansion can be expressed by ``linked terms''
    only\footnote{A thorough treatment of these terms are found in the end of
    \autoref{ch:perturbation}},
    leading to the Coupled Cluster method\cite{coester1958bound,coester1960short}.
    Coupled Cluster has become one of the most prevalent methods in quantum chemistry,
    as it is very accurate, whilst maintaining size-consistency and -extensivity.
    The Coupled Cluster method also has elegant truncation levels, similar to the 
    truncated Configuration Interaction method\footnote{Coupled Cluster and
    Configuration Interaction has an elegant correspondence,
    see \autoref{app:cc_vs_ci}}.

    Computational scaling of \emph{ab initio} quantum mechanical models range from 
    $\mathcal{O}(N!)$ for Full Configuration Interaction (FCI), via $\mathcal{O}(N^6)$ 
    for Coupled Cluster Singled Doubles with Perturbed Triples (CCSD(T)) to 
    $\mathcal{O}(N^4)$ for formal Hartree-Fock (HF)\cite{ratcliff2017challenges}.


\section{Goals}

    The specific task of this thesis is to study the time evolution of quantum 
    mechanical systems using the Coupled Cluster method. In particular this 
    pertains to the study of the time-dependent electronic systems, with an emphasis 
    on quantum dots. By use of proper object-orientation the implementation should 
    also be used to study simple atoms and molecules in three dimensions.
    The aim should be to 
    extend the formalism and algorithms developed in the thesis of 
    \citeauthor{haakon2017time}\cite{haakon2017time}.

    The these progression milestones have been split into the following steps,

    \begin{enumerate}
        \item Write a Coupled Cluster Doubles code with double exciations (CCD)
            capable of solving a system of two electrons in two or three dimensions 
            in a single harmonic oscillator well.
        \item Extend Coupled Cluster solver to include singles excitations (CCSD)
            and time evolution with static orbitals (TDCCSD). Moreover, expand system
            implementations to include more interesting systems, for example double
            potential wells or simple atoms.
        \item Include orbital dependencies in the time-dependent coupled cluster solver, as 
            discussed by \citeauthor{kvaal2012ab}\cite{kvaal2012ab}.
    \end{enumerate}

\section{Our Contributions}

    There already exists an abundance of optimised and well-tested ab inito quantum chemistry 
    software packages. As such we have not sought to build software meant to compete 
    with such software, but function as a supplement. As far as we know 
    there is currently no software that allows for time-dependent Coupled Cluster 
    computations in widespread use, making our contribution worthwhile. The interest 
    we have seen for our work at the Hylleraas centre at the University of Oslo 
    is testament to this\cite{islandwind2019numerically}.

    We have constructed two modules in the Python programming language that works
    well in conjunction, but 
    can also be used separately. The module \lstinline{coupled_cluster}, as the  
    name suggests contains all our Coupled Cluster solvers, both ground state and 
    time-dependent. The module \lstinline{quantum_systems} provides basis sets
    designed to work with the \lstinline{coupled_cluster} module. The 
    \lstinline{quantum_systems} module also contains functions for extracting basis sets 
    from the very popular quantum chemistry modules \lstinline{PySCF}\cite{PYSCF}
    and Psi4\cite{parrish2017psi4}. The source code for the \lstinline{coupled_cluster} module 
    can be found at
    \url{github.com/Schoyen/coupled-cluster}\footnote{At the time this document is 
    first printed, the source code may not be public. We plan to make the source code 
    open and public as soon as possible.} and documentation at 
    \url{www.coupledcluster.com}.
    The source code for the \lstinline{quantum_systems} module 
    can be found at
    \url{github.com/Schoyen/quantum-systems} and documentation at 
    \url{Schoyen.github.io/quantum-systems}.

    We have chosen Python as a development language for a couple of reasons. 
    First, even though Python is essently a  ``scripting language'', in the sense 
    that it is usually slower than a compiled language like \CC,
    most of our computations are matrix or tensor computations which are very fast
    in Python. We do matrix multiplications and linear algebra in NumPy, which 
    is very fast because of its use of Basic Liner Algebra Subprograms (BLAS) 
    and Linear Algebra PACKage (LAPACK).
    Both implemented in Fortran, BLAS consists of a low-level matrix and vector 
    arithmetic operations, while LAPACK is a collection of common algorithms 
    used in linear algebra such as routines for solving systems of linear 
    equations. For heavy operations that are not vectorisable, we have found
    just-in-time compilations with numba very beneficial.

    Second, Python is a high level language which has sped up development greatly 
    compared to developing in a lower level language. For this reason we 
    have a feeling that we have accomplished more than we would have 
    otherwise. The drawback of developing in a higher level language is the 
    lack of low level control, like memory management. 

\section{Structure of this Thesis}

    This thesis consists of five main parts. The first part, \emph{Fundamentals} 
    introduces the basis of quantum mechanics as well as the second quantisation 
    formalism.
    
    In the second part, \emph{Quantum Many-Body Approximation}, we 
    thoroughly present three many-body methods, each in its own chapter.
    We start with a chapter on \emph{Hartree-Fock Theory}, because this is 
    the first ab initio many-body method that saw truly widespread use, but 
    also because the Lagrange multiplier path we take to derive the method is 
    similar to what we will do later, when we talk about Lagrangian Coupled Cluster.
    We also introduce a time-dependent Hartree-Fock scheme here.
    Next, we present many-body \emph{Perturbation Theory}, which is often used as
    a supplement to other many-body methods, but it also highlights the origins 
    of the Coupled Cluster method, the topic of the next chapter.
    We derive both Brillouin-Wigner- and Rayleigh-Schrödinger Perturbation theory. 
    In the chapter on \emph{Coupled Cluster} we provide a detailed derivation
    of the Coupled Cluster equations, before detailing the problems caused 
    by the method's non-variational nature. This leads naturally to 
    the Lagrangian formulation of Coupled Cluster\cite{helgaker1988analytical}
    and the bivariational principle\cite{arponen1983variational}. Lastly,
    we introduce time-dependent Coupled Cluster theory, with both static and 
    adaptive orbitals\cite{kvaal2012ab}.

    In \emph{Implementation}, the third part of the thesis, we bestow upon the reader 
    a specification of the two python modules, \lstinline{quantum_systems} 
    and \lstinline{coupled_cluster}. Furthermore, we seek to explain the machinery
    in the two python modules, and their functionality. 

    In the fourth part of the thesis, we provide \emph{Results} generated with the
    coupled cluster solvers we have implemented. We start by a validation against
    some existing literature on time-dependent ab initio solvers. This includes 
    the reproduction of several results from relatively well-known articles. 
    The second and last chapter in this part contains the central results of 
    this thesis. Here we provide results from simulations of one- and two-dimensional 
    quantum dots, as well as two-dimensional quantum 
    dots acted on by an approximation of a magnetic field. Becuase this study 
    has been conducted in close conjunction with
    \citeauthor{islandwind2019}, more results that are computed with the 
    same solvers, relating to atoms and molecules can be found in Ref.
    \cite{islandwind2019}.
   
    Lastly, in the fifth and final section we impart a \emph{Conclusion} and 
    some suggestions for future work in the same field of study.
