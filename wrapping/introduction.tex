\chapter{Introduction}

    The aim and raison d'étre of this thesis is to implement methods that enables the 
    study of many-body quantum systems in time. The word `ìmplement'' entails construction 
    of numerical solvers on a computer. Such computational modelling have since its 
    inception around the middle of the last century become present in all of the natural 
    sciences, and have since made its foray in the social sciences as well. In the 
    natural sciences, one could argue that computation plays as big a role as the two 
    conventional congregations of theory and experiment. In physics, computations are 
    a central component in a vast area of fields so diverse as
    quantum chromodynamics\cite{pederiva2018computing,vege2019solving} to 
    molecular dynamics\cite{stende2017constructing,treider2017speeding}.

    Quantum mechanics is the description of the behaviour of matter and light in all 
    its details, and in particular of the happenings on an atomic scale. A state of 
    a quantum system is described by a wavefunction $\Psi(\vb{r}, t)$, which provides 
    us with all there is to know about a particular system. If we have a sufficient 
    initial condition for the wavefunctions we can determine $\Psi(\vb{r}, t)$
    at any point in the future by solving the Schrödinger equations,
    \begin{equation}
        i\hbar \frac{\partial}{\partial t} \Psi(\vb{r}, t) = \hat{H} \Psi(\vb{r}, t).
    \end{equation}
    Our ability to solve the Schrödinger equations analytically is constrained to only 
    a few quantum systems, and vanishes rapidly as the number of constituent particles in 
    such a system exceed just a few. It has therefore been necessary to extend the chalk and 
    the blackboard with numerics and a computer. The underlying theory for all of chemistry 
    is well-known and has been known for more than half a century, but the only element 
    we are able to \emph{solve} the very simples element, hydrogen, on the blackboard.
    Adding an electron to the quantum hydrogren atom results in an analytically 
    unsolvable Schrödinger equation. As such, in order to \emph{solve chemistry} we 
    need numerical approximations.


\section{The Quantum Many-Body Problem}

    We have established that solving the Schrödinger equations exactly by hand is 
    impossible in the overwhelming majority of interesting cases. For this reason, 
    a plethora of computational, approximative methods have been developed aiming 
    to solve the many-body Schrödinger equation. Starting from firt principles,
    or \emph{ab initio}, the goal of such algorithms is to procure some informations 
    about as quantum system in a reasonable amount of time. In order to accomplish 
    this some sacrifices must be made in the form of simplifications. 

    The Hartree-Fock
    method\cite{hartree1928wave,fock1930naherungsmethode,szabo2012modern} which 
    has seen extensive use since its inception in 1930
    employs a mean-field appriximation, which provides an efficiently computed result, 
    but it is not very accurate. The most popular approximative method is without doubt 
    the density functional theory (DFT), devloped by Kohn and Sham in
    1965\cite{kohn1965self}, simplifies the quantum many-body problem by reformulating 
    it in terms of eletron number density. DFT is also insufficient if one requires a 
    high degree of accuracy, however.

    It is possible to solve a quantum many-body problem correctly by direct diagonalisation 
    of the Hamiltionian, often called Full Configuration Interaction (FCI). Such a method 
    will yield an absolutely accurate result in the limit of an infinite orbital basis set,
    but it suffers from exponential complexity scaling\cite{helgaker2014molecular}.
    A sophisticate Monte-Carlo scheme, like Diffusion Monte-Carlo (DMC) can in principle 
    provide the exact solution to the Schrödinger equation by imaginary time-evolution 
    of an inital wave function ansatz\cite{hammond1994monte}. The high dependence of the 
    initial guess and often require the input of another less accurate method. Another 
    example of a similar method is the Variational Monte-Carlo (VMC) method, which is
    simpler and faster than DMC, but not as accurate.

    The improvement in the 

    \begin{itemize}
        \item Size-extensivity and -consistency.
        \item FCI really f-s things up.
        \item Why is CC good?
    \end{itemize}

\section{Goals}

    \paragraph{Methods}
    \begin{enumerate}
        \item Write a Coupled Cluster Solver including singles and doubles excitations.
        \item Extend Coupled Cluster sovler to include time evolution.
        \item Include orbital dependencies in the coupled cluster solver, as 
            discussed by \citeauthor{kvaal2012ab}\cite{kvaal2012ab}.
    \end{enumerate}

    \paragraph{Systems}
    \begin{enumerate}
        \item Harmonic quantum dots in one and two dimensions.
        \item Atoms and molecules.
        \item Magnetic fields. 
    \end{enumerate}

\section{Our Contributions}

    \begin{itemize}
        \item Two modules
        \item Why python? Speed of development. Actually quite fast. 
            Problems: memory control.
    \end{itemize}

\section{Structure of This Thesis}

Just look at contents.