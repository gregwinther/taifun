We advance the Coupled Cluster method's foray into the time-dependent realm, 
by implementing a robust solver based on the orbital-adaptive time-dependent 
Coupled Cluster (OATDCC)\cite{kvaal2012ab} method.
This involves implementing both a simplified static
orbital time-dependent coupled cluster solver with single and double excitations 
(TDCCSD) and an orbital-adaptive scheme with double excitations (OATDCCD).
To supplement the time-dependent methods we implement several ground state 
solvers based on the Lagrangian Coupled Cluster formulation, with single and double 
excitations, as well as a non-orthogonal orbital-optimised Coupled Cluster (NOCC)
solver\cite{myhre2018demonstrating}.

We construct several quantum dot basis sets with different potential functions
in one- and two dimensions, including interactions with magnetic fields.
What is more, we also implement an 
interface with popular quantum chemistry software modules PySCF\cite{PYSCF}
and Psi4\cite{parrish2017psi4} for extraction of additional 
basis sets for atoms and molecules. The quantum systems are allowed to vary with time 
by addition of a time-dependent addition to the Hamiltonian, with which we simulate 
a laser field in the dipole approximation.

As a validation we reproduce results from the scientific literature, both for 
atoms, molecules and quantum dots. We show that our methods leads to convergence in 
the ever-increasing basis set size limit, for simple quantum dot systems. For the 
same quantum dot system, we show how sensitive a system is to changes in the frequency
of 
a driving oscillating field. Frequencies closer to the resonant frequency 
leads to exctiations and increased energy. We are able to simulate systems that 
are fairly large - quantum dots in one- and two dimensions with up to twelve 
electrons. For systems that meander far from the reference state, we show that 
the orbital-adaptive method has far superior stability, compared with the 
method with static orbitals.

For all quantum dot system we find 
strong comformity with the harmonic potential theorem\cite{kohn1961cyclotron},
yet we see a slight many-body 
effect for a two-dimensional double dot system. By subjecting the two-dimensional
quantum dots 
to a homogoenous, static magnetic field in the form of an angular 
momentum operator, we see two frequencies in the dipole spectrum, instead of one 
frequency. This
is also in accordance with the harmonic potential theorem.
The difference between the two frequencies in the new spectrum is the same as the
Larmor frequency of the magnetic field, within acceptable tolerance levels. 