As a project such as this eventually comes to an end, for better or for
worse\footnote{I have always believed that even an existence with severe stress is preferred
compared to an existence without purpose.},
the custom is to devote some words in a message of gratitude. After all, 
there are people in one's life that make everything a bit more enjoyable through their
support - be it scientific or spiritual.

Firstly, I would like to thank my supervisor Morten. You welcomed me, willingly, with open arms,
and despite of my previous background in finance, into the computational physics group. I value your 
unterminable optimism - after a chat with you I always feel invigorated and motivated. You bring 
great inspiration to your students and others around you.

Words of emphatic thanks are also imparted in the direction of my serendipitous co-supervisor Håkon.
Your aid and insights have proven invaluable beyond a doubt. I consider myself very lucky to have
someone examine my 
work as closely as you have done. More than that, I am delighted to have a friend like you.

To my partner in despair, Øyvind, who has been invaluable in all respects. You have that one personal 
trait that I look up to with high regard: unbounded resolve and grit that makes for true
strength of character. I have found it motivating to work beside you and I am thankful for your 
friendship over the the many months (years!) we have struggled with very difficult physics together.

I would like to give my regards to the amazing people that make up the computational physics graduate
students.
If nothing more, you definitely make for the most enjoyable luncheons that oft-times include
pristine
philosphical discussions. The tales of the camaraderie of this gang of nonconformists will 
undoubtedly become material for an operette or at least some form of low-budget musical theatre.

I am also thankful to the two troupes of troubadours I have been affiliated with during my time as a student 
at the University of Oslo. It was in the ranks\footnote{I am not sure if ``rank'' is the best word, because 
if Biørneblæs were even able to line up, it would more closely resemble a pile of overcooked spaghetti
than ranks and files.} of Studentorchestert Biørneblæs where I first 
picked up the saxophone, and here I found tremendous musical joy. Eventually I was head-hunted to 
become Il Maestro of Blindern Haarn oc Blaese. An event that actually \emph{was}
the basis of some form of very low-budget musical theatre. One should never let the (low) quality
of the music be hindrance to a great orchestral experience.

To Vilde: I feel fortunate to have someone like you as my companion. The adventures we've had make up
some of my most treasured memories, and I hope we will have many more together. Thank you for your 
patience, love and encouragement throughout these extra-ordinary months.

\paragraph{Collaboration details}

The software that has been developed as part of this thesis was developed in joint 
effort between \citeauthor{islandwind2019}\cite{islandwind2019} and myself. As such,
it exists as only one product, as we saw fit to join forces instead of manufacturing two 
duplicates of the same invention.