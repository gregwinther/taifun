\chapter{Quantum Dots}

Here we present results related to time-dependent simulations of parabolic quantum
wells. We have simulated the behaviour of such quantum dots in both one- and two 
dimensions, producing time dependent energies and ground state probabilities over
time as the system is under the influence of oscilalting fields. We also present 
dipole spectra of the one- and two-dimensional quantum dot as well as the 
two-dimensional double dot and a two-dimensional double dot under the influence of
a homogenous, static magnetic field. We find that the \emph{harmonic potential theorem}
holds for all simulations.

The \emph{harmonic potential theorem}\cite{kohn1961cyclotron}
states that electrons trapped in a parabolic quantum well shows behaviour as if it 
was one large quantum oscilaltor, instead of consisting of many smaller parts. This 
includes exhibiting only one frequency it the dipole spectrum of the system. If one 
were to compute the Fourier transform of the dipole of an $n$-electron quantum dot with 
parabolic potential the result would be one line in the spectrum corresponding to the
oscillator frequency of the spectrum. This means that there are no many-body effects 
in a harmonic quantum dot.

An extension of the theorem to systems under the influence 
of a magntic field\cite{brey1989optical}. State that one would expect to see a shift,
both up and down, creating two frequencies $\Omega_+$ and $\Omega_-$ in the dipole 
spectrum. The resulting dipole spectrum would show two frequencies with a difference 
equalling the Larmor frequency $\omega_c = \Omega_+ - \Omega_-$.

\section{Harmonic Oscillators in One Dimension}

For the one-dimensional quantum dot with a harmonic potential we simulate a laser by 
adding an oscillating 
field with a sinusoidal envelope, similar to the one in
\citeauthor{pedersen2019symplectic}\cite{pedersen2019symplectic},
\begin{equation}
    \vb{e}(t) = \vb{E}_\text{max}\sin^2\left(\frac{t\pi}{T}\right)\cos(\omega t).
\end{equation}
We set the period of the envelope equal to the duration of the entire simulation,
$T=20$, so that we have a field that at first will gradually increase, then decrease.
The oscillator frequency for all simulations are set to $\Omega=1$, and at first we 
set the frequency of the oscillating field to twice this, $\omega = 2$. We do this 
to make sure that we are far from the resonant frequency of the system, and we pick 
relatively high laser frequency in order to enforce a more dynamic system. We use the 
more standard time-dependent coupled cluster singles doubles (TDCCSD) method, for 
these simulations, with the symplectic Gaussian integrator and a time step of $dt=0.01$.
The simulations are performed for an increasing number of electrons
$n = \{2,4,6,8,10,12\}$. We have computed the energy and the time-dependent overlap,
i.e. the time-dependent probability of being in the ground state, for each simulation.
We repeat the simulations for a wide range of different number of spin-orbitals 
to the convergent properties of the simulations as the number of spin-orbitals increase.

\begin{figure}[ht]
    \centering
    \includegraphics[width=0.75\textwidth]{results/figures/1D/n=2energy.png} 
    \caption{Time-dependent energy of a one-dimensional harmonic oscialltor 
        with $n=2$ electrons
        under the influence of a laser field for different number of spinorbitals
        $l\in\{6,8,10,12,20\}$.
    }
    \label{fig:1d_n2_E}
\end{figure}

First we study the time-dependent energy of a quantum dot acted upon by an oscillating 
field. The result for $n=2$ electrons is shown in \autoref{fig:1d_n2_E}. We have 
produced comparative results for other number of particles $n=\{4,6,8,10,12\}$,
which can be found in \autoref{app:1d_qd}. We see an apparent convergence in the 
time-dependent energy as we increase the number of spin-orbitals in the basis set.
For larger systems with more electrons it reasonably becomes necessary to also increase
the size of the basis set. As is the tendency with ground state coupled cluster computations
for quantum dots\cite{jorgensen2011many,lohne2010coupled}, the time dependent energy 
of a quantum dot is decreasing until convergence for increasing basis set size.

\begin{figure}[ht]
    \centering
    \includegraphics[width=0.75\textwidth]{results/figures/1D/n=2overlap.png} 
    \caption{Probability of being in the ground state $|\braket*{\Psi(0)}{\Psi(t)}|^2$
        for a one-dimensional quantum dot with $n=2$ electrons under 
        the influence of a laser field for different number of spinorbitals 
        $l\in\{6,8,10,12,20\}$.
    }
    \label{fig:1d_n2_overlap}
\end{figure}

\begin{figure}[ht]
    \centering
    \includegraphics[width=0.75\textwidth]{results/figures/1D/n_compare_overlap.png} 
    \caption{Probability of being in the ground state for $|\braket*{\Psi(0)}{\Psi(t)}|^2$
        for a one-dimensional quantum dot for different number of electrons 
        $n\in\{2,4,6,8,10,12\}$.
    }
    \label{}
\end{figure}

\begin{figure}
    \centering
    \makebox[\textwidth][c]{
    \includegraphics[width=1.4\textwidth]{results/figures/1D/1d_spectrum.png} 
    }
    \caption{Fourier transform of expected value of dipole moment for 
        a one-dimensional quantum dot with different number of electrons
        $n\in\{2,4,6,8,10\}$.
    }
    \label{}
\end{figure}


\section{Magnetic Field}

We start the study of two-dimensional quantum dots under the influence of a magnetic 
field by defining a system of only one particle and solving the time-dependent 
Schrödinger equation
directly. This is a accomplished by using the \lstinline{TwoDimHarmonicOscB} class
to produce a basis set, single-particle functions and transition/intearction matrix (dipole elements),
which is everything we need. All of these items are properties of the class and can be
easily extracted. A simple periodic function simulates an electric field is constructed, as 
the product of such a time-dependent operator and the interaction matrix defines the 
time propagation. We then use a simple integration scheme, in this case the fourth-order 
Runge-Kutta method, to propagate the ground state single particle function of the system.
Taking care to extract the dipole for every time step, we can compute the discrete Fourier 
transform of the dipole and compute the frequency spectrum of our system. This procedure is 
applied to a system comletely absent of a magnetic field, and a system under direct influence 
of a magnetic field.

Before going straight to the results, we study the shell structure and allowed transitions of 
our two systems. The left part of \autoref{fig:shell_structure_yes_no_b} presents the 
shell structure of a the regular two-dimensional quantum dot. The states have all been assigned
a number for easier examination. This shell structure is 
identical to the one presented in \autoref{fig:2d_basis_states}. Additionally, here we have added
coloured double arrows to illustrate the allowed transitions in the quantum dot. These 
transitions are can be encountered in the transition matrix for the system, which is
reproduced in the artistic way in \autoref{fig:transition_no_b}. Notice that the coloured 
arrows representing allowed transitions match in colour with the elements of the transition 
matrix.

\begin{figure}
    \begin{center}
    \begin{tikzpicture}[scale=0.9, background rectangle/.style={fill=grey},
        show background rectangle]
    \begin{scope}
      
        % TOP
        \foreach \i in {0, 3, 6} {
            \draw(\i, 2) -- (\i + 2, 2);
            \node at (\i + 0.75, 2) {$\uparrow$};
            \node at (\i + 1.25, 2) {$\downarrow$};
        }
       
        \node[below, inner sep=.5em] at (1, 2) {$(0, -2)$};
        \node[above] at (1, 2) {5};
        \node[below, inner sep=.5em] at (4, 2) {$(1, 0)$};
        \node[above] at (4, 2) {3};
        \node[below, inner sep=.5em] at (7, 2) {$(0, 2)$};
        \node[above] at (7, 2) {4};

        % MIDDLE
        \foreach \i in {1.5, 4.5} {
            \draw(\i, 1) -- (\i + 2, 1);
            \node at (\i + 0.75, 1) {$\uparrow$};
            \node at (\i + 1.25, 1) {$\downarrow$};
        }

        \node[below, inner sep=.5em] at (2.5, 1) {$(0, -1)$};
        \node[above] at (2.5, 1) {1};
        \node[below, inner sep=.5em] at (5.5, 1) {$(0, 1)$};
        \node[above] at (5.5, 1) {2};

        % BOTTOM
        \draw(3, 0) -- (5, 0);
        \node at (3 + 0.75, 0) {$\uparrow$};
        \node at (3 + 1.25, 0) {$\downarrow$};

        \node[below, inner sep=.5em] at (4, 0) {$(0, 0)$};
        \node[above] at (4, 0) {0};

        % Transitions
        \draw [<->, line width=.1em, transition11] (3.5, 0) -- (3.25, 1);
        \draw [<->, line width=.1em, transition11] (4.5, 0) -- (4.75, 1);

        \draw [<->, line width=.1em, transition12] (3, 1) -- (3.25, 2);
        \draw [<->, line width=.1em, transition12] (5, 1) -- (4.75, 2);

        \draw [<->, line width=.1em, transition13] (2, 1) -- (1.75, 2);
        \draw [<->, line width=.1em, transition13] (6, 1) -- (6.25, 2);

    \end{scope}

    \begin{scope}[xshift=20em]

        % MIDDLE 2
        \foreach \i in {1.5, 4.5} {
            \draw(\i, 3) -- (\i + 2, 3);
            \node at (\i + 0.75, 3) {$\uparrow$};
            \node at (\i + 1.25, 3) {$\downarrow$};
        }

        \node[below, inner sep=.5em] at (2.5, 3) {$(1, 0)$};
        \node[above] at (2.5, 3) {3};
        \node[below, inner sep=.5em] at (5.5, 3) {$(0, 3)$};
        \node[above] at (5.5, 3) {6};

        % MIDDLE 1
        \foreach \i in {1.5, 4.5} {
            \draw(\i, 2) -- (\i + 2, 2);
            \node at (\i + 0.75, 2) {$\uparrow$};
            \node at (\i + 1.25, 2) {$\downarrow$};
        }

        \node[below, inner sep=.5em] at (2.5, 2) {$(0, -1)$};
        \node[above] at (2.5, 2) {1};
        \node[below, inner sep=.5em] at (5.5, 2) {$(0, 2)$};
        \node[above] at (5.5, 2) {4};

        % BOTTOM 2
        \draw(3, 1) -- (5, 1);
        \node at (3 + 0.75, 1) {$\uparrow$};
        \node at (3 + 1.25, 1) {$\downarrow$};

        \node[below, inner sep=.5em] at (4, 1) {$(0, 1)$};
        \node[above] at (4, 1) {2};
        
        % BOTTOM 1
        \draw(3, 0) -- (5, 0);
        \node at (3 + 0.75, 0) {$\uparrow$};
        \node at (3 + 1.25, 0) {$\downarrow$};

        \node[below, inner sep=.5em] at (4, 0) {$(0, 0)$};
        \node[above] at (4, 0) {0};
        
        % Transitions 

        % 0 -> 1
        \draw [<->, line width=.1em, transition21] (3.5, 0) to[out=120, in=-90] (1.75, 2);
        % 0 -> 2
        \draw [<->, line width=.1em, transition21] (4.5, 0) -- (4.5, 1);

        % 2 -> 3
        \draw [<->, line width=.1em, transition23] (3.5, 1) to[out=60, in=-50] (3.25, 3);
        % 2 -> 4
        \draw [<->, line width=.1em, transition22] (4.75, 1) -- (4.75, 2);

        % 1 -> 3
        \draw [<->, line width=.1em, transition23] (2, 2) -- (2, 3);
        % 4 -> 6 
        \draw [<->, line width=.1em, transition24] (6, 2) -- (6, 3);

    \end{scope}
    \end{tikzpicture}
    \end{center} 
    \caption{Shell structure of six lowest orbitals before (left), and after (right)
        a magnetic field is applied to a 2D quantum dot.}
    \label{fig:shell_structure_yes_no_b}
\end{figure}

When we apply apply a magnetic field of strength $\omega_c/\omega = \sqrt{2}/2$ we 
obtain the shell structure represented to the right in 
\autoref{fig:shell_structure_yes_no_b}, where the allowed transitions correpsond to the 
transition matrix in \autoref{fig:transition_yes_b}. The chosen magnetic field strength 
was not chosen arbitrarily, as these accidental degenracies occur only rarely as 
a function of magnetic field strength\footnote{Hence the term ``accidental''.}.
For succinctness we repeat the function for energy eigenvalues for two-dimensional 
quantum dot influenced by a magnetic field (\autoref{eq:2d_b_eigenvalues}),
\begin{equation}
    \epsilon_{nm} = \hbar\Omega(2n + |m| + 1) - \frac{\hbar\omega_c}{2}m,
\end{equation}
where $\Omega = \sqrt{\omega_0^2 + \frac{\omega_c^2}{4}}$.
Apart from a general shift up in energy by adding a magnetic field, the states with 
negative azimuthal quantum number $m$ will experience an increase in energy eigenvalue,
and vice versa. We see this effect clearly in the new shell structure in
\autoref{fig:shell_structure_yes_no_b}. The states with negative $m$ have indeed
undergone a relative shift upwards, whilst the states with positive $m$ have been
shifted downwards, relative to the other states. The ground state, labelled 0, 
remains relatively stationary, the states labelled 2 ($m=1$) and 4 ($m=2$) have been 
shifted downwards and the states labelled 1 ($m=-1$) and 5 ($m=-2$) have been shifted 
upwards. State number 5 so much that it has disappeared from the shell structure, with a 
new state 6 ($=3$) appearing. This is due to our restriction to include only the six
lowest-energy orbitals. We see that the possible remaining allowed transitions remain the 
same, with the exception of transitioning between state 1 and 5, because state 5 is no more,
and the addition of a possible transition between state 4 and 6.

\begin{figure}
    \begin{center}
    \begin{minipage}{0.49\textwidth}
        \centering
        \includegraphics[width=\textwidth]{results/figures/dipole_no_b.png}
        \caption{Transition matrix dictating the allowed transitions for a
            2D quantum dot.}
        \label{fig:transition_no_b}
    \end{minipage}
    \begin{minipage}{0.49\textwidth}
        \includegraphics[width=\textwidth]{results/figures/dipole_yes_b.png}
        \caption{transition matrix for a 2D quantum dot when a magnetic field
            is applied.}
        \label{fig:transition_yes_b}
    \end{minipage}        
    \end{center}
\end{figure}

If we compute the frequency spectrum of the two systems
(\autoref{fig:transmission_spectrum_b_field}) we get a single line for the 
normal quantum dot. This is expected, as the quantum harmonic oscillator has 
the same energy difference between each level. However, when we apply a magnetic
field and shift the energies of the orbitals in the quantum dot, we see that we 
get two different energy transitions. This is revealed as two lines in the 
frequency spectrum in \autoref{fig:transmission_spectrum_b_field}. This is equivalent 
to a splitting in transmisstion spectra of quantum dot arrays under the effect 
of a magnetic field in experiments\cite{heitmann1993spectroscopy,meurer1992single}.

\begin{figure}
    \centering
    \includegraphics[width=0.75\textwidth]{results/figures/transmission_spectrum.png}
    \caption{Spectrum of a 2D quantum dot both with and without a magnetic field.}
    \label{fig:transmission_spectrum_b_field}
\end{figure}
