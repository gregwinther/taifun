\chapter{Benchmarks}

Here we present a series of reproduced results from the scientific literature as 
a validation of our computational implementation. We manage to reproduce the 
instantaneous dipole results from the simulation of the 
hydrogen molecule in \citeauthor{li2005time}\cite{li2005time},
the time-dependent ground state probability of a quantum dot from 
\citeauthor{Zanghellini04}\cite{Zanghellini04}
and the spectrum of Helium from
\citeauthor{pedersen2019symplectic}\cite{pedersen2019symplectic}.
The simulation of the ionisation of beryllium 
from ????INSERT REFERENCE????, serves as an illustration of the advantage of
adaptive orbitals versus static orbitals in a time-dependent coupled cluster method.


\section{Instantaneous dipole in $H_2$}

\citeauthor{li2005time}\cite{li2005time} employ a time-dependent Hartree-Fock 
apporach in order to study the electronic optical response of molecules 
in intense fields. To be specific, they model the hydrogen molecule $H_2$ 
with a \lstinline{6-311++G(d,p)} basis set, subject to an oscillating field 
of $1.72\times10^13\text{ W} \text{cm}^{-2}$ and $456\text{ nm}$. They find the time-dependent
Hartree-Fock method to be nearly indistinguishable from calculations using the 
full time-dependent Schrödinger equation. We have managed to replicate the 
instantaneous dipole of this simulation of hydrogen.

A \lstinline{6-311++G(d,p)} basis set corresponds to a \lstinline{6-311++Gss} 
basis set in \lstinline{PySCF}, and we can extract it from here,
\begin{python}
molecule = "
    h 0.0 0.0 -0.6948522960236121;
    h 0.0 0.0  0.6948522960236121
    "
basis = "6-311++Gss"
system = construct_pyscf_system_ao(molecule, basis=basis)
\end{python}
The bond length of the Hydrogen molecule is approximately $0.7354\text{ Å}$, converted 
to multiples of Bohr radii here. As the naming suggests, the basis set is a split-valence 
triple-zeta basis set, with one added s-type diffuse function and a set of p-type
polarisation 
functions for each Hydrogen atom\footnote{This would be obvious for a quantum chemist,
but basis set configurations looks like incantantions from a spellbook to a physicist}.

In their simulations \citeauthor{li2005time} have used a linearly polarised and 
spatially homogenous external field, 
\begin{equation}
    \vb{e}(t) = \vb{E}(t)\sin(\omega t).
\end{equation}
The field envelope $|\vb{E}|$ is linearly increased with time to a maximum value 
$|\vb{E}_\text{max}|$ at the end of the first cycle and remains at $|\vb{E}_\text{max}|$
for one cycle and then decreases linearly to zero by the end of the next cycle,
\begin{equation}
    \begin{aligned}
        \vb{E}(t) = (\omega t / 2\pi) \vb{E}_\text{max} \quad &\text{for} \quad
            0 \leq t \leq 2\pi / \omega \\ 
        \vb{E}(t) = \vb{E}_\text{max} \quad &\text{for} \quad 
            2\pi / \omega \leq t \leq 4\pi / \omega \\ 
        \vb{E}(t) = (3 - \omega t / 2\pi) \vb{E}_\text{max} \quad &\text{for} \quad 
            4\pi / \omega \leq t \leq 6\pi / \omega \\
        \vb{E}(t) = 0 \quad &\text{for} \quad
            t < 0 \text{ and } t > 6\pi / \omega,
    \end{aligned}
\end{equation}
where the maximum field intensity if $1.72\times10^14 \text{ W} \text{cm}^{-2}$ 
($E_\text{max} = 0.07 \text{ au}$). \citeauthor{li2005time} also run a simulation 
for a lower intensity, but we are conecerned only with this relatively more intensive 
pulse. The entire simulation lasts for $T=225 \text{ au}$.

The result of our simulation is shown in \autoref{fig:li_compare}, where we have 
computed the instantaneous dipole over time using three different methods. The 
time-dependent Hartre-Fock result is shown in the bottom sub-figure, and is expected 
to be exactly the same as figure 4.a from \citeauthor{li2005time}\cite{li2005time},
which it appear to be. For comparrison we have computed the result with both of 
our time-dependent coupled cluster methods. The result of the time-dependent 
coupled cluster method with single and double excitations are showed in the top subfigure,
and the result of the orbital-adaptive coupled cluster method with double excitations 
are shown in the middel subfigure. We see that there is no perceptible difference between 
the results of the three methods.

\begin{figure}
    \centering
    \includegraphics[width=0.75\textwidth]{results/figures/li_compare.png}
    \caption{Instantaneous dipole for $H_2$ in an oscillating electric field
        $E_\text{max} = 0.07 \text{ au}$ ($1.72\times10^14 \text{ W} \text{cm}^{-2}$)
        and $\omega=0.1 \text{ au}$ ($456\text{ nm}$) using a \lstinline{6-311++G(d,p)}
        basis set.
    }
    \label{fig:li_compare}
\end{figure}


\section{Ground State Probability in 1D Quantum Dot}

\citeauthor{Zanghellini04}\cite{Zanghellini04} calculate the time development of a 
one-dimensional quantum dot with two electrons using the multi-configurational 
time-dependent Hartree-Fock method (MCTDHF). This method yields excact results for 
a very large number of configurations, $\eta \to \infty$. This study would provide a 
proper benchmark for our implementation because the coupled cluster method with singles and 
doubles excitations (CCSD) is excact for $n=2$ particles. 
The harmonic oscillator potential applied in
their study had a frequency of $\Omega=0.25$, used a strong laser-like field with 
maximum intensity of $\vb{E} = 1$ and a laser frequency of $\omega = 8 \Omega = 2$.
The oscillating field is described much more simply than in
\citeauthor{li2005time}\cite{li2005time}, using a simple sine function,
\begin{equation}
    \vb{e}(t) = \vb{E}\sin(\omega t),
\end{equation}
where the envelope $\vb{E}$ does not vary in time.

\citeauthor{Zanghellini04}\cite{Zanghellini04} find that their multi-configurational time-dependent
Hartree-Fock scheme convergences as the number of configurations is 
$\eta \geq15$, up to the resulotion of their figures.
We are able to reproduce this result precisesly by employing the 
time-dependent coupled cluster method with singles and double excitations (TDCCSD).
We have used our own one-dimensional quantum dot class, \lstinline{ODQD}, with 
a harmonic potential and $l=20$ spin-orbitals
in the basis set for this simulation.

\begin{figure}
    \centering
    \includegraphics[width=0.75\textwidth]{results/figures/zanghellini_fig1.png}
    \caption{
        \label{fig:zanghellini_fig1}
        Electron density for the ground state wavefunction of a quantum dot with 
        $n=2$ eletrons and $l=20$ spin-orbitals in the basis set computed with
        CCSD. This plot 
        corresponds precisely with figure 1 in
        \citeauthor{Zanghellini04}\cite{Zanghellini04}.
    }
\end{figure}

\begin{figure}
    \centering
    \includegraphics[width=0.75\textwidth]{results/figures/zanghellini_fig2.png}
    \caption{
        \label{fig:zanghellini_fig2}
        Probability of being in the ground state $|\braket{\Phi(0)}{\Phi(t)}|$
        using both TDHF and TDCCSD, for a one-dimensional quantum dot with $n=2$
        particles and $l=20$ spin-orbitals. This plot corresponds precisely with 
        figure 2 in \citeauthor{Zanghellini04}\cite{Zanghellini04}.
    }           
\end{figure}

In \autoref{fig:zanghellini_fig1} we see the ground state electron density for the 
ground state wavefunction computed with CCSD. \citeauthor{Zanghellini04} computed the electron
density for an increasing number of configurations $\eta$ using multi-configurational
Hartree-Fock (MCHF). This figure matches the convergent electron density found by
\citeauthor{Zanghellini04} as $\eta \to \infty$, in figure 1 from their article. 

\autoref{fig:zanghellini_fig2} depicts the probability fo the system being in the ground 
state as a function of time. Here we have included both a time-dependent Hartree-Fock
computation, corresponding to a multi-configurational time-dependent 
Hartree-Focke computation with $\eta=1$ configurations, and 
a time-depenedent coupled cluster computation with single and double excitations.
We see that our coupled cluster scheme corresponds to the multi-configurational Hartree-Fock 
scheme employed by \citeauthor{Zanghellini04} when $\eta\to\infty$, as
\autoref{fig:zanghellini_fig2} match figure 2 in
\citeauthor{Zanghellini04}\cite{Zanghellini04} precicely.


\section{Dipole Spectrum of Helium}

In their comparison study of symplectic and regular Runge-Kutta type integrators, 
\citeauthor{pedersen2019symplectic}\cite{pedersen2019symplectic} produce a dipole 
spectrum of helium. 

The basis set employed by \citeauthor{pedersen2019symplectic} is a cc-pVDZ 
basis set which we extract from \lstinline{Psi4},
\begin{python}
He = "
    He 0.0 0.0 0.0
    symmetry c1
"
options = {"basis": "cc-pvdz", "scf_type": "pk", "e_convergence": 1e-8}
system = construct_psi4_system(He, options)
\end{python}
The \lstinline{cc-pVDZ} basis set is a correlation consistent, polarised, valence-only 
basis set with double zeta-functions. For hydrogen this basis set amounts to five 
orbitals in total.

In their study \citeauthor{pedersen2019symplectic} use an oscillating field with 
frequency $\omega=2.8735643 \text{ au}$ and maximum intensity
$\vb{E}_\text{max} = 10 \text{ au}$. This frequency corresponds to the lowest-lying 
electric-dipole allowed transition from the ground state of helium. The oscillating 
field can be described as 
\begin{equation}
    \vb{e}(t) = \vb{E}(t) \cos(\omega t),
\end{equation}
with a sinusoidal envelope
\begin{equation}
    \vb{E}(t) = \vb{E}_\text{max}\sin^2\left(\frac{\pi t}{t_d}\right) H(t_d - t),
\end{equation}
where $H$ is the Heaviside step function designed to return zero when the field 
has reached its designated halting time $t_d$. This envelope is similar in behaviour 
to the one in the study by \citeauthor{li2005time}\cite{li2005time} - it 
increaes gradually at first, and then gradually decreases. 

The oscillation field is only meant to ``disturb'' the ground state of the atom,
as it is quickly 
switched off at $t_d=5$. Then the system is allowed to propagate in time for 
a long period. In our reproduction of the system, we have let the system 
evolve for a total time $T=1500 \text{ au}$. For each time step we compute the 
dipole in the same direction as the polarisation of the oscillating field. The 
fourier transform of this signal will then yield the dipole spectrum of the 
atom. The time-development is performed with the orbital-adaptive time-dependent 
coupled cluster method with double excitations. The result from this simulation is 
depicted in \autoref{fig:helium_spectrum}, which is qualitatively equal to figure 
7 in \citeauthor{pedersen2019symplectic}\cite{pedersen2019symplectic}.

\begin{figure}
    \centering
    \includegraphics[width=0.75\textwidth]{results/figures/helium_spectrum.png} 
    \caption{Dipole spectrum of He at field strength $10\text{ au}$ using 
        OATCCCD and a \lstinline{cc-pVDZ} basis set.      
    }
    \label{fig:helium_spectrum}
\end{figure}
