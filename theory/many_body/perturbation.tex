\chapter{Perturbation Theory}

Perturbation theory is a very powerful method and a generic method applicable to all 
matrix problems. Additionally, perturbation theory is relatively cheap in terms of computing
time especially compared with coupled cluster theory. As the method provides a different 
route to the solution of the Schrödinger equation, by approaching the exact solution 
systematically, based on an order-by-order expansion of the energy and wave function. 
Therefore, perturbation theory is often used to improve the results from other computation 
schemes. What is more, the exponential form of the wave function in coupled cluster theory 
stems from the non-degenerate Rayleigh-Schrödinger perturbation theory (RSPT) expansion.

\section{Formal perturbation theory}

We split the Hamiltonian into a known part and a perubed part,
\begin{equation}
    \hat{H} = \hat{H}_0 + \hat{V}.
\end{equation}
Sometimes it is convenient to write
\begin{equation}
    \label{eq:schrodinger_w_order_param}
    \hat{H} = \hat{H}_0 + \lambda\hat{V},
\end{equation}
where we have included an "order parameter" $\lambda$. This parameter is used to categorise
the contributions of different order. 
The exact solution is given by
\begin{equation}
    \label{eq:perturbation_schrodinger}
    \begin{gathered}
        \hat{H} \Psi_n = E_n \Psi_n \\
        (\hat{H}_0 + \hat{V}) \Psi_n = E_n \Psi_n, \quad \Psi_n = \Phi_n + \chi_n
    \end{gathered}
\end{equation}
while the solvable and simple zero order problem is given by
\begin{equation}
    \label{eq:zero_order_perturbed_schrodinger}
    \hat{H}_0 \Phi_n = E_n^{(0)} \Phi_n
\end{equation}

By projecting \autoref{eq:perturbation_schrodinger} with $\bra{\Phi_0}$ we get
\begin{equation}
    \begin{gathered}
    \bra{\Phi_n} \hat{H}_0 \ket{\Psi_n} + \bra{\Phi_n} \hat{V} \ket{\Psi_n}
        = E_n \braket{\Phi_n}{\Psi_n} \\
    \to E_n = \bra{\Phi_n} \hat{H} \ket{\Psi_n} \\
    \to \Delta E_n = E_n - E_n^{0} = \bra{\Phi_n} \hat{V} \ket{\Psi_n}
    \end{gathered}
\end{equation}
where we have used that 
\begin{gather}
    \label{eq:intermediate_normalisation}
    \braket{\Phi_m}{\Phi_n} = \delta_{mn} \\
    \braket{\Psi_n}{\Phi_n} = \braket{\Phi_n + \chi_n}{\Phi_n} = 1 \\
    \braket{\Psi_n} = 1 + \braket{\chi_n}.
\end{gather}
This is called the intermediate normalisation assumption.

\subsection{Energy- and Wavefunction Expansion}

We now have need for the order parameter from $\lambda$ \autoref{eq:schrodinger_w_order_param}
as we expand the wavefunction and energy,
\begin{equation}
    \begin{aligned}
        \Psi_n &= \Phi_n + \chi_n = \Psi_n^{(0)} + \lambda\Psi_n^{(1)} 
            + \lambda^2\Psi_n^{(2)} + \dots \quad (\Psi_n^{(0)} \equiv \Phi_n) \\
        E_n &= E_n^{(0)} + \Delta E_n = E_n^{(0)} + \lambda E_n^{(1)} 
            + \lambda^2 E_n^{(2)} + \dots 
    \end{aligned}
\end{equation}
We insert these expansions into the Schrödinger equation,
\begin{equation}
    \begin{aligned}
        (\hat{H} - E_n) \Psi_n = 0 \\
        (\hat{H}_0 + \lambda \hat{V})\Psi_n = 0,
    \end{aligned}
\end{equation}
resulting in 
\begin{equation}
    (\hat{H}_0 + \lambda \hat{V} - E_n^{(0)} - \lambda E_n^{(1)} 
        -\lambda^2 E_n^{(2)} - \dots)
    (\Psi_n^{(0)} + \lambda \Psi_n^{(1)} + \lambda \Psi_n^{(2)} + \dots) = 0.
\end{equation}
Now we gather the coefficients of different powers of $\lambda$,
\begin{align}
    (\hat{H}_0 - E_n^{(0)})\Psi_n^{(0)} &= 0 \\
    \label{eq:perturbation_lambda_power_first}
    (\hat{H}_0 - E_n^{(0)})\Psi_n^{(1)} &= (E_n^{(1)} - \hat{V})\Psi_n^{(0)}  \\
    (\hat{H}_0 - E_n^{(0)})\Psi_n^{(2)} &= (E_n^{(1)} - \hat{V})\Psi_n^{(1)}
        + E_n^{(2)}\Psi_n^{(0)} \\
        &\dots \nonumber \\
    \label{eq:perturbation_lambda_power_mth}
    (\hat{H}_0 - E_n^{(0)})\Psi_n^{(m)} &= (E_n^{(1)} - \hat{V})\Psi_n^{(m-1)}
        + \sum_{l=0}^{m-2} E_n^{(m-l)}\Psi_n^{(l)}.
\end{align}
Where the last line gives a general $m$th-order equation. This equation can be 
simplified somewhat,
\begin{equation}
    (E_n^{(0)} - \hat{H}_0)\Psi_n^{(m)} = \hat{V}\Psi_n^{(m-1)} 
        - \sum_{l=0}^{m-1}E_n^{(m-l)}\Psi_n^{(l)}.
\end{equation}
By applying $\bra{\Phi_n}$ to each of the equations, we get expressions 
for $E_n^{(m)}$. 
For $\lambda^1$ (\autoref{eq:perturbation_lambda_power_first}) we get,
\begin{equation}
    \begin{gathered}
    \bra{\Phi_n} \hat{H}_0 - E_n^{(0)}\ket{\Psi_n^{(1)}}
        = \bra{\Phi_n} E_n^{(1)} - \hat{V}\ket{\Phi_n} \\
    \bra{(\hat{H}_0 - E_n^{(0)})\Phi_n} \ket{\Psi_n^{(1)}}
        = \bra{\Phi_n} E_n^{(1)} - \hat{V}\ket{\Phi_n} \\
    \to E_n^{(1)} = \bra{\Phi_n} \hat{V} \ket{\Phi_n} = \hat{V}_{nn}
    \end{gathered}
\end{equation}
Since we have an expression for $E_n^{(1)}$, we can solve the inhomogeneous 
differential equation for $\Psi_n^{(1)}$, by also requiring the intermediate
normalisation condition $\braket{\Phi_n}{\Psi_n^{(1)}} = \delta_{l0}$. For 
the general $m$th-order expression (\autoref{eq:perturbation_lambda_power_mth}),
\begin{equation}
    \begin{gathered}
        \bra{\Phi_n} E_n^{0} - \hat{H}_0\ket{\Psi_n^{(m)}}
            = \bra{\Phi_n} \hat{V} \ket{\Psi_n^{(m-1)}}
            - \sum_{l=0}^{m-1}E_n^{(m-l)}\braket{\Phi_n}{\Psi_n^{(l)}} \\
        E_n^{(m)} = \bra{\Phi_n^{(m)}} = \bra{\Phi_n} \hat{V} \ket{\Psi_n^{(m-1)}}.
    \end{gathered}
\end{equation} 
In principle, we can obtain every next-order energy contribution $E_n^{(m)}$ from 
the previous-order wavefunctions $\Psi_n^{(m-1)}$ and then solve for $\Psi_n^{(m)}$.

\subsection{The $2n+1$ Wigner Rule}

\subsection{Projection Operators}
We define the projection operators in terms of the zero-order wave functions,
\begin{equation}
    \begin{aligned}
        \hat{P} &= \ket{\Phi_0} \bra{\Phi_0} \\
        \hat{Q} &= \hat{1} - \hat{P} = \sum_{i=1}^N \ket{\Phi_i} \bra{\Phi_i}.
    \end{aligned}
\end{equation}
The projecton operators have the following conventient properties,
\begin{equation}
    \begin{gathered}
    \hat{P}^2 =  \ket{\Phi_0}\bra{\Phi_0}\ket{\Phi_0}\bra{\Phi_0} 
        = \ket{\Phi_0}\bra{\Phi_0} = \hat{P} \\
    \hat{Q}^2 = (1 - \hat{P})^2 = \hat{1} - \hat{P} - \hat{P} + \hat{P}
        = \hat{1} - \hat{P} = \hat{Q} \\
    \hat{P}\hat{Q} = \hat{Q} \hat{P} = 0 \\
    [ \hat{P}, \hat{H}_0 ] = [ \hat{Q}, \hat{H}_0 ] = 0
    \end{gathered}
\end{equation}

If we write the wavefunction as a linear expansion in terms of $\Phi_i$,
\begin{equation}
    \Phi = \sum_i a_i \Phi_i,
\end{equation}
acting on it with the projection operators will yield
\begin{equation}
    \hat{P} \Psi = \sum_i a_i \ket{\Phi_0}\bra{\Phi_0}\ket{\Phi_i}
        = \sum_i a_i\ket{\Phi_0}\delta_{0i} = a_0\Phi_0.
\end{equation}
In not so many greek letters, the operator $\hat{P}$ will extract $\Phi_0$ from 
$\Psi$, while $\hat{Q}$ annihilates $\hat{Q}$,
\begin{equation}
    \hat{Q}\Psi = (\hat{1} - \hat{P}) = \Psi - a_0\Phi_0 
        = \sum_{i=1}^N a_i\Phi_i,
\end{equation}
meaning we can write
\begin{equation}
    \Psi = \hat{P} \Psi + \hat{Q} \Psi.
\end{equation}

\subsection{The Resolvent}
Now follows what some considers a more elegant derivation of the perturbation equations,
including the introduction of the \emph{resolvent} of the unperturbed part of the Hamiltonian
$\hat{H}_0$.

Starting from a rearrangment of the Schrödinger equation,
\begin{equation}
    \begin{gathered}
        (\hat{H}_0 + \hat{V})\Psi = E\Psi, \\
        \to -\hat{H}_0 \Phi = (\hat{V} - E)\Psi,
    \end{gathered}
\end{equation}
we introduce a seemingline arbirary parameter $\zeta$ by adding $\zeta\Phi$ to both sides,
\begin{equation}
    (\xi - \hat{H}_0)\Phi = (\hat{V} - E + \xi)\Phi.
\end{equation}
Next, we apply $\hat{Q}$ to both sides,
\begin{equation}
    \label{eq:q_space_schrodinger_1} 
    \hat{Q}(\zeta - \hat{H}_0)\Psi = \hat{Q}(\hat{V} - E + \zeta)\Psi.
\end{equation}
The right-hand side of this expression can be rewritten as,
\begin{equation}
    \begin{aligned}
    \hat{Q}(\zeta - \hat{H}_0)\Psi &= \hat{Q}^2(\zeta - \hat{H}_0)
        = \hat{Q}(\zeta - \hat{H}_0)\hat{Q}\Psi \\
        &= \sum_{i \neq 0} \sum_{j \neq 0} \ket{\Phi_i} \bra{\Phi_i}
            \zeta - \hat{H}_0 \ket{\Phi_j} \bra{\Phi_j},
    \end{aligned}
\end{equation}
\autoref{eq:q_space_schrodinger_1} is now 
\begin{equation}
    \label{eq:q_space_schrodinger_2}
    \hat{Q}(\zeta - \hat{H}_0)\hat{Q}\Psi = \hat{Q}(\hat{V} - E + \zeta)\Psi.
\end{equation}

By restricting to choice of $\zeta$, so they do not coincide with the eigenvalues
of $\hat{H}_0$ in $\hat{Q}$-space, we ensure that the inverse of 
$\hat{Q}(\zeta - \hat{H}_0)\hat{Q}$ exists. This inverse is the \emph{resolvent} of
$\hat{H}_0$,
\begin{equation}
    \hat{R}_0(\zeta) = \frac{\hat{Q}}{\zeta - \hat{H}_0}
        \equiv \sum_{i \neq 0} \sum_{j \neq 0}
            \ket{\phi_i}\bra{\Phi_i} 
            (\zeta - \hat{H}_0)^{-1}
            \ket{\Phi_j}\bra{\Phi_j}.
\end{equation}
The resolvent simplifies in the diagonal case to
\begin{equation}
    \hat{R}_0(\zeta)
        = \sum_{i \neq 0} \ket{\Phi_i} \bra{\Phi_i}
            (\zeta - E_j^{(0)})^{-1} \ket{\Phi_j}\bra{\Phi_j}
        = \sum_{i \neq 0} \frac{\ket{\Phi_i} \bra{\Phi_i}}{(\zeta - E_i^{(0)})}. 
\end{equation}

It is somewhat straightforward to prove that $\hat{R}_0(\zeta)$ is the inverse of
$\hat{Q}(\zeta - \hat{H}_0)\hat{Q}$ in $\hat{Q}$-space,
\begin{equation}
    \begin{aligned}
        \frac{\hat{Q}}{\zeta - \hat{H}_0} &\hat{Q}(\zeta - \hat{H}_0)\hat{Q} \\
            &= \left(
                \sum_{i,j \neq 0} \ket{\Phi_i}\bra{\Phi_i}
                    (\zeta - \hat{H}_0)^{-1} \ket{\Phi_j}\bra{\Phi_j}
            \right) \left(
                \sum_{k,l \neq 0} \ket{\Phi_k}\bra{\Phi_k}
                    (\zeta - \hat{H}_0) \ket{\Phi_l}\bra{\Phi_l}
            \right) \\
            &= \sum_{i,l \neq 0} \ket{\Phi_i}
                \bra{\Phi_i}
                    (\zeta - \hat{H}_0)^{-1}
                    \left(\sum_{j \neq 0} \ket{\Phi_j} \bra{\Phi_j} \right)
                    (\zeta - \hat{H}_0)
                \ket{\Phi_l}
            \bra{\Phi_l} \\
            &= \sum_{i, l \neq 0} \ket{\Phi_i}
                \bra{\Phi_i} (\zeta - \hat{H}_0)^{-1}
                (1 - \ket{\Phi_0}\bra{\Phi_0})
                (\zeta - \hat{H}_0) \ket{\Phi_l}
            \bra{\Phi_l} \\
            &= \sum_{i \neq 0} \ket{\Phi_i}\bra{\Phi_i} = \hat{Q}.
    \end{aligned}
\end{equation}

Applying the resolvent to both sides of \autoref{eq:q_space_schrodinger_2},
\begin{equation}
    \begin{gathered}
    \hat{Q}\Psi = \hat{R}_0(\zeta)(\hat{V} - E + + \zeta)\Psi \\
    \to \Psi = \Phi_0 + \hat{R}_0(\zeta)(\hat{V} - E + \zeta)\Psi,
    \end{gathered}
\end{equation}
which can be interpreted as a recursive relation for $\Psi$. Substituting the
right-hand side into $\Psi$ on the right-hand side repeatedly yields,
\begin{equation}
    \Psi = \sum_{m=0}^\infty \{\hat{R}_0(\zeta)(\hat{V}) - E + \zeta \}^m \Phi_0.
\end{equation}
The problem with this equation is that $E$, which is unknown, appears on the right-hand
side. A question also arises regarding what to do with $\zeta$. There are two common 
choices for $\zeta$ that give rise to two important theories,
\begin{align*}
    \zeta = E &\leftarrow \text{Brillouin-Wigner Perturbation} \\
    \zeta = E^{(0)}_0 \to -E + \zeta = -\Delta E 
        &\leftarrow \text{Rayleigh-Schrödinger Perturbation}
\end{align*}

\section{Brillouin-Wigner Perturbation Theory}

Set $\zeta = E$ and get BWPT\cite{brillouin1932problemes,wigner1935modification}.

\section{Rayleigh-Schrödinger Perturbation Theory}

set $\zeta = E_n^{(0)}$ and get RSPT\cite{rayleigh1894theory,schrodinger1926quantisierung}.
