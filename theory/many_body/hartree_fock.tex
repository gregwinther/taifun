\chapter{Hartree-Fock Theory} 

    In 1927, soon after the discovery of the Schrödinger equation in 1926, Douglas R.
    Hartree introduced a procedure which he called the self-consistent field 
    method\cite{hartree1928wave}. 
    Hartree sought to do without empirical parameters and
    to solve the many-body time-independent Schrödinger equation from fundamental
    principles, \emph{ab initio}. A year later John C. Slater and John A. Gaunt 
    provided a sounder theoretical basis for the Hartree method by applying the 
    variational principle to a trial wave function as a product of single-particle
    functions\cite{slater1928self}\cite{gaunt1928theory}. 
    Slater later pointed out, with support from Vladimir A. Fock, that the method
    merely applied the Pauli exclusion principle in its older, incorrect formulation;
    forbidding presence of two electrons in the same state, but neglecting 
    quantum statistics\cite{slater1930note}\cite{fock1930naherungsmethode}. It was
    shown that a Slater determinant satisfies the antisymmetric property of the
    exact solution and would be a suitable ansatz for applying the variational
    principle. Later, Hartree reformulated the method for 
    calculation\cite{hartree1935self}.

    \section{Deriving the Hartree-Fock Equations}

