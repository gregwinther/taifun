\chapter{Hartree-Fock Theory} 

    In 1927, soon after the discovery of the Schrödinger equation in 1926, Douglas R.
    Hartree introduced a procedure which he called the self-consistent field 
    method\cite{hartree1928wave}. 
    Hartree sought to do without empirical parameters and
    to solve the many-body time-independent Schrödinger equation from fundamental
    principles, \emph{ab initio}. A year later John C. Slater and John A. Gaunt 
    provided a sounder theoretical basis for the Hartree method by applying the 
    variational principle to a trial wave function as a product of single-particle
    functions\cite{slater1928self}\cite{gaunt1928theory}. 
    Slater later pointed out, with support from Vladimir A. Fock, that the method
    merely applied the Pauli exclusion principle in its older, incorrect formulation;
    forbidding presence of two electrons in the same state, but neglecting 
    quantum statistics\cite{slater1930note}\cite{fock1930naherungsmethode}. It was
    shown that a Slater determinant satisfies the antisymmetric property of the
    exact solution and would be a suitable ansatz for applying the variational
    principle. Later, Hartree reformulated the method for 
    calculation\cite{hartree1935self}.

    The Hartree-Fock methods makes the following simplications to the multi-electron 
    atomic (molecular) problem,
    \begin{itemize}
        \item The full molecular wavefunction is contrained to a function of the
            coordinates of only the electrons in the molecule. In not so many words,
            the Born-Oppernheimer approximation is inerent in the method.
        \item Any relativistic effects are completely ignored, i.e. the momentum
            operator is assumet to be completely non-relativistic.
        \item A variational solution is assumed to be a linear combination of a basis
            set, which is assumed to be approximately complete. This set of basis
            functions is usually orthogonal, but may not be. 
        \item Some electron correlation effects are ignored, as the methods implies
            a mean-field approximation. Coulomb correlation is fully incorporated 
            in the Hartree-Fock method, but it ignores Fermi Correlation and is
            therefore inable to describe some effects, like London 
            dispersion\footnote{Named after Fritz London; London dispersion
            forces (LDF) are a type of force between atoms and 
            molecules\cite{heitler1927wechselwirkung}}.
        \item Any energy eigen function is assumed to be describable by a single 
            Slater determinant. 
    \end{itemize}

    Relaxation of the last two simplifications give rise to the large group of
    many-body methods commonly referred to as post-Hartree-Fock methods. 

    \section{Deriving the Hartree-Fock Equations}

        Consider a Hamiltionian for some system