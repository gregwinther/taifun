\chapter{Hartree-Fock Theory} 
    \label{ch:hartree_fock}

    In 1927, soon after the discovery of the Schrödinger equation in 1926, Douglas R.
    Hartree introduced a procedure which he called the self-consistent field 
    method\cite{hartree1928wave}. 
    Hartree sought to do without empirical parameters and
    to solve the many-body time-independent Schrödinger equation from fundamental
    principles, \emph{ab initio}. A year later John C. Slater and John A. Gaunt 
    provided a sounder theoretical basis for the Hartree method by applying the 
    variational principle to a trial wave function as a product of single-particle
    functions\cite{slater1928self}\cite{gaunt1928theory}. 
    Slater later pointed out, with support from Vladimir A. Fock, that the method
    merely applied the Pauli exclusion principle in its older, incorrect formulation;
    forbidding presence of two electrons in the same state, but neglecting 
    quantum statistics\cite{slater1930note}\cite{fock1930naherungsmethode}. It was
    shown that a Slater determinant satisfies the antisymmetric property of the
    exact solution and would be a suitable ansatz for applying the variational
    principle. Later, Hartree reformulated the method for 
    calculation\cite{hartree1935self}.

    The Hartree-Fock methods makes the following simplifications to the multi-electron 
    atomic (molecular) problem,
    \begin{itemize}
        \item The full molecular wavefunction is constrained to a function of the
            coordinates of only the electrons in the molecule. In not so many words,
            the Born-Oppernheimer approximation is inherent in the method.
        \item Any relativistic effects are completely ignored, i.e. the momentum
            operator is assumed to be completely non-relativistic.
        \item A variational solution is assumed to be a linear combination of a basis
            set, which is assumed to be approximately complete. This set of basis
            functions is usually non-orthogonal.
        \item Some electron correlation effects are ignored, as the methods implies
            a mean-field approximation. Coulomb correlation is fully incorporated 
            in the Hartree-Fock method, but it ignores Fermi Correlation and is
            therefore inable to describe some effects, like London 
            dispersion\footnote{Named after Fritz London; London dispersion
            forces (LDF) are a type of force between atoms and 
            molecules\cite{heitler1927wechselwirkung}}.
        \item The ground state is assumed to be describable by a single 
            Slater determinant. 
    \end{itemize}

    Relaxation of the last two simplifications give rise to the large group of
    many-body methods commonly referred to as post-Hartree-Fock methods. 

    \section{Deriving the Hartree-Fock Equations}

    Consider a Hamiltionian for some system
    \begin{equation}
        \label{eq:hf_hamiltonian}
        \hat{H} = \hat{h} + \hat{u}
    \end{equation}
    where the ground state of $\hat{h}$ is a Slater determinant consisting 
    of $N$ single-particle functions,
    \begin{equation}
        \Phi = \mathscr{A}\phi_1 \phi_2 \dots \phi_N, \quad 
            \braket{\phi_i}{\phi_j} = \delta_{ij}.
    \end{equation} 
    If $\hat{u}$ is 
    only a limited perturbation to the system, it is reasonable to assume that the 
    actual ground state of the full system can also be represented by a Slater 
    determinant. Because the Hartree-Fock theory includes a mean-field approximation,
    each particle moves independently of the others interacting with the remainding
    electrons only indirectly through an average potential $\hat{v}^{\text{HF}}$.

    The expectation value of the Hamiltionian in \autoref{eq:hf_hamiltonian} is 
    \begin{equation}
        \label{eq:hf_energy}
        \begin{aligned}
         \bra{\Phi} \hat{H} \ket{\Phi} 
            &= \sum_i \bra{\phi_i} \hat{h} \ket{\phi_i}
            + \frac{1}{2} \sum_{ij} \bra{\phi_i \phi_k} \hat{u} 
                \ket{\phi_i \phi_j - \phi_j \phi_i} \\
            &= \sum_i \bra{\phi_i} \hat{h} \ket{\phi_i}
            + \frac{1}{2} \sum_{ij} \bra{\phi_i \phi_k} \hat{u} 
                \ket{\phi_i \phi_j}_\text{AS}
        \end{aligned}
    \end{equation}
    where 
    \begin{equation*}
        \bra{\phi_i \phi_j} \hat{u} \ket{\phi_k \phi_l}
         = \int \int \bar{\phi}_i(x_1) \bar{\phi}_j(x_2) 
            \hat{u}(x_1, x_2) \phi_k(x_1) \phi_l(x_2) dx_1 dx_2
    \end{equation*}

    Now we want to minimise the energy (\autoref{eq:hf_hamiltonian}) under the
    constraint of orthonormal single-particle functions, id est
    $\braket{\phi_i}{\phi_k} = \delta_{ij}$. The minimum solution is called the 
    Hartree-Fock state, $\ket{\Phi_{\text{HF}}}$. An optimisation problem with a 
    constraint begs the formulation of a Lagrangian functional with a Lagrange
    multiplier for each constraint,
    \begin{equation}
        \label{eq:hf_lagrangian}
        \begin{aligned}
        \mathscr{L}(\phi_1, &\dots, \phi_n, \lambda)
            = \bra{\Phi} \hat{H} \ket{\Phi} 
                - \sum_{ij}\lambda_{ij}(\braket{\phi_i}{\phi_j} - \delta_{ij}) \\
            &= \sum_i \bra{\phi_i} \hat{h} \ket{\phi_i}
                + \frac{1}{2}\sum_{ij} \bra{\phi_i \phi_j} 
                    \hat{u} \ket{\phi_i \phi_j -  \phi_j \phi_i}
                - \sum_{ij} \lambda_{ij} (\braket{\phi_i}{\phi_j - \delta_{ij}}).
        \end{aligned}
    \end{equation}

    The constraints can always be treated separately, 
    $\partial \mathscr{L}/\partial \lambda_{ij} \braket{\phi_i}{\phi_j} - \delta_{ij}$,
    as this demand will be fullfilled by finding that the solutions $\phi_i$ are 
    orthonormal.

    In order to find the optimum of the Lagrangian in (\autoref{eq:hf_lagrangian}),
    we choose a $k \in \{1, \dots, N \}$ and compute the directional derivative of
    $\phi_k^*$, by varying this single particle function and leaving all others fixed,
    \begin{equation}
        \delta \phi_k = \epsilon \eta, \quad \delta \phi_l = 0, k \neq l,
    \end{equation}
    where $\epsilon$ is some small number, and $\eta$ is a normalized single-particle 
    function. We define a function representing this variation,
    \begin{equation}
        \label{eq:variation_hf_lagrangian}
        f(\epsilon) = \mathscr{L} (\phi_1, \dots, \phi_k + \epsilon \eta,
            \dots, \phi_N, \lambda),
    \end{equation}
    expanded to first order in $\epsilon$,
    \begin{equation}
        f(\epsilon) = f(0) + \epsilon f'(0) + \mathscr{O}(\epsilon^2).
    \end{equation}
    For an optimum we must have 
    \begin{equation}
        f'(0) = 0, \quad \forall \eta,
    \end{equation}
    which means that the directional derivative of $\mathscr{L}$ at 
    $\{\phi_i\}_{i=1}^N$, in the direction $\eta$ vanishes.

    We compute the Taylor expansion of the varied Lagrangian
    (\autoref{eq:variation_hf_lagrangian}),
    \begin{gather}
        \begin{gathered}
            f(\epsilon) =
            \sum_i \bra{\phi_i + \delta_{ki}\epsilon\eta} \hat{h} \ket{\phi_i}
            + \frac{1}{2} \sum_{ij} 
            \bra{(\phi_i + \delta_{ki}\epsilon\eta)(\phi_j + \delta_{kj}\epsilon\eta)}
            \hat{u} \ket{\phi_i\phi_j - \phi_j\phi_i} \\
            - \sum_{ij} \lambda_{ij} (\bra{\phi_i + \delta_{ik} \epsilon\eta}
                \ket{\phi_j} - \delta_{ij}) + \mathscr{O}(\epsilon^2)
        \end{gathered} \\
        \quad 
        \begin{gathered}
            =
            \sum_i \bra{\phi_i} \hat{h} \ket{\phi_i} 
                + \frac{1}{2} \sum_ij \bra{\phi_i \phi_j} \hat{u} 
                    \ket{\phi_i \phi_j - \phi_j \phi_i}
                + \epsilon\bra{\eta} \hat{h} \ket{\phi_k} \\
            + \frac{1}{2} \sum_{ij}\bra{\phi_i\delta_{kj}\epsilon\eta}\hat{u}
                \ket{\phi_i\phi_j - \phi_j\phi_i} \\
            + \frac{1}{2} \sum_{ij}\bra{\delta_{ki}\epsilon\eta\phi_j}\hat{u} 
                \ket{\phi_i\phi_j - \phi_j\phi_i} \\
            - \sum_{ij} \lambda_{ij}(\braket{\phi_i}{\phi_j} - \delta_{ij})
            - \sum_{ij} \lambda_{ij}(\braket{\delta_{ik}\epsilon\eta}{\phi_j} - \delta_{ij})
                + \mathscr{O}(\epsilon^2)
        \end{gathered} \\
        \quad
        \begin{gathered}
            =
            \sum_i \bra{\phi_i} \hat{h} \ket{\phi_i} 
            + \frac{1}{2} \sum_ij \bra{\phi_i \phi_j} \hat{u} 
                    \ket{\phi_i \phi_j - \phi_j \phi_i}
            + \epsilon\bra{\eta} \hat{h} \ket{\phi_k}
            - \sum_{ij}\lambda_{ij}(\braket{\phi_i}{\phi_j} - \delta_{ij}) \\
            + \frac{1}{2} \epsilon \sum_i \bra{\phi_i\eta} \hat{u}
                \ket{\phi_i \phi_k}
            - \frac{1}{2} \epsilon \sum_i \bra{\phi_i\eta} \hat{u}
                \ket{\phi_k \phi_i} \\
            + \frac{1}{2} \epsilon \sum_j \bra{\eta\phi_j} \hat{u}
                \ket{\phi_k \phi_j}
            - \frac{1}{2} \epsilon \sum_j \bra{\eta\phi_j} \hat{u}
                \ket{\phi_j \phi_k} \\
            - \epsilon \sum_j \lambda_{jk}\braket{\eta}{\phi_j}
            + \mathscr{O}(\epsilon^2) 
        \end{gathered} \\
        \quad
        \begin{gathered}
            \label{eq:hf_taylor_exp}
            =
            \sum_i \bra{\phi_i} \hat{h} \ket{\phi_i} 
            + \frac{1}{2} \sum_ij \bra{\phi_i \phi_j} \hat{u} 
                    \ket{\phi_i \phi_j - \phi_j \phi_i}
            - \sum_{ij}\lambda_{ij}(\braket{\phi_i}{\phi_j} - \delta_{ij}) \\
            + \epsilon\bra{\eta} \hat{h} \ket{\phi_k}
            + \epsilon \sum_i \bra{\eta\phi_i} \hat{u} \ket{\phi_k \phi_i}
            - \epsilon \sum_i \bra{\eta\phi_i} \hat{u} \ket{\phi_i \phi_k} \\
            - \epsilon \sum_j \lambda_{jk}\braket{\eta}{\phi_j}
            + \mathscr{O}(\epsilon^2) 
        \end{gathered}
    \end{gather}

Notice that the zeroth term, represented by the first line in \autoref{eq:hf_taylor_exp},
is simply the original Lagrangian in \autoref{eq:hf_lagrangian}. We equate all the 
first-order terms to zero,
\begin{equation}
    \bra{\eta} \hat{h} \ket{\phi_k} 
    + \sum_i \bra{\eta \phi_i} \hat{u} \ket{\phi_k\phi_i}
    - \sum_i \bra{\eta \phi_i} \hat{u} \ket{\phi_i\phi_k}
    - \sum_i \lambda_{ik} \braket{\eta}{\phi_i} = 0.
\end{equation}
This must be valid for any choice $\eta$, meaning
\begin{equation}
    \label{eq:unfinished_hf_eqn}
    \hat{h} \ket{\phi_k} 
    + \sum_i \bra{\cdot\ \phi_i} \hat{u} \ket{\phi_k\phi_i}
    - \sum_i \bra{\cdot\ \phi_i} \hat{u} \ket{\phi_i\phi_k}
    - \sum_i \lambda_{ik} \ket{\phi_i} = 0,
\end{equation}
where $\bra{\cdot\ \phi_1} \hat{u} \ket{\phi_2 \phi_3} \in L_1^2$ is interpreted
as an integral over only the second particle in the matrix element. We define,
\begin{gather}
    \hat{v}_{\text{HF}} = \hat{v}_{\text{direct}} + \hat{v}_{\text{exchange}}
        =\sum_i \bra{\cdot\ \phi_i} \hat{u} \ket{\phi_k\phi_i}
        -\sum_i \bra{\cdot\ \phi_i} \hat{u} \ket{\phi_i\phi_k} \\
    \hat{f} = \hat{h} + \hat{v}_{\text{HF}},
\end{gather}
and can then rewrite \autoref{eq:unfinished_hf_eqn} to 
\begin{equation}
    \label{eq:noncanonical_hf_eqns}
    \hat{f}\ket{\phi_i} = \sum_j \lambda_{ij}\ket{\phi_j},
\end{equation} 
which are the non-canonical Hartree-Fock equations.

It so happens that the Slater determinant $\ket{\Phi}$ is invariant under unitary
transformation of the single particle functions. Consider
\begin{equation}
    \tilde{\phi}_k = \sum_j C^j_k\phi_j ,
\end{equation}
where $C$ is a unitary matrix. This implies that $\ket*{\tilde{\Phi}} = \det(C)\ket{\Phi}$,
is the same state and the energy must be the same as well. We choose a particular
unitary transformation $C$, rotating the single particle functions in a certain 
manner so that $\lambda = CEC^\dagger$, where $E_{jk} = \delta_{jk}\epsilon_k$ are the 
elements of a diagonal matrix (the eigenvalues of $\lambda$). This provides us with 
a new set of eigenvalue equations,
\begin{equation}
    \label{eq:canonical_hf_eqns}
    \hat{f}(\tilde{\phi}_1, \dots, \tilde{\phi}_N)\ket*{\tilde{\phi}_i}
        = \epsilon_i \ket*{\tilde{\phi}_i},
\end{equation}
which are the canonical Hartree-Fock equations. From now on we will stick with these 
equations and suppress the tilde notations.

\section{The Roothan-Hall Equations}
\label{sec:roothan_hall_eqns}

In order to solve the Hartree-Fock equations (\autoref{eq:canonical_hf_eqns}) we 
render the equations in a finite, fixed basis $\{\chi_p \}_{p=1}^L$ of a finite 
size $L$. It is not a necessity for this basis to be orthonormal, and we therefore 
define the overlap matrix,
\begin{equation}
    S^\alpha_\beta \equiv \braket{\chi_\alpha}{\chi_\beta}.
\end{equation}

The Hartree-Fock single-particle functions are expanded in this basis,
\begin{equation}
    \label{eq:roothan_hall_exp}
    \ket{\phi_p} = \sum_\alpha C^\alpha_p \ket{\chi_\alpha},
\end{equation}
where $C$ is not necessarily unitary, because the basis is not necessarily 
orthongonal. However, we do have $C^\dagger SC = 1$.

We insert the expansion from \autoref{eq:roothan_hall_exp} into the expression 
for the canonical Hartree-Fock equations from \autoref{eq:canonical_hf_eqns},
\begin{equation}
    \hat{f}\sum_\alpha C^\alpha_p \ket{\chi_\alpha} = \epsilon_p \sum_\alpha C^\alpha_p \ket{\chi_\alpha}
\end{equation}
Then we left project with an arbitrary function from our new basis,
\begin{equation}
    \begin{gathered}
        \bra{\chi_\beta} \hat{f} \sum_\alpha C^\alpha_p \ket{\chi_\alpha}
            = \epsilon_\alpha \braket{\chi_\beta}{\chi_\alpha} \sum_r C^\alpha_p \\
        \sum_\alpha f^\beta_\alpha C^\alpha_p = \epsilon_p \sum_\alpha S^\beta_\alpha C^\alpha_p \\
        F(D) C = S C \epsilon.
    \end{gathered}
\end{equation}
where the last line is the Roothan-Hall equations, and $D=CC^\dagger$ is 
the density matrix.

Elaborating on the computation of the Fock matrix element,
\begin{equation}
    f^\alpha_\beta = \bra{\chi_\alpha} \hat{f} \ket{\chi_\beta}
        = \bra{\chi_\alpha} \hat{h} \ket{\chi_\beta} 
        + \bra{\chi_\alpha} \hat{v}_{\text{direct}} \ket{\chi_\beta}
        - \bra{\chi_\alpha} \hat{v}_{\text{exchange}} \ket{\chi_\beta},
\end{equation}
where
\begin{align}
    &\begin{gathered}
    \bra{\chi_\alpha} \hat{v}_{\text{direct}} \ket{\chi_p}
        = \sum_j \bra{\chi_\alpha \phi_j} \hat{u} \ket{\chi_p \phi_j} 
        = \sum_{\beta'\alpha'j} C^{\alpha'}_j (C^j_{\beta'})^*
            \bra{\chi_\alpha \chi_{\alpha'}} \hat{u} \ket{\chi_p \chi_{\beta'}} \\
        = \sum_{\beta'\alpha'} D^{\alpha'}_{\beta'} \bra{\chi_\alpha \chi_{\alpha'}} \hat{u} \ket{\chi_p \chi_{\beta'}}
    \end{gathered} \\
    &\begin{gathered}
    \bra{\chi_\alpha} \hat{v}_{\text{exchange}} \ket{\chi_p}
        = \sum_j \bra{\chi_\alpha \phi_j} \hat{u} \ket{\phi_j \chi_p} 
        = \sum_{\beta'\alpha'j} C^{\alpha'}_j (C^j_{\beta'})^* 
            \bra{\chi_\alpha \chi_{\alpha'}} \hat{u} \ket{\chi_{\beta'} \chi_p} \\
        = \sum_{\beta'\alpha'} D^{\alpha'}_{\beta'} \bra{\chi_\alpha \chi_{\alpha'}} \hat{u} \ket{\chi_{\beta'} \chi_p},  
    \end{gathered}
\end{align}
giving us,
\begin{equation}
    f^\alpha_\beta = \bra{\chi_\alpha}\hat{h} \ket{\chi_\beta} 
        + \sum_{\beta'\alpha'}D^{\alpha'}_{\beta'}(
            \bra{\chi_\alpha\chi_{\alpha'}}\hat{u} \ket{\chi_\beta\chi_{\beta'}}
            - \bra{\chi_\alpha\chi_{\alpha'}}\hat{u} \ket{\chi_{p'}\chi_\beta}
        ).
\end{equation}

The benefit of the Roothan-Hall equations (\autoref{eq:roothan_hall_exp}), is that they 
are represented by matrices, and therefore easy to implement on a computer. The
Roothan-Hall equations are solved iteratively, starting from an initial guess for $C$. 
This guess can be used to compute the density matrix, 
$D^{(k)} = C^{(k)}(C^{(k)})^H$, where $k$ denotes the $k$th iteration. The density 
matrix is used to compute the Fock matrix. This provides us with a general eigenvalue 
problem, from which a new $C$ and $\epsilon$ can be found. This formula is then repeated
until the the iterations converge. At this point we say that we have self-consistency 
in the mean field, and this method is usually called the method of self-consistent
field (SCF) iterations.

\section{Restricted Hartree-Fock Theory}

Consider $N$ electrons confined in a potential. 
To begin with we will assume that these are non-interacting, and can therefore be 
described by the one-body part of the Hamiltionian, only
\begin{equation}
    \hat{h}(\vb{r}) = \hat{t}(\vb{r}) + \hat{v}(\vb{r}),
\end{equation}
where $\hat{v}$ is potential set up by an atomic nucleus, or some other confining
force. The one-body operators $\hat{h}$ does not couple to electron spin, so the 
spin-orbitals or single-particle eigenfunctions of $\hat{h}$ separate,
\begin{equation}
    \phi_P(\vb{r}, \sigma) = \varphi_p(\vb{r})\chi_\alpha(\sigma),
\end{equation}
where $P = (p, \alpha)$ is the combined spin- and spatial index, $\alpha = \pm 1/2$
is the value of the projection of the electron spin along the $z$-axis. The spin 
index/coordinate can only take values $\sigma \pm 1$, and we have orthonormal 
spinorbitals, $\braket{\chi_\alpha}{\chi_\beta} = \delta_{\alpha\beta}$. 

We restrict the orbitals to have the same spatial wavefunction for spin up and spin
down, and we consider only closed-shell configurations. This means that our 
molecular wavefunctions, in the form of a Slater determinant, can only have an 
even number of $N$ electrons, with all electrons paired in such a manner that there 
are two spin values for each of the $n=N/2$ spatial orbitals. The $N$-electron 
ground state of $\hat{h}$ is given by the first $N$ eigenfunctions $\phi_{(p,\sigma)}$
occupied,
\begin{equation}
    \ket{\Phi} = \ket*{\phi_{1,+}\phi_{1,-}\dots \phi_{\frac{N}{2},+}\phi_{\frac{N}{2},-}},
\end{equation}
commonly also written as 
\begin{equation}
    \ket{\Phi}_{\text{RHF}} = 
    \ket*{\varphi_1 \bar{\varphi}_1 \dots \varphi_{N/2} \bar{\varphi}_{N/2}}.
\end{equation}
The reasoning behind this restriction is that one would assume, for many systems, that 
the exact wavefunctions has the same kind of structure. This is true for almost all 
electronic systems in nature. We therefore do not optimise all the $N$ single-particle 
functions freely, but asume that they form sets of doubly occupied spatial orbitals. 
Matrix elements can now be computed more easily on the restricted form,
\begin{equation}
    \mel*{\phi_{(p,\alpha)}}{\hat{h}}{\phi_{(q,\beta)}} 
    = \braket{\chi_\alpha}{\chi_\beta}
        \int d\vb{r} \varphi_p(\vb{r})^*\hat{h}\varphi_q(\vb{r}).
\end{equation}
And similarly for two-body operators,
\begin{equation}
    \mel{\phi_{p\alpha}\phi_{q\beta}}{\hat{u}}{\phi_{r\gamma}\phi_{s\delta}}
    = \braket{\chi_\alpha}{\chi_\gamma}\braket{\chi_\beta}{\chi_\delta}
        \int \int d \vb{r}_1 d \vb{r}_2 \varphi_p(\vb{r}_1)\varphi_q(\vb{r}_2)
            \hat{u}(\vb{r}_1 \vb{r}_2) \varphi_r(\vb{r}_1)\varphi_s(\vb{r}_2).
\end{equation}

Now we will find the special form of the Fock operator in restricted 
Hartree-Fock theory. First we insert the wavefunction restriction into 
the Hartree-Fock equation
\begin{equation}
    \begin{aligned}
        \hat{f}\phi_I(\vb{r}, \sigma) &= \epsilon_i \phi_I(\vb{r}, \sigma) \\
        \hat{f}\varphi_i(\vb{r}) \chi_\alpha(\sigma)
        &= \epsilon_i \varphi_i(\vb{r})\chi_\alpha(\sigma).
    \end{aligned}
\end{equation}
Here we have joined the spatial- and spin index with a capital letter $I = (i, \alpha)$.
We left multiply with $\chi_\alpha^*$, suppress indices for brevity and integrate over spin,
\begin{equation}
    \mel{\chi_\alpha}{\hat{f}}{\varphi_i\chi_\alpha}
    = \mel{\chi_\alpha}{\hat{f}}{\phi_{I = (i,\alpha)}}
    = \epsilon_i\varphi_i.
\end{equation}
Next we insert the Fock operator,
\begin{equation*}
    \hat{f} = \hat{h} + \sum_i\mel{\cdot\varphi_i}{\hat{u}}{\cdot\varphi_i}_\text{AS}
\end{equation*} 
This special notation means that we integrate over the second orbital in 
the bra and ket only. After insertion we get
\begin{equation}
   \begin{gathered}
        \mel{\chi_\alpha}{\hat{h}}{\chi_\alpha}\varphi_i
        + \sum_J \mel{\chi_\alpha\phi_J}{\hat{u}}{\phi_I\phi_J}_\text{AS}
        = \epsilon_i\varphi_i \\
        \to\cancel{\braket{\chi_\alpha}}\hat{h}\varphi_i 
        + \sum_J \mel{\chi_\alpha\phi_J}{\hat{u}}{\phi_I\phi_J}
        - \sum_J \mel{\chi_\alpha\phi_J}{\hat{u}}{\phi_J\phi_I}
        = \epsilon_i\varphi_i.
   \end{gathered} 
\end{equation}
Because we have a closed-shell system, the sum over occupied spinorbitals include an 
equal sum over spin up and spin down functions so that
\begin{equation*}
    \sum_J^N = \sum_\beta\sum_j^{n/2}.
\end{equation*} 
We next insert this into our eigenvalue equation and split the single-particle functions 
into separate spin- and spatial orbitals,
\begin{equation}
    \begin{gathered}
        \hat{h}\varphi_i 
        + \sum_\beta\sum_j^{n/2}
            \mel{\chi_\alpha\varphi_j\chi_\beta}{\hat{u}}{\varphi_i\chi_\alpha\varphi_j\chi_\beta} 
        - \sum_\beta\sum_j^{n/2}
            \mel{\chi_\alpha\varphi_j\chi_\beta}{\hat{u}}{\varphi_j\chi_\beta\varphi_i\chi_\alpha} \\
        = \hat{h}\varphi_i 
        + 2\sum_j^{n/2} \mel{\cdot \varphi_j}{\hat{u}}{\varphi_i\varphi_j}
        +  \sum_j^{n/2} \mel{\cdot \varphi_j}{\hat{u}}{\varphi_j\varphi_i}
        = \epsilon_i\varphi_i.
    \end{gathered}
\end{equation}
We now have the form of the Fock operator within the restricted Hartree-Fock theory,
\begin{equation}
    \label{eq:RHF_fock_operator}
    \begin{aligned}
    \hat{f} &= \hat{h} + \sum_i^{n/2} \mel{\cdot\varphi_i}{(2\hat{u} - \hat{P}_{12})}{\cdot\varphi_i} \\
        &= \hat{h} + 2\sum_i^{n/2} \int d\vb{r}_2 \varphi_i^*(\vb{r}_2) \hat{u} \varphi_i(\vb{r}_2)
        -\sum_i^{n/2} \int d\vb{r}_2 \varphi_i^*(\vb{r}_2) \hat{u} \varphi_j(\vb{r}_2)
    \end{aligned}
\end{equation}

The Hartree-Fock energy also has a special form in the restricted Hartree-Fock domain,
\begin{equation}
    \begin{aligned}
        \mel{\Phi}{\hat{H}}{\Phi}
        =& \sum_P \mel{\phi_P}{\hat{h}}{\phi_P}
        + \frac{1}{2}\sum_P \sum_Q \mel{\phi_P \phi_Q}{\hat{u}}{\phi_P \phi_Q}_\text{AS} \\
        =& \sum_\alpha \sum_p^{n/2} \mel{\phi_{(p,\alpha)}}{\hat{h}}{\phi_{(p,\alpha)}} \\
            &\quad + \sum_\alpha \sum_p^{n/2}\sum_\beta \sum_q^{n/2} 
        \bra{\phi_{(p,\alpha)}\phi_{(q,\beta)}}\hat{u}
        \left( 
            \ket{\phi_{(p,\alpha)}\phi_{(q,\beta)}} - \ket{\phi_{(q,\beta)}\phi_{(p,\alpha)}} 
        \right) \\
        =& 2\sum_p^{n/2} \mel{\varphi_p}{\hat{h}}{\varphi_q}
        + 2 \sum_{pq}^{n/2} \mel{\varphi_p\varphi_q}{\hat{u}}{\varphi_p\varphi_q}
        - \sum_{pq}^{n/2} \mel{\varphi_p\varphi_q}{\hat{u}}{\varphi_p\varphi_q}
    \end{aligned}
\end{equation}

\section{Unrestricted Hartree-Fock Theory}

The restricted Hartree-Fock model is often a good enough approximation, but under some 
circumstances it will fail to provide a good result. The unrestricted Hartree-Fock model 
is an intermediate between the general Hartree-Fock model and the restricted Hatree-Fock 
model. Compared with the restricted Hartre-Fock single-particle wavefunction form, what we do 
in unrestricted form is somewhat obvious - we now allow the spins to be different,
\begin{equation}
    \phi_P(\vb{r}, \sigma) = \varphi_p^{\alpha}(\vb{r})\chi_\alpha(\sigma),
\end{equation}
where we have given the spatial orbitals a spin-index as well. As before, a capital index 
is the combined spatial- and spin index $P=(p,\alpha)$, where $P \in [1,L]$, $p\in [1, L/2]$
and $\alpha = \pm 1/2$. 
Like before, we require the states to be orthonormal 
\begin{equation}
    \braket{\phi_P}{\phi_Q} 
    = \braket{\varphi^\alpha_p}{\varphi^\beta_q}\braket{\chi_\alpha}{\chi_\beta}
    = \delta_{PQ}.
\end{equation}
We can write a general unrestricted Hartree-Fock state as 
\begin{equation}
    \ket{\Phi}_{\text{UHF}}
        =
        \ket*{\varphi_1^{1/2} \varphi_1^{-1/2} \varphi_2^{1/2} \varphi_2^{-1/2}
            \dots \varphi_{L/2}^{1/2} \varphi_{L/2}^{-1/2}}
        =
        \ket{\phi_1\phi_2\phi_3\phi_4 \dots \phi_{L-1} \phi_L}.
\end{equation}

In order to find an expression for the Fock operator we insert the wavefunction into the 
canonical Hartree-Fock equation,
\begin{equation}
    \hat{f}\phi_P = \epsilon_p\phi_P, \ \to \ 
    \hat{f}\varphi_p^\alpha \chi_\alpha = \epsilon_p\varphi_p^\alpha \chi_\alpha. 
\end{equation}
Now we left multiply by $\chi_\alpha^*$ and integrate over spin,
\begin{gather}
    \mel{\chi_\alpha}{\hat{f}}{\varphi_p^\alpha\chi_\alpha}
    = \mel{\chi_\alpha}{\epsilon_p}{\varphi_p^\alpha\chi_\alpha} \\
    \hat{f}^\alpha \varphi_p^\alpha
    = \left[ 
        \int d\sigma_1 \chi_\alpha(\sigma_1)^*\hat{f}(\vb{r}, \sigma_1)\chi_\alpha(\sigma_1)     
    \right] \varphi_p^\alpha
    = \epsilon_p\varphi_p^\alpha.
\end{gather}
We now have what is called the spatial unrestricted Hartree-Fock equations. Inserting for 
the canonical Fock operator yields the following left-hand side
\begin{equation}
    \begin{aligned}
        \hat{f}^\alpha\varphi_p^\alpha 
        =& \hat{h}\varphi_p^\alpha 
        + \sum_Q^L \mel{\chi_\alpha \phi_Q}{\hat{u}}{\varphi_p^\alpha\chi_\alpha\phi_Q}q
        - \sum_Q^L \mel{\chi_\alpha \phi_Q}{\hat{u}}{\phi_Q\varphi_p^\alpha\chi_\alpha} \\
        =& \hat{h}\varphi_p^\alpha
        + \sum_\beta \sum_q^{L/2}
            \mel{\chi_\alpha\varphi^\beta_q\chi_\beta}{\hat{u}}{\varphi^\alpha_p\chi_\alpha\varphi^\beta_q\chi_\beta} \\
        & \quad - \sum_\beta \sum_q^{L/2}
            \mel{\chi_\alpha\varphi^\beta_q\chi_\beta}{\hat{u}}{\varphi^\beta_q\chi_\beta\varphi^\alpha_p\chi_\alpha} \\
        =& \hat{h} \varphi_p^\alpha
        + \sum_\beta \sum_q^{L/2}
            \mel{\cdot\varphi_q^\beta}{\hat{u}}{\cdot\varphi_q^\beta}\varphi_p^\alpha
        - \sum_q^{L/2}
            \mel{\cdot\varphi_q^\alpha}{\hat{u}}{\cdot\varphi_p^\alpha}\varphi_q^\beta
    \end{aligned}
\end{equation} 
This means that we get the following form for the spatial Fock operators in unrestricted Hartree-Fock 
\begin{align}
\label{eq:fock_up_uhf}
\hat{f}^\uparrow &= 
    \hat{h} + \sum_p^{L/2}[\hat{v}^\uparrow_{\text{Coulomb}} - \hat{v}^\uparrow_{\text{exchange}}]
    + \sum_p^{L/2}\hat{v}^\downarrow_{\text{Coulomb}}  \\
\label{eq:fock_down_uhf}
\hat{f}^\downarrow &= 
    \hat{h} + \sum_p^{L/2}[\hat{v}^\downarrow_{\text{Coulomb}} - \hat{v}^\downarrow_{\text{exchange}}]
    + \sum_p^{L/2}\hat{v}^\uparrow_{\text{Coulomb}}
\end{align}
From the definition of the two spatial Fock operators in \autoref{eq:fock_up_uhf} and 
\autoref{eq:fock_down_uhf},
we see that the two integro-differential eigenvalue equations that arises from inserting
$\hat{f}^\uparrow$ and $\hat{f}^\downarrow$ into the canonical Hartree-Fock equation,
\begin{align}
    \label{eq:fock_eqn_up_uhf}
    \hat{f}^\uparrow \varphi_p^\uparrow &= \epsilon_p^\uparrow\varphi_p^\uparrow \\
    \label{eq:fock_eqn_down_uhf}
    \hat{f}^\downarrow \varphi_p^\downarrow &= \epsilon_p^\downarrow\varphi_p^\downarrow,
\end{align}
are 
coupled and cannot be solved independently. The spin-up orbitals depend on the occupied spin-down 
orbitals and vice versa. This means that the two equations must be solved by a simultaneous 
iterative process.

We can also derive an equation for the unrestricted Hartree-Fock energy,
\begin{equation}
    \begin{aligned}
    E_{UHF} =& \mel{\Phi_{\text{UHF}}}{\hat{H}}{\Phi_{\text{UHF}}}\\
    =& \sum_\alpha\sum_p^{L/2} \mel{\varphi_p^\alpha\chi_\alpha}{\hat{h}}{\varphi_p^\alpha\chi_\alpha}
    + \sum_\alpha\sum_p^{L/2}\sum_\beta\sum_q^{L/2}
        \mel{\varphi_p^\alpha\chi_\alpha\varphi_q^\beta\chi_\beta}{\hat{u}}
            {\varphi_p^\alpha\chi_\alpha\varphi_q^\beta\chi_\beta}_\text{AS} \\
    =& \sum_\alpha\sum_{pq}^{L/2} \mel{\varphi_p^\alpha}{\hat{h}}{\varphi_p^\alpha} 
    + \sum_{\alpha\beta}\sum_q^{L/2}
        \mel{\varphi_p^\alpha\varphi_q^\beta}{\hat{u}}{\varphi_p^\alpha\varphi_q^\beta} 
    - \sum_\alpha\sum_{pq}^{L/2}
        \mel{\varphi_p^\alpha\varphi_q^\alpha}{\hat{u}}{\varphi_q^\alpha\varphi_p^\alpha}
    \end{aligned}
\end{equation}

If we were to expand the unrestricted Hartree-Fock equations, \autoref{eq:fock_eqn_up_uhf} and 
\autoref{eq:fock_eqn_down_uhf}, in a basis like we did in \autoref{sec:roothan_hall_eqns}, we 
would get the Pople-Nesbet-Bethier equations\cite{berthier1954extension,pople1954self}.

\section{Time-Dependent Hartree-Fock}

This section follows closely the narrative of \citeauthor{hochstuhl2014time}\cite{hochstuhl2014time}.
Deriving the time-dependent Hartree-Fock equations start,
of course, with the time-dependent Schrödinger equation,
\begin{equation}
    i\hbar \frac{\partial}{\partial t}\ket{\Phi(t)} = \hat{H}(t)\ket{\Phi(t)},
\end{equation}
where the Hamiltonian is 
\begin{equation}
    \hat{H}(t) = \hat{h}(t) + \hat{u}(t).
\end{equation}
This is the same Hamiltonian that we started with in this chapter (\autoref{eq:hf_hamiltonian}),
except for the introduction of a 
time-dependence. We start by multiplying from the left with the reference Slater
determinant $\bra{\Phi}$. The right-hand side of the Schrödinger equation becomes
the familiar Hartree-Fock energy,
\begin{equation}
    \mel{\Phi}{\hat{H}}{\Phi} 
    = \sum_p \mel{\phi_p}{\hat{h}}{\phi_p}
    + \frac{1}{2} \sum_{pq}\mel{\phi_p\phi_q}{\hat{u}}{\phi_p\phi_q}_\text{AS}.
\end{equation}
The left-hand side, is more interesting,
\begin{equation}
    \mel{\Phi}{\frac{\partial}{\partial t}}{\Phi} 
    = \sum_p \mel{\phi_p}{\frac{\partial}{\partial t}}{\phi_p},
\end{equation}
which we will deal with in due time, but before doing so we need to introduce 
functional derivatives and the functional derivatives of various matrix elements.
First, the one-body matrix elements,
\begin{equation}
        \frac{\delta}{\delta\phi^*_r} \sum_p \mel{\phi_p}{\hat{h}}{\phi_p}
        = \sum_p \frac{\delta}{\delta \phi^*_r}
            \int dr \phi_p^* \hat{h} \phi_p 
        = \sum_p \delta_{pr} \hat{\phi} = \hat{h}\ket{\phi_r}.
\end{equation}
Second, the matrix elements of the time-derivative,
\begin{equation}
    \frac{\delta}{\delta \phi_r^*} \sum_p\mel{\phi_p}{\frac{\partial}{\partial t}}{\phi_p}
    = \frac{\partial}{\partial t} \ket{\phi_q},
\end{equation}
which is so similar to the one-body computation that the result is simply written down,
instead of computing the result explicitly.
Lastly, we have the two-body matrix elements,
\begin{equation}
    \begin{aligned}
        \frac{\delta}{\delta \phi_r^*} &\sum_{pq}
            \mel{\phi_p\phi_q}{\hat{u}}{\phi_p\phi_q}_\text{AS} \\
        &\ = \frac{\delta}{\delta \phi_r^*(r_1)} \sum_{pq} \int dr_1 dr_2 
            \phi_p^*(r_1) \phi_q^*(r_2)\hat{u}[\phi_p(r_1)\phi_q(r_2) - \phi_q(r_1)\phi_p(r_2)] \\
        &\ = \sum_{pq}\delta_{pq} \int dr_2 \phi_q^*(r_2) 
            \hat{u} [\phi_p(r_1) \phi_q(r_2) - \phi_q(r_1)\phi_p(r_2)]
        = \sum_q \mel{\cdot \phi_q}{\hat{u}}{\phi_r\phi_q}_\text{AS}.
    \end{aligned}
\end{equation}

Now we want to vary the reference state to find the optimal one, 
applying the so-called time-dependent variational principle\cite{dirac1930principles},
\begin{equation}
    \mel{\delta\Phi}{(\hat{H} - i\hbar\frac{\partial}{\partial t})}{\Phi} = 0,
\end{equation}
which we want to minimise under the requirement of orthonormal single-particle functions
in time,
\begin{equation}
    \braket{\phi_p(t)}{\phi_q(t)} = \delta_{pq}.
\end{equation}
Such an optimization problem under a constraint begs the formulation of a Lagrangian,
which we will allow to manifest,
\begin{equation}
    \mathscr{L}(\Phi, \lambda_{pq}) = 
        \mel{\Phi}{(\hat{H} - i\hbar\frac{\partial}{\partial t})}{\Phi}
        - \sum_{pq} \lambda_{pq} (\braket{\phi_p}{\phi_q} - \delta_{pq}).
\end{equation}
We find a stationary point of this Lagrangian functional, by variation of the single-particle 
functions so that 
\begin{equation}
    \frac{\delta\mathscr{L}}{\delta \phi_r^*} = 0, \quad \forall r.
\end{equation}
This is where we will make use of the functional derivatives we computed before,
\begin{equation}
    \label{eq:derivative_of_HF_time_lagrangian}
    \frac{\delta\mathscr{L}}{\delta \phi_r^*}
    = \hat{h}\ket{\phi_r} 
        + \sum_q\mel{\cdot\phi_q}{\hat{u}}{\phi_r\phi_q}_\text{AS}
        - i\hbar\frac{\partial}{\partial t} \ket{\phi_r}
        - \sum_q \lambda_{rq} \ket{\phi_r} = 0.
\end{equation}
Now we want to solve for the Langrange multiplier, we do this by left-projection of the 
functional derivative above with $\bra{\phi_s}$ and move the resulting multiplier $\lambda_{sq}$
to the left, and all other terms to the right. We 
get the following expression for the Lagrange multiplier,
\begin{equation}
    \lambda_{sq} = \mel{\phi_s}{\hat{h}}{\phi_r} 
        + \mel{\phi_s\phi_q}{\hat{u}}{\phi_r\phi_q}_\text{AS}
        - i\hbar \mel{\phi_s}{\frac{\partial}{\partial t}}{\phi_r}
\end{equation}
We insert this expression for the Lagrange multiplier into \autoref{eq:derivative_of_HF_time_lagrangian}
which results in,
\begin{equation}
    \label{eq:projected_time_hf_eqn}
    \hat{P}\left[\hat{h}\ket{\phi_r}
        + \sum_q\mel{\cdot\phi_q}{\hat{u}}{\phi_r\phi_q}_\text{AS}
        - i\hbar\frac{\partial}{\partial t} \ket{\phi_k}\right] = 0,
\end{equation}
where we have introduced the projection operator $\hat{P}$, 
\begin{equation}
    \hat{P} = \hat{1} - \sum_p \outerproduct{\phi_p}.
\end{equation}
Rearranging \autoref{eq:projected_time_hf_eqn} yields
\begin{equation}
    \label{eq:tdhf_before_Q}
    i\hbar \hat{P}\frac{\partial}{\partial t} \ket{\phi_r} 
    = \hat{P}\left[\hat{h}\ket{\phi_r} 
        + \mel{\cdot\phi_q}{\hat{u}}{\cdot\phi_q} \right]\ket{\phi_r}_\text{AS}
    = \hat{P}\hat{f}\ket{\phi_r},
\end{equation}
where we see that Fock operator has appeared. This equation is an integro-differential 
equation, as the projection operator $\hat{P}$ appear on both sides of the equality 
sign, and a solution can be difficult to find. Because the time-dependent Hartree-Fock 
wavefunction is invariant under unitary transformation, we can obtain equations that are 
numerically more apporiate, by applying a unitary transformation $\hat{Q}(t)$ which 
satisfies
\begin{equation}
    i\hbar\mel{\phi_p}{\frac{\partial}{\partial t}}{\phi_q}
    \equiv \mel{\phi_p}{\hat{Q}(t)}{\phi_q}.
\end{equation}
It turns out that a reasonable choice for $\hat{Q}(t)$ is $\hat{f}(t)$, in which case 
\autoref{eq:tdhf_before_Q} becomes
\begin{equation}
    \label{eq:tdhf_eqn}
    i\hbar\frac{\partial}{\partial t}\ket{\phi_p(t)} = \hat{f}(t)\ket{\phi_p(t)},
\end{equation}
where we have explicitly written out the time-dependence. This is the time-dependent 
Hartree-Fock equation.

Now we pick a specific, finite and static basis $\{\chi_p\}_{p=1}^{L}$ and expand the Hartree-Fock single-particle
functions in this basis,
\begin{equation}
    \ket{\phi_p(t)} = \sum_\alpha C^\alpha_p(t) \ket{\chi_\alpha}.
\end{equation}
Notice that the basis set is indeed static, with no time-dependence, only the coefficients 
of the expansions $U^\alpha_p(t)$ evolve in time.  
We insert the expansion into \autoref{eq:tdhf_eqn},
\begin{equation}
    i\hbar\frac{\partial}{\partial}\sum_\alpha C^\alpha_p \ket{\chi_\alpha} = \hat{f}(t)\sum_\alpha C^\alpha_p \ket{\chi_\alpha}.
\end{equation}
We left-project this equation with $\bra{\chi_\beta}$,
\begin{equation}
    \begin{gathered}
    i\hbar\frac{\partial}{\partial t} \sum_\alpha C^\alpha_p(t)\braket{\chi_\beta}{\chi_\alpha}
        = \sum_\alpha C^\alpha_p(t) \mel{\chi_\beta}{\hat{f}(t)}{\chi_\alpha}  \\
    \to i\hbar \sum_\alpha \dot{C}^\alpha_p S^\beta_\alpha = \sum_\alpha C^\alpha_p\hat{f}^\beta_\alpha(t),
    \end{gathered}
\end{equation}
which can be written as a matrix equation,
\begin{equation}
    \label{eq:tdhf}
    i\hbar \vb{S}\dot{\vb{C}}(t) = \vb{F}(t)\vb{C}(t).
\end{equation}

We have landed at the time-dependent Hartree-Fock equations. These 
are a set of elegant equations of motion that dictate
the time-development of 
the system by simple propagation of the Hartree-Fock coefficients. 
Moreover, we come to the surprising realisiation that it is necessary 
to compute the Hatree-Fock self-consistent field iterations only once, 
and after that we can treat \autoref{eq:tdhf} as set of ordinary 
differential equations, which can solved numerically without great 
effort. The only consideration 
one must make is to update the Fock matrix at each time step.