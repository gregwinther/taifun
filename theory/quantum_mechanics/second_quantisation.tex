\chapter{Second Quantisation}
 
The second-quantisation formalism is a very useful tool used in the description of 
many-body systems. Here the particles themselves are discrete quanta created and 
destroyed by \emph{creation}- and \emph{annihilation} operators. We start by introduction 
of the Slater determinant, a very useful description of an anti-symmetric wavefunction,
in order to build a nomenclature for describing the many-electron systems with 
which we are concerned.

\section{Slater Determinants}

For some smaller
systems it can be satisfactory or even provident to use a single, special function 
to describe the entire system. Here however, we introduce the Slater determinant which 
is a way to write a product of wavefunctions.  
We will only consider many-electron wavefunctions that can be written as a single 
Slater determinant or as a linear combination of several Slater determinants.

We define an \emph{orbital}\footnote{Sometimes also called a single-particle function,
a single-particle orbital, a single-electron orbital or similar. There is a chance that 
these terms will be used interchangably throughout this text without warning.} which is the
wavefunction for a single particle, or more precicely a single electron. The wavefunction
for a larger group of electrons,
for instance those electrons that surround an atom or molecule,
we call the \emph{molecular orbital}. We also discriminate between 
spatial orbitals which are functions of spatial coordinates; and spin-orbitals, which 
are functions of the space and spin coordinates (typically a product of a spatial orbital
and a spin function). A very complete description and thorough discussion of various aspects 
concerning electronic sructure wavefunctions is given 
by \citeauthor{szabo2012modern} \cite{szabo2012modern}.

The best description for a multiple-electron wavefunction, given by the independent-particle
approximation is the Slater determinant,
\begin{equation}
    \label{eq:general_slater_determinant}
    \begin{aligned}
    \Phi(x_1, x_2, \dots x_N) &= \frac{1}{\sqrt{N!}} \begin{vmatrix}
        \phi_1(x_1) & \phi_2(x_1) & \dots & \phi_N(x_1) \\
        \phi_1(x_2) & \phi_2(x_2) & \dots & \phi_N(x_2) \\
        \vdots & \vdots & \ddots & \vdots \\
        \phi_1(x_N) & \phi_2(x_N) & \dots & \phi_N(x_N)
    \end{vmatrix} \\
    &\equiv 
    \braket{x_1,x_2,\dots x_N}{\phi_1,\phi_2,\dots \phi_N}
    %= \mathcal{A}\phi_1 \phi_2 \dots \phi_N,
    \end{aligned}
\end{equation}
where $\phi_i(x_\mu)$ is a spinorbital indexed by $i$, at coordinates
$x_\mu$. The
spin-orbitals, are single-particle functions in the proper two-dimensional 
Hilbert space and they are not necessarily orthonormal.
The Slater determinant in \autoref{eq:general_slater_determinant}
defines an $N$-electron system. Adding a column and a row to the
determinant in \autoref{eq:general_slater_determinant} corresponds to 
adding a particle to the system, and we obtain an $N+1$-electron Slater determinant.
Similarly, removing a columns and a row 
from the determinant corresponds to removing a particle from the system, and we 
obtain an $N-1$ Slater determinant.
In the final expression in \autoref{eq:general_slater_determinant},
a compact notation is introduced. Here the normalisation constant and labels
for the fermion coordinates are understood – only the wavefunctions are exhibited.

To illustrate why this is a good approximation of the electronic wave function,
consider first the two-electron case,
\begin{equation}
    \Phi_{N=2} = \frac{1}{\sqrt{2}}(\phi_1(1)\phi_2(2) - \phi_1(2)\phi_2(1)).
\end{equation}
We see from this relatively simple expression that if the electrons were to occupy
the same state the wavefunction would equal zero, in effect forbidding such a state.
This ensures that the Pauli exclusion principle for 
fermions is satisfied \cite{pauli1925zusammenhang}. Moreover, if we switch coordinates of any two 
single-particle functions (spin-orbitals), corresponding to the interchange of rows in
\autoref{eq:general_slater_determinant}, the result is a change of sign. This attribute
of a determinant accomodates the total anti-symmetry necessary for a fermionic wavefunction.
Conversely, a bosonic wavefunction can be constructed as a \emph{permanent}.

We usually write a Slater determinant in a much more simple way,
    \begin{equation}
        \ket{\Phi} = \ket{\phi_i \phi_j \phi_k \dots \phi_z} = \ket{ijk \dots z}.
    \end{equation}
The $N$ single-particle functions $\phi_i\dots\phi_z$, that make up this Slater determinant
now form a basis for an $N$-particle Hilbert space $\mathcal{H}_N$. 

\section{Creation and Annihilation Operators}
   
    The introduction of creation- and annihilation operators are what establishes
    the second quantisation formalism. As we will see, such operators make the 
    construction of $N$-particle wavefunctions as symmetrised or anti-symmetrised
    products redundant, because these symmetry properties are encompassed in the 
    anticommutation properties of the operators. This is a great advantage of the 
    second quantisation framework. Another advantage is the relatively easy 
    management of many-particle systems. 

    The notation of creation and annihilation operators vary,
    \begin{align*}
        \text{creation operator for spinorbital } \phi_i, &\ 
        \hat{X}_i^\dagger, \hat{a}^\dagger_i, \hat{c}_i^\dagger, \hat{i}^\dagger; \\
        \text{annihilation operator for spinorbital } \phi_i, &\ 
        \hat{X}_i, \hat{a}_i, \hat{c}_i, \hat{i}.       
    \end{align*}
    Herein, $\hat{a}_i^\dagger$, $\hat{a}_i$ is used and, if there is no chance of 
    confusion, $\hat{i}^\dagger$, $\hat{i}$.

    \paragraph{The Creation Operator}
    For every single-particle index $q$,
    we define the creation operator
    $a_i^\dagger$ acting on the vacuum state by
    \begin{equation}
        \hat{a}_i^\dagger \ket{0} = \ket{i}.
    \end{equation}
    For any slater determinant with $N > 0$, the action is defined by
    \begin{align}
        \hat{a}_i^\dagger\ket{jk \dots z} &= \ket{ijk \dots z}, \\
        \hat{a}_i^\dagger\ket{ijk \dots z} &= 0.
    \end{align}
    We see that this the the same as inserting a column into the matrix-form of the 
    Slater determinant in \autoref{eq:general_slater_determinant}.

    \paragraph{The Annihilation Operator}
    It is sufficient to state that the 
    annihilation operator $a_i$ is the hermitian adjoint of the creation operator
    $\hat{a}_i$, but to be specific we need
    \begin{equation}
        \hat{a}_i\ket{0} = 0,
    \end{equation}
    as there is no particle in the vacuum state to annihilate. 
    For any arbitrary Slater determinant, we have 
    \begin{align}
        \hat{a}_i \ket{ijk \dots z} &= \ket{ij \dots z}, \\
        \hat{a}_i \ket{jk \dots z} &= 0.
    \end{align}

    The creation- and annihilation operators map wavefunctions between Hilbert spaces of 
    different dimensionality, or particle number;
    \begin{align}
        \hat{a}^\dagger_i: &\quad \mathcal{H}_N \to \mathcal{H}_{N+1} \\
        \hat{a}_i: &\quad \mathcal{H}_N \to \mathcal{H}_{N-1},
    \end{align}
    where $\mathcal{H}_N$ is the Hilbert space for $N$ particles. The space comprising
    all Hilbert spaces of different particles numbers is called the Fock space, defined 
    as a direct sum of all Hilbert spaces
    \begin{equation}
        \mathscr{F} = \bigoplus_{N=0}^\infty \mathcal{H}_N.
    \end{equation}
    The creation- and annihilation operators act on this Fock space.

    We can now build a Slater determinant as the result of successive operations of
    several creation operators $\hat{a}_q^\dagger$ on the vacuum state $\ket{0}$,
    \begin{equation}
        \label{eq:creation_annihilation_slater}
        \hat{a}^\dagger_i \hat{a}^\dagger_j \hat{a}^\dagger_k \dots \hat{a}^\dagger_z
        \ket{0} = \ket{ijk \dots z}.
    \end{equation}
    It is convenient to arrange the spin-orbitals in a Slater determinant in
    alphabetical order, as in \autoref{eq:creation_annihilation_slater}. This 
    makes it necessary to ascertain the effects a creation or annihilation
    operator will have on a Slater determinant when the affected orbital is
    not at the beginning of the string of orbitals in the Slater determinant.
    Generally we have,
    \begin{equation}
        \hat{P} \ket{ijk \dots z} = (-1)^{\sigma(\hat{P})} \ket{ijk \dots z},
    \end{equation}
    where $\hat{P}$ permutes the string of orbitals and $\sigma(\hat{P})$ is the
    parity of the permutation $\hat{P}$. We have
    \begin{align}
        \hat{a}_p^\dagger \ket{ijk \dots z} &= 
            (-1)^{\eta_p} \ket{ijk \dots p \dots z}, \\
        \hat{a}_p \ket{ijk \dots p \dots z} &=
            (-1)^{\eta_p} \ket{ijk \dots z},
    \end{align}
    where $\eta_p$ is the number of orbitals preceeding the orbital $\phi_p$, pertaining
    to the creation (annihilation) operator, in the Slater determinant.

\section{Anticommutator Relations}

    To show how the anticommutator relations are built into the creation- and annihilation 
    operators, we start by considering two arbitrary creation operators acting on a Slater determinant,
    \begin{equation}
        \begin{aligned}
            \hat{a}_p^\dagger \hat{a}_q^\dagger \ket{ijk \dots} &= \ket{pqijk \dots}, \\
            \hat{a}_q^\dagger \hat{a}_p^\dagger \ket{ijk \dots} &= \ket{qpijk \dots}
                = - \ket{pqijk \dots}.
        \end{aligned}
    \end{equation}
    We demand that these two operations be equivalent, or that 
    \begin{equation}
        \begin{aligned}
            \hat{a}_p^\dagger \hat{a}_q^\dagger &= 
                -\hat{a}_q^\dagger \hat{a}_p^\dagger, \\
            \{\hat{a}_p^\dagger, \hat{a}_q^\dagger\} &\equiv 
                \hat{a}_p^\dagger \hat{a}_q^\dagger + 
                \hat{a}_q^\dagger \hat{a}_p^\dagger = \hat{0}.
        \end{aligned}
    \end{equation}
    This is the first of three anti-commutator relations we are going to derive.

    The logical next step is to perform a similar operation with annihilation 
    operators,
    \begin{equation}
        \begin{aligned}
            \hat{a}_p \hat{a}_q \ket{qpijk \dots} &= 
                \hat{a}_p\ket{pijk \dots} = \ket{ijk \dots}, \\
            \hat{a}_q \hat{a}_p \ket{qpijk \dots} &= 
                - \hat{a}_q \hat{a}_p \ket{pqijk \dots} = 
                - \hat{a}_q \ket{qijk \dots} = - \ket{ijk \dots}.
        \end{aligned}
    \end{equation}
    We also require these two operations to be equivalent, 
    \begin{equation}
        \begin{aligned}
            \hat{a}_p \hat{a}_q &= 
                -\hat{a}_q \hat{a}_p, \\
            \{\hat{a}_p, \hat{a}_q\} &\equiv 
                \hat{a}_p \hat{a}_q + 
                \hat{a}_q \hat{a}_p = \hat{0}.
        \end{aligned}
    \end{equation}

    One case remains, when a creation operator and an annihilation operator is
    applied together on a Slater determinant,
    \begin{equation}
        \hat{a}_p^\dagger \hat{a}_q \ket{qijk \dots} 
            = \hat{a}_p^\dagger \ket{ijk \dots}
            = \ket{pijk \dots}.
    \end{equation}
    This operation will replace $\phi_q$ by $\phi_p$ even if $\phi_p$ would have
    been somewhere else in the interior of the Slater determinant. Any sign change
    as an effect of moving the orbital to the front of the string would be negated
    when the orbital is moved back to the original position. Exchanging the order
    of the operators however, gives
    \begin{equation}
        \hat{a}_q \hat{a}_p^\dagger \ket{qijk \dots} 
            = \hat{a}_q \ket{pqijk \dots} 
            = -\hat{a}_q \ket{qpijk \dots}
            = -\ket{pijk \dots}.
    \end{equation}
    We again see a sign change and have,
    \begin{equation}
        \{\hat{a}_p^\dagger, \hat{a}_q\} = \hat{0} \quad (p \neq q).
    \end{equation}
    If, on the other hand, $p=q$ we have 
    \begin{equation}
        \begin{aligned}
            \hat{a}_p^\dagger \hat{a}_p \ket{pijk \dots} &= \ket{pijk \dots}, \\
            \hat{a}_p \hat{a}_p^\dagger \ket{pijk \dots} &= 0,
        \end{aligned}
    \end{equation}
    and if the orbital $\phi_p$ in question does not appear in the Slater 
    determinant,
    \begin{equation}
        \begin{aligned}
            \hat{a}_p^\dagger \hat{a}_p \ket{ijk \dots} &= 0, \\
            \hat{a}_p \hat{a}_p^\dagger \ket{ijk \dots} &= 
                \hat{a}_p\ket{pijk \dots} = \ket{ijk \dots}.           
        \end{aligned}
    \end{equation} 
    For all cases we have that,
    \begin{equation}
        \left(\hat{a}_p^\dagger \hat{a}_p + \hat{a}_p \hat{a}_p^\dagger \right)
        \ket{\dots} = \ket{\dots},
    \end{equation}
    or
    \begin{equation}
        \{\hat{a}_p^\dagger, \hat{a}_p\} = \{\hat{a}_p, \hat{a}_p^\dagger\} = \hat{1}.
    \end{equation}
    In conclusion, the anti-commutator relations of the creation and annihilation
    operators are,
    \begin{align}
        \{\hat{a}_p, \hat{a}_q\} &= \hat{0}, \\
        \{\hat{a}_p^\dagger, \hat{a}_q^\dagger\} &= \hat{0}, \\
        \{\hat{a}_p^\dagger, \hat{a}_q\} &= \{\hat{a}_p, \hat{a}_q^\dagger\} 
         = \delta_{pq},
    \end{align}
    where $\delta_{pq}$ is the Kronecker delta.

\section{Representation of Operators}
    
    Here we shall see that it is very useful to express operators in terms of 
    creation- and annihilation operators. We introduce a general one- and two-body 
    operator. It is possible to create operators pertaining to any number of particles,
    but these are very uncommon to see in quantum chemistry, which is our domain.

    A second-quantised one-body operator is written like
    \begin{equation}
        \label{eq:one_body_operator}
        \hat{h} = \sum_{i=1}^N \hat{h}(i) 
        = \sum_{ij} \bra{i} \hat{h} \ket{j}\hat{a}_i^\dagger \hat{a}_j,
    \end{equation}
    where in general, $\bra{p} \hat{h} \ket{q}$ is the matrix element of the single-particle
    operator $\hat{h}$ in a given one-particle basis,
    \begin{equation}
        h^p_q = \mel{p}{\hat{h}}{q} = \int dx \phi_p(x)^* \hat{h} \phi_q(x).
    \end{equation}
    More accurately, we see from \autoref{eq:one_body_operator}, that $\hat{h}$
    weighs each occupied orbital of a Slater determinant with the appropriate 
    matrix element. 

    A second-quantised two-body operator is written like
    \begin{equation}
        \label{eq:2nd_quant_two_body_operator}
        \hat{u} = \sum_{i, j}^N\hat{u}(i,j) 
       = \frac{1}{2} \sum_{ijkl} u^{pq}_{rs}
            \hat{a}_i^\dagger \hat{a}_j^\dagger \hat{a}_l \hat{a}_k,
        = \frac{1}{4} \sum_{ijkl} \bra{ij}\hat{u}\ket{kl}_\text{AS} 
            \hat{a}_i^\dagger \hat{a}_j^\dagger \hat{a}_l \hat{a}_k,
    \end{equation}
    where 
    \begin{equation}
        \mel{ij}{\hat{u}}{kl} 
        \equiv \int \int \phi_i^*(x_1) \phi_j^*(x_2)u(x_1, x_2) \phi_k(x_1)\phi_(x_2)
            dx_1 dx_2.
    \end{equation} 
    Notice the transposed order of indices in \ref{eq:2nd_quant_two_body_operator}.
    The interpretation of the expression is that a fermion is removed from state 
    $\ket{k}$ and $\ket{l}$ and created in state $\ket{i}$ and $\ket{j}$, with 
    the probability $\mel{ij}{\hat{u}}{kl}$.
    The antisymmetric two-eletron integral for $\hat{u}$ is abreviated,
    \begin{equation}
        \bra{ij} \hat{u}\ket{kl} - \bra{ij}\hat{u} \ket{kl} 
        = \bra{ij}\hat{u} \ket{kl}_\text{AS}
    \end{equation}

    We see that a full second-quantised Hamiltonian can be written,
    \begin{equation}
        \label{eq:second_quant_hamiltionian}
        \hat{H} = \hat{h} + \hat{u} 
        = \sum_{ij} \hat{h}_{ij}\hat{a}^\dagger_i \hat{a}_j
            + \frac{1}{4} \mel{ij}{\hat{u}}{kl}_\text{AS}
            \hat{a}^\dagger_i \hat{a}^\dagger_j \hat{a}_l \hat{a}_k.
    \end{equation}

\section{Normal Order and Wick's Theorem}

    We have built the foundations necessary to describe wavefunctions in terms
    of creation- and annihilation operators as well as a simple way of writing
    a general electronic Hamiltionian in the second-quantised manner. The following
    is a necessity to be able to compute vacuum expectation values 
    $\bra{-} \hat{A} \hat{B} \dots \ket{-}$ of products of creation- 
    and annihilation operators. Such expectation values are very important for 
    several computational methods, see
    \citeauthor{harris1992algebraic} \cite{harris1992algebraic}.

        \subsection{Normal ordering and contractions}

        The normal-ordered product of a string of operators $\hat{A_1}$, $\hat{A_2}$,
        $\hat{A_3}$, $\dots$, is defined as the rearranged product of operators such 
        that all the creation operators are to the left of all the annihilation
        operators, including a phase factor corresponding to the parity of the
        permutation, producing the rearrangement
        \begin{equation} 
            \begin{aligned}
            n[\hat{A}_1\hat{A}_2 \dots \hat{A}_n ]
            &\equiv (-1)^{\abs{\sigma}}
                \hat{A}_{\sigma(1)}\hat{A}_{\sigma(2)} \dots \hat{A}_{\sigma(n)} \\
            &= (-1)^{\sigma(\hat{P})}\hat{P}(\hat{A}_1\hat{A}_2 \dots \hat{A}_n) \\
            &= (-1)^{\abs{\sigma}}
            [\text{creation operators}] \cdot [\text{annihilation operators}] \\
            &= (-1)^{\abs{\sigma}}\hat{a}^\dagger \hat{b}^\dagger \dots \hat{u} \hat{v},
            \end{aligned}
        \end{equation}
        where $\hat{P}$ is a permutation operator acting on the product of operators, 
        and $\sigma$ is the parity of the permutation. One should bear in mind that this 
        definition is by no means unique. Here are some examples,
        \begin{gather*}
            n[\hat{a}^\dagger \hat{b}] = \hat{a}^\dagger \hat{b} \quad
            n[\hat{b} \hat{a}^\dagger] = -\hat{a}^\dagger \hat{b} \\
            n[\hat{a} \hat{b} ] = \hat{a} \hat{b} = - \hat{b} \hat{a} \\
            n[\hat{a}^\dagger\hat{b}^\dagger ]
                = \hat{a}^\dagger \hat{b}^\dagger 
                = -\hat{b}^\dagger \hat{a}^\dagger \\
            n[\hat{a}^\dagger \hat{b} \hat{c}^\dagger \hat{d}]
                = -\hat{a}^\dagger \hat{c}^\dagger\hat{b} \hat{d} 
                = \hat{c}^\dagger \hat{a}^\dagger\hat{b} \hat{d} 
                = \hat{a}^\dagger \hat{c}^\dagger\hat{d} \hat{b} 
                = -\hat{c}^\dagger \hat{a}^\dagger\hat{d} \hat{b}.
        \end{gather*}
        Note that the second quantised Hamiltonian in \autoref{eq:second_quant_hamiltionian}
        is already on normal-ordered form.

        For two arbitrary creation and annihilation operators, we define their contraction
        as 
        \begin{equation}
            \wick{\c {\hat{A}} \c {\hat{B}} } \equiv \bra{-} \hat{A} \hat{B}\ket{-},
        \end{equation}
        or equivalently,
        \begin{equation}
            \wick{\c {\hat{A}} \c {\hat{B}} } \equiv \hat{A}\hat{B} - n[\hat{A} \hat{B}].
        \end{equation}
        For a creation- and annihilation operator there are four possible contractions,
        \begin{equation}
            \begin{aligned}
            \wick{\c{\hat{a}^\dagger} \c{\hat{b}^\dagger}} 
            &= \bra{-}\hat{a}^\dagger \hat{b}^\dagger \ket{-}
            = \hat{a}^\dagger \hat{b}^\dagger - n[\hat{a}^\dagger \hat{b}^\dagger]
            = 0 \\
            \wick{\c{\hat{a}} \c{\hat{b}}} 
            &= \bra{-}\hat{a} \hat{b} \ket{-}
            = \hat{a} \hat{b} - n[\hat{a} \hat{b}]
            = 0 \\
            \wick{\c{\hat{a}^\dagger} \c{\hat{b}}} 
            &= \bra{-}\hat{a}^\dagger \hat{b} \ket{-}
            = \hat{a}^\dagger \hat{b} - n[\hat{a}^\dagger \hat{b}]
            = 0 \\
            \wick{\c{\hat{a}} \c{\hat{b}^\dagger}} 
            &= \bra{-}\hat{a} \hat{b}^\dagger \ket{-}
            = \hat{a} \hat{b}^\dagger - n[\hat{a} \hat{b}^\dagger]
            = \hat{a} \hat{b}^\dagger - (- \hat{b}^\dagger\hat{a})
            = \{\hat{a}, \hat{b}^\dagger \} = \delta_{ab}.
            \end{aligned}
        \end{equation}
        We see that all contractions between creation- and annihilation operators
        are a number, most of them are zero and only those with an annihilation operator
        to the left and a creation operator to the right can be one.
       
		Contractions inside a normal ordered product is defined as follows,
		\begin{equation}
			n[\hat{A}\hat{B}\hat{C} \dots 
			\wick{\c1{\hat{R}} \dots \c2{\hat{S}} \dots 
					\c1{\hat{T}} \dots \c2{\hat{U}} \dots}]
			= (-1)^{\sigma} \wick{\c1{\hat{R}} \c1{\hat{T}} \c2{\hat{S}} \c2{\hat{U}}}
				\dots n[\hat{A}\hat{B}\hat{C}\dots],
		\end{equation}
		where all contracted operator pairs are moved to the front of the normal ordered 
		product, and $\sigma$ is the parity of the permutations required for this relocation.
		The result will be zero, or plus or minus the normal ordered product without the
		contracted operator pairs.

	\subsection{Wick's Theorem}
		Wick's theorem states that every string of creation and annihilation operators can
		be written as a sum of normal-ordered products with all possible contractions,
		\begin{equation}
			\begin{aligned}
			\hat{A}\hat{B}\hat{C}\hat{D}\dots
			=&n[\hat{A}\hat{B}\hat{C}\hat{D}\dots]
			+ n[\wick{\c{\hat{A}}\c{\hat{B}}\hat{C}\hat{D}}\dots]
			+ n[\wick{\c{\hat{A}}\hat{B}\c{\hat{C}}\hat{D}}\dots]
			+ n[\wick{\c{\hat{A}}\hat{B}\hat{C}\c{\hat{D}}}\dots] \\
			&+ \dots
			+ n[\wick{{\hat{A}}\c{\hat{B}}\c{\hat{C}}\hat{D}}\dots]
			+ n[\wick{{\hat{A}}\c{\hat{B}}{\hat{C}}\c{\hat{D}}}\dots]
			+ \dots 
			+ n[\wick{{\hat{A}}{\hat{B}}\c{\hat{C}}\c{\hat{D}}}\dots]
			+ \dots + \\
			&+ n[\wick{\c1{\hat{A}}\c1{\hat{B}}\c2{\hat{C}}\c2{\hat{D}}}\dots]
			+ n[\wick{\c1{\hat{A}}\c2{\hat{B}}\c1{\hat{C}}\c2{\hat{D}}}\dots]
			+ n[\wick{\c2{\hat{A}}\c1{\hat{B}}\c1{\hat{C}}\c2{\hat{D}}}\dots]
			+ \dots,
		\end{aligned}
		\end{equation}
		where eventually all possible contractions of one, two pairs etc, are included.

		Especially when computing vacuum expectation values of normal-ordred products 
        Wick's theorem becomes very important. The reason for this is that each contraction
        will not contribute to the result, unless it is a fully contracted operator string,
		\begin{equation} 
			\bra{}\hat{A} \dots \hat{B} \dots \hat{C} \dots \hat{D} \dots \ket{}
				= \sum_{\substack{\text{all possible} \\ \text{contractions}}}
				\wick{
				\bra{} n[\c1{\hat{A}} \dots \c2{\hat{B}} \dots 
				\c1{\hat{C}} \dots \c2{\hat{D}} \dots] \ket{}
				}.
		\end{equation}

		Most vacuum expectation values contain operator strings that already have substrings
		that are on normal-ordered form. This warrants a very useful generalisation of
		Wick's theorem for such strings,
		\begin{equation}
			\begin{aligned}
			n[\hat{A}_1\hat{A}_2 \dots ]
			n[\hat{B}_1\hat{B}_2 \dots ] \dots
			n[\hat{Z}_1\hat{Z}_2 \dots ]
			= n[\hat{A}_1\hat{A}_2\dots:
				\hat{B}_1\hat{B_2}\dots:
				\dots:
				\hat{Z}_1\hat{Z}_2\dots] \\
			+ \sum_{(1)}
				\wick{n[
				\c{\hat{A}}_1\hat{A}_2\dots:
				\c{\hat{B}}_1\hat{B_2}\dots:
				\dots:
				\hat{Z}_1\hat{Z}_2\dots]}
			+ \dots 
			+ \sum_{(n)}
				\wick{n[		
				\hat{A}_1 \c1{\dots} \c2{\dots} \c3{\dots} 
					\c1{\dots} \c2{\ } \c3{\ } \hat{Z}_N
				]},
			\end{aligned}
		\end{equation}
		where we sum over all combinations of contractions that each involve operators 
		from different substrings, starting with one contractions and up to when all
		operators, or as many as possible, are contracted.

    \subsection{Particle-Hole Formalism}

		We see that a Slater determinant can be built recursively with creation operators,
		\begin{equation}
			\Phi = \hat{i}_1\hat{i}_2\dots \hat{i}_N 
				= \hat{i}_1^\dagger \hat{i}_2^\dagger \dots \hat{i}_N^\dagger\ket{}.
		\end{equation}
		Instead of rewriting Slater determinants with operators applied to the vacuum 
		state in this manner we will introduce the reference state, or 
		Fermi vacuum,
		\begin{equation}
			\ket{0} = \ket{\Phi_0} = \ket{ijk\dots n}. 
		\end{equation}
		We will define other Slater determinants relative to this reference state.
		For instance,
		\begin{align}
		\label{eq:single_excitation}
		\ket{\Phi^a_i} \equiv \hat{a}^\dagger \hat{i}\ket{\Phi_0} &= \ket{ajk\dots n} \\
		\label{eq:double_excitation}
		\ket{\Phi^{ab}_{ij}} \equiv \hat{a}^\dagger\hat{b}^\dagger \hat{j} \hat{i}\ket{\Phi_0}
			&= \ket{abk\dots n} \\
		\label{eq:electron_removal}
		\ket{\Phi_i} \equiv \hat{i}\ket{\Phi_0} &= \ket{jk\dots n} \\
		\label{eq:electron_attachment}
		\ket{\Phi^a} \equiv \hat{a}^\dagger \ket{\Phi_0} &= \ket{aijk\dots n}
		\end{align}
		where equations \ref{eq:single_excitation}, \ref{eq:double_excitation}
		\ref{eq:electron_removal} and \ref{eq:electron_attachment}
		constitute a single excitation, a double excitation, an electorn removal and an
        electron attachment, respectively. Note that the reference state Slater
        determinants excited relative to the Fermi vacuum have the following
        properties,
		\begin{equation}
			\ket{\Phi^{ab}_{ij}} = \ket{\Phi^{ba}_{ji}}
				= - \ket{\Phi^{ba}_{ij}} = - \ket{\Phi^{ab}_{ji}}.
		\end{equation}
		Take note of the specific letters used for creating and annihilating electrons 
		in the example above. $i,j,k,l,\dots$ are letters restricted to indices of 
		\emph{hole} states, and $a,b,c,d,\dots$ are letters restriced to indices of
		\emph{particle} states, while $p,q,r,\dots$ are for general use, indicating 
		any state. Notice that
		\begin{equation}
			\begin{aligned}
				\hat{i}^\dagger\ket{0} = 0 &\quad \hat{a}\ket{0} = 0, \\
				\bra{0}\hat{i} = 0 &\quad \bra{0}\hat{a}^\dagger = 0.
			\end{aligned}	
		\end{equation}
		Whenever we try to insert an electron where there already is one, or when we 
		try to remove an electron that is not there, we get zero as result.

	\subsection{Wick's theorem relative to the Fermi vacuum}

		Now we will modify the concepts of normal-ordering, contractions and Wick's 
        theoreom so that they relate to the reference state and the 
        Fermi vacuum, instead of the physical vacuum.
		First we introduce the pseudo-operators,
		\begin{equation}
			\begin{aligned}
				\hat{b}_i = \hat{i}^\dagger, &\quad \hat{b}_i^\dagger = \hat{i} \\
				\hat{b}_a = \hat{a}, &\quad \hat{b}_a^\dagger = \hat{a}^\dagger,
			\end{aligned}
		\end{equation}
		where $\hat{b}_i^\dagger$ is a hole creation operator and $\hat{b}_i$ is a particle
		creation operator, but only for vacant spaces below the fermi level. The reason 
		for introducing such operators is to be able to work with the fermi vacuum in the 
        same manner as regular operators work with the physical vacuum.
		Excited Slater determinants can easily be written using pseudo-operators,
		\begin{align}
			\ket{\Phi_i^a} &\equiv \hat{b}_a^\dagger\hat{b}_i^\dagger \ket{\Phi_0} \\
			\ket{\Phi_{ij}^{ab}} &\equiv 
				\hat{b}_b^\dagger \hat{b}_j^\dagger \hat{b}_a^\dagger \hat{b}_i^\dagger 
				\ket{\Phi_0}.
		\end{align}

		We introduce a new type of normal ordering for the pseudo-operators and for the
		actual operators that they represent,
		\begin{equation}
			\{\hat{A}\hat{B}\hat{C} \} 
				= (-1)\hat{b}_p^\dagger\hat{b}_q^\dagger \dots \hat{b}_u \hat{b}_v.
		\end{equation}
		We write a contraction in the same manner,
		\begin{equation}
			\wick{\c{\hat{A}}\c{\hat{B}}} = \hat{A}\hat{B} - \{\hat{A}\hat{B}\}.
		\end{equation}
		A normal-ordered product with contractions inside is also defined in the same 
		way. We see that the only non-zero contractions are
		\begin{equation}
			\wick{\c{\hat{b}_i}\c{\hat{b}_j^\dagger}} =	
			\wick{\c{\hat{i}^\dagger}\c{\hat{j}}} = \delta_{ij}, \quad
			\wick{\c{\hat{b}_a}\c{\hat{b}_b^\dagger}} =
			\wick{\c{\hat{a}}\c{\hat{b}^\dagger}} = \delta_{ab}.
		\end{equation}
        Here we are also made aware of the benefit of pseudo-operators, as we now 
        have only have non-zero contributions from contractions that have a creation 
        operator to the right, and an annihilation operator to the left. More generally 
		we have the anticommutator relations
		\begin{equation}
			\{\hat{b}_p, \hat{b}_q^\dagger\} = \delta_{pq}, \quad
			\{\hat{b}_p, \hat{b}_q\} = 0.
		\end{equation}

\clearemptydoublepage