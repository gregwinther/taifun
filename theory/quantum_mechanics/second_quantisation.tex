\chapter{Second Quantisation}

Some Slater determinant is written,
\begin{equation}
    \ket{\Phi} = \ket{\phi_i \phi_j \phi_k \dots \phi_z} = \ket{ijk \dots z}.
\end{equation}

\section{Creation and Annihilation Operators}

    I SHOULD PROBABLY CHANGE TO Qs HERE...?

    The notation of creation and annihilation operators vary,
    \begin{align*}
        \text{creation operator for spinorbital } \phi_i, &\ 
        \hat{X}_i^\dagger, \hat{a}^\dagger_i, \hat{c}_i^\dagger, \hat{i}^\dagger; \\
        \text{annihilation operator for spinorbital } \phi_i, &\ 
        \hat{X}_i, \hat{a}_i, \hat{c}_i, \hat{i}.       
    \end{align*}
    Herein, $\hat{a}_i^\dagger$, $\hat{a}_i$ is used and, if there is no change of 
    confusion, $\hat{i}^\dagger$, $\hat{i}$.

    \paragraph{The Creation Operator}. For every single-particle index $q$,
    we define the creation operator
    $c_q^\dagger$ acting on the vacuum state by
    \begin{equation}
        \hat{a}_i^\dagger \ket{0} = \ket{q}.
    \end{equation}

    For any slater determinant with $N > 0$, the action is defined by
    \begin{align}
        \hat{a}_i^\dagger\ket{jk \dots z} &= \ket{ijk \dots z}, \\
        \hat{a}_i^\dagger\ket{ijk \dots z} &= 0
    \end{align}

    \paragraph{The Annihilation Operator}. It i sufficient to state that the 
    annihilation $c_i$ operator is the hermitian adjoint of the creation operator
    $\hat{c}_i$, but to specify we have
    \begin{equation}
        \hat{a}_i\ket{0} = 0,
    \end{equation}
    as there is no particle in the vacuum state to annihilate. 

    For any arbitrary Slater determinant, we have 
    \begin{align}
        \hat{a}_i \ket{ijk \dots z} &= \ket{ij \dots z}, \\
        \hat{a}_i \ket{jk \dots z} &= 0       
    \end{align}

    SOMETHING MORE ABOUT THE DIFFERENT PERMUTATIONS.

    We can now build a Slater determinant as the result of successive operation of
    several creation operators $\hat{a}_q^\dagger$ on the vacuum state,
    \begin{equation}
        \label{eq:creation_annihilation_slater}
        \hat{a}^\dagger_i \hat{a}^\dagger_j \hat{a}^\dagger_k \dots \hat{a}^\dagger_z
        \ket{0} = \ket{ijk \dots z}.
    \end{equation}

    It is convenient to arrange the spinorbitals in a Slater determinant in
    alphabetical order, as in \autoref{eq:creation_annihilation_slater}. This 
    makes it necessary to ascertain the effects a creation or annihilation
    operator will have on a Slater determinant when the affected orbital is
    not at the beginning of the string of orbitals in the Slater determinant.
    Generally we have,
    \begin{equation}
        \hat{P} \ket{ijk \dots z} = (-1)^{\sigma(\hat{P})} \ket{ijk \dots z},
    \end{equation}
    where $\hat{P}$ permutes the string of orbitals and $\sigma(\hat{P})$ is the
    parity of the permutation $\hat{P}$. we have
    \begin{align}
        \hat{a}_p^\dagger \ket{ijk \dots z} &= 
            (-1)^{\eta_p} \ket{ijk \dots p \dots z}, \\
        \hat{a}_p \ket{ijk \dots p \dots z} &=
            (-1)^{\eta_p} \ket{ijk \dots z},
    \end{align}
    where $\eta_p$ is the number of orbitals preceeding the orbital $\phi_p$, pertaining
    to the creation (annihilation) operator, in the Slater determinant.

\section{Anticommutator Relations}

    Consider some creation operators acting on a Slater determinant,
    \begin{equation}
        \begin{aligned}
            \hat{a}_p^\dagger \hat{a}_q^\dagger \ket{ijk \dots} &= \ket{pqijk \dots} \\
            \hat{a}_q^\dagger \hat{a}_p^\dagger \ket{ijk \dots} &= \ket{qpijk \dots}
                = - \ket{pqijk \dots}.
        \end{aligned}
    \end{equation}
    We demand that these two opartions be equivalent, or that 
    \begin{equation}
        \begin{aligned}
            \hat{a}_p^\dagger \hat{a}_q^\dagger &= 
                -\hat{a}_q^\dagger \hat{a}_p^\dagger \\
            \{\hat{a}_p^\dagger, \hat{a}_q^\dagger\} &\equiv 
                \hat{a}_p^\dagger \hat{a}_q^\dagger + 
                \hat{a}_q^\dagger \hat{a}_p^\dagger = \hat{0}.
        \end{aligned}
    \end{equation}
    This is one of several important anti-commutator relations for creation and
    annihilation operators.

    Similarly, for annihilation operators we have
    \begin{equation}
        \begin{aligned}
            \hat{a}_p \hat{a}_q \ket{qpijk \dots} &= 
                \hat{a}_p\ket{pijk \dots} = \ket{ijk \dots} \\
            \hat{a}_q \hat{a}_p \ket{qpijk \dots} &= 
                - \hat{a}_q \hat{a}_p \ket{pqijk \dots} = 
                - \hat{a}_q \ket{qijk \dots} = - \ket{ijk \dots}.
        \end{aligned}
    \end{equation}
    These two operations must also be equivalent, 
    \begin{equation}
        \begin{aligned}
            \hat{a}_p \hat{a}_q &= 
                -\hat{a}_q \hat{a}_p \\
            \{\hat{a}_p, \hat{a}_q\} &\equiv 
                \hat{a}_p \hat{a}_q + 
                \hat{a}_q \hat{a}_p = \hat{0}.
        \end{aligned}
    \end{equation}

    One case remains, when a creation operator and an annihilation operator is
    applied together on a Slater determinant,
    \begin{equation}
        \hat{a}_p^\dagger \hat{a}_q \ket{qijk \dots} 
            = \hat{a}_p^\dagger \ket{ijk \dots}
            = \ket{pijk \dots}.
    \end{equation}
    This operation will replace $\phi_q$ by $\phi_p$ even if $\phi_p$ would have
    been somewhere else in the interior of the Slater determinant. Any sign change
    as an effect of moving the orbital to the front of the string would be negated
    when the orbital is moved back to the original position. Exchanging the order
    of the operators however,
    \begin{equation}
        \hat{a}_q \hat{a}_p^\dagger \ket{qijk \dots} 
            = \hat{a}_q \ket{pqijk \dots} 
            = -\hat{a}_q \ket{qpijk \dots}
            = -\ket{pijk \dots}.
    \end{equation}
    We again see a sign change and have,
    \begin{equation}
        \{\hat{a}_p^\dagger, \hat{a}_q\} = \hat{0} \quad (p \neq q).
    \end{equation}
    If, on the other hand, $p=q$ we have 
    \begin{equation}
        \begin{aligned}
            \hat{a}_p^\dagger \hat{a}_p \ket{pijk \dots} &= \ket{pijk \dots}, \\
            \hat{a}_p \hat{a}_p^\dagger \ket{pijk \dots} &= 0,
        \end{aligned}
    \end{equation}
    and if the orbital $\phi_p$ in question does not appear in the Slater 
    determinant,
    \begin{equation}
        \begin{aligned}
            \hat{a}_p^\dagger \hat{a}_p \ket{ijk \dots} &= 0, \\
            \hat{a}_p \hat{a}_p^\dagger \ket{ijk \dots} &= 
                \hat{a}_p\ket{pijk \dots} = \ket{ijk \dots}.           
        \end{aligned}
    \end{equation} 
    For all cases we have that,
    \begin{equation}
        \left(\hat{a}_p^\dagger \hat{a}_p + \hat{a}_p \hat{a}_p^\dagger \right)
        \ket{\dots} = \ket{\dots},
    \end{equation}
    or
    \begin{equation}
        \{\hat{a}_p^\dagger, \hat{a}_p\} = \{\hat{a}_p, \hat{a}_p^\dagger\} = \hat{1}.
    \end{equation}
    In conclusion, the anti-commutator relations of the creation and annihilation
    operators are,
    \begin{align}
        \{\hat{a}_p, \hat{a}_q\} &= \hat{0}, \\
        \{\hat{a}_p^\dagger, \hat{a}_q^\dagger\} &= \hat{0}, \\
        \{\hat{a}_p^\dagger, \hat{a}_q\} &= \{\hat{a}_p, \hat{a}_q^\dagger\} 
         = \hat{\delta}_{pq}.
    \end{align}
