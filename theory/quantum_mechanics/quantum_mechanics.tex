\chapter{Quantum Mechanics}
    \epigraph{Hierzu ist es notwendig, die Energy nicht als eine stetige
    unbeschränkt teilbare, sondern als eine discrete, ause einer ganzen
    Zahl von endlichen gleichen Teilen zusammengesetzte Grösse 
    aufzufassen.}
    {--- Max Planck}


\section{Classical Mechanics}

    The formalism used in quantum mechanics largely stems from William Rowan Hamilton's 
    formulation of classical mechanics. Through the process of canonical
    quantisation any classical model of a physical system is turned into
    a quantum mechanical model.

    In Hamilton's formulation of classical mechanics, a complete description of a system
    of N particles is described by a set of canonical coordinates 
    $q = (\vec{q}_1, \dots, \vec{p}_N)$ and corresponding conjugate momenta
    $p = (\vec{p}_1, \dots \vec{p}_N)$. Together, each pair of coordinate and momentum
    form a point $\xi = (q, p)$ in phase space, which is a space of all possible states
    of the system. Moreover, pairs of generalised coordinates and conjugate
    momenta are canonical if they satisfy the Poisson brackets so that 
    $\{q_i, p_k\} = \delta_{ij}$. The Poisson bracket of two functions is defined as 

    \begin{equation}
        \{f, g\} = \frac{\partial f}{\partial q} \frac{\partial g}{\partial g}
        - \frac{\partial f}{\partial p} \frac{\partial g}{\partial q}.
    \end{equation}

    The governing equations of motion in a classical system is Hamilton's equations,

    \begin{align}
        \dot{q} &= \frac{\partial}{\partial p} \mathscr{H}(q, p) \\
        \dot{p} &= -\frac{\partial}{\partial q} \mathscr{H}(q, p)
    \end{align}

    where $\mathscr{H}(q, p)$ is the Hamiltonian, a function for the total energy of the
    system. Hamilton's equations may also be stated in terms of the Poisson brackets,

    \begin{equation}
        \frac{dp_i}{dt} = \{p_i, \mathscr{H}\}, \ \frac{dq_i}{dt} = \{q_i, \mathscr{H}\}.
    \end{equation}

    A system consisting of $N$ of equal mass $m$, subject forces caused by an external
    potential, as well as acting on each other with forces stemming from a central
    potetntial $w(q_ij)$ has the following Hamiltonian,

    \begin{equation}
        \mathscr{H}(q, p) = \mathscr{T}(q) + \mathscr{V}(p) + \mathscr{W}(p) 
            = \frac{1}{2m}\sum_{i} \abs{\vec{p}_i}^2 + \sum_{i} v(\vec{r}_i)
                + \frac{1}{2}\sum_{i<j} w(\vec{r_{ij}}).
    \end{equation}

    This Hamiltonian conveniently contains several parts - the kinetic energy, the
    external potential energy and the interaction energy; denoted by $\mathscr{T}$,
    $\mathscr{V}$ and $\mathscr{W}$ respectively,

\section{The Dirac-von Neumann Postulates}

    This is the TL;DR version of Quantum Mechanics.

    \paragraph{Hilbert Space}
    A quantum state of an isolated physical system is described by a vector
    with unit norm in a Hilbert space, a complex vector space quipped with a
    scalar product. 

    \paragraph{Observables}
    Each physical observable of a system is accociated with a \emph{hermitian}
    operator acting on the Hilbert space. The eigenstates of each such
    operator define a \emph{complete}, \emph{orthonormal} set of vectors.

    With $\hat{O}$ an operator, hermiticity means,
    \begin{equation}
        \bra{\phi}\hat{O}\psi\rangle = \langle\hat{O}\phi\ket{\psi} \equiv \bra{\phi} \hat{O} \ket{\psi}.
    \end{equation}

    Completness means,
    \begin{equation}
        \sum_i \ket{i}\bra{i} = \mathds{1}.
    \end{equation}

    Orthonormal means,
    \begin{equation}
        \braket{i}{j} = \delta_{ij}.
    \end{equation}

    \paragraph{Time Evolution}
    The time evolution of the state vector, $\ket{\psi} = \ket{\psi (t)}$, is given by the Schrödinger
    equation\footnote{In the Schrödinger picture.}.
    \begin{equation}
        i\hbar \frac{d}{dt} \ket{\psi (t)} = \hat{H} \ket{\psi (t)}.
    \end{equation}

    \paragraph{Measurements}
    Physically measurable values, associated with an obeservable $\hat{O}$ are defined by the 
    eigenvalues $o_n$ of the observable,
    \begin{equation}
        \hat{O}\ket{n} = o_n\ket{n}.
    \end{equation}
    The probability for finding a particular eigenvalue in the measurement is
    \begin{equation}
        p_n = \abs{\braket{n}{\psi}}^2,
    \end{equation}
    with the system in state $\ket{\psi}$ before the measurement, and $\ket{n}$ as the 
    eigenstate corresponding to the eigenvalue $o_n$.