\chapter{Quantum Mechanics}
    % \epigraph{Hierzu ist es notwendig, die Energy nicht als eine stetige
    % unbeschränkt teilbare, sondern als eine discrete, ause einer ganzen
    % Zahl von endlichen gleichen Teilen zusammengesetzte Grösse 
    % aufzufassen.}
    % {--- Max Planck}

    Here we present basic and foundational quantum theory, a theory that seeks to 
    describe the nature at the smallest scales of energy. The name ``quantum'' stems 
    from the need to see energy not as continuous and infinitely divisible 
    , but rather 
    as a sum of discrete quantities of equal size. Quantum mechanics came to the 
    rescue when classical physics was unable to explain phenomena such as the mysterious 
    sodium line, the ultraviolet catastrophe, and the bewildering photo-electric effect.
    With the new theory came new challenges like an axiomatic uncertainty, 
    the wave-particle duality of light and a 
    probabilistic interpretation of nature. In this chapter we will not be so hubristic 
    as to delve into the philosophical particulars, but hope only to revitalise the 
    reader's faculties with the principal ideas on which this thesis is built.

\section{Classical Mechanics}

    The formalism used in quantum mechanics largely stems from William Rowan Hamilton's 
    formulation of classical mechanics. Through the process of canonical
    quantisation any classical model of a physical system is turned into
    a quantum mechanical model.

    In Hamilton's formulation of classical mechanics, a complete description of a system
    of $N$ particles is described by a set of canonical coordinates 
    $q = (\vec{q}_1, \dots, \vec{q}_N)$ and corresponding conjugate momenta
    $p = (\vec{p}_1, \dots \vec{p}_N)$. Together, each coordinate-momentum pair
    forms a point $\xi = (q, p)$ in phase space, which is the space of all possible states
    of the system. Moreover, pairs of generalised coordinates and conjugate
    momenta are canonical if they satisfy the Poisson brackets so that 
    $\{q_i, p_k\} = \delta_{ij}$. The Poisson bracket of two functions is defined as 

    \begin{equation}
        \label{eq:poisson_bracket}
        \{f, g\} = \frac{\partial f}{\partial q} \frac{\partial g}{\partial g}
        - \frac{\partial f}{\partial p} \frac{\partial g}{\partial q}.
    \end{equation}

    The governing equations of motion in a classical system is Hamilton's equations,

    \begin{align}
        \dot{q} &= \frac{\partial}{\partial p} \mathscr{H}(q, p) \\
        \dot{p} &= -\frac{\partial}{\partial q} \mathscr{H}(q, p)
    \end{align}

    where $\mathscr{H}(q, p)$ is the Hamiltonian, a function for the total energy of the
    system. Hamilton's equations may also be stated in terms of the Poisson brackets,

    \begin{equation}
        \frac{dp_i}{dt} = \{p_i, \mathscr{H}\}, \ \frac{dq_i}{dt} = \{q_i, \mathscr{H}\}.
    \end{equation}

    A system consisting of $N$ particles of equal mass $m$, subject to forces caused by an external
    potential, as well as acting on each other with forces stemming from a central
    potential $w(q_i, q_j)$ has the following Hamiltonian,

    \begin{equation}
        \mathscr{H}(\vb{q}, \vb{p}) = \mathscr{T} + \mathscr{V} + \mathscr{W} 
            = \frac{1}{2m}\sum_{i} \abs{\vec{p}_i}^2 + \sum_{i} v(\vec{q}_i)
                + \frac{1}{2}\sum_{i<j} w(\vec{q_i}, \vec{q_j}),
    \end{equation}
    where the sum over $i<j$ implies sum over different indices. 
    This Hamiltonian conveniently contains several parts - the kinetic energy, the
    external potential energy and the interaction energy; denoted by $\mathscr{T}$,
    $\mathscr{V}$ and $\mathscr{W}$ respectively.

\section{Canonical Quantisation}

    In order to transition from a classical system to a quantum system, we move from 
    the classical phase space to the Hilbert space, through the procedure known as 
    canonical, or first\footnote{Second quantisation comes later.}-, quantisation.
    Whilst the state of a classical system is a point in phase space, a quantum state
    is a complex-valued state vector in discrete, infinite-dimensional, Hilbert space.
    A physicist would define a Hilbert space as a complete vector space equipped
    with an inner product. This space
    is most commonly chosen to be the space of square-integrable functions $\Psi$,
    dependent on all coordinates
    
    \begin{equation}
        \Psi = \Psi(x_1, x_2, \dots, x_N).
    \end{equation}
    
    These functions
    are dubbed wavefunctions and are maps from a point $(x_1, \dots, x_N)$ in
    configuration space to the complex vector space,

    \begin{equation}
        \Psi: X^N \to \mathds{C}.
    \end{equation}

    It has been widely discussed how such an object can represent the state of a 
    particle. One answer is provided by Max Born's probabilistic interpretation,
    which says that $\abs{\Psi(x_1, \dots, x_N)}^2$, gives the probability of finding
    the particle at a certain position. For a situation with one particle in one
    dimension we have,

    \begin{equation}
        \int_a^b \abs{\Psi(x)}^2 dx = 
        \left\{\begin{aligned}
            \text{probability of finding the} \\
            \text{particle between $a$ and $b$}
        \end{aligned}\right\}
    \end{equation}

    while $\abs{\Psi(x_1, x_2, \dots, x_N)}^2$ is the probability density for locating
    all particles at the point $(x_1, \dots x_N) \in X^N$. Since the total probability
    must be 1, we are provided with a normalisation condition for the wavefunction,
    
    \begin{equation}
        \int_{X^N} \abs{\Psi(x_1, x_2, \dots, x_N)}^2 dx_1 dx_2\dots dx_N = \mathds{1}.
    \end{equation}

    The relation between a classical- and quantum description of a mechanical system 
    is most clearly seen when the two descriptions are expressed in terms of the same 
    variables. In fact, we may apply \emph{quantisation} of the classical variables to 
    produce the quantum equivalent of the system. The classical phase space variables 
    are changed into quantum observables,
    \begin{equation}
        q_i \to \hat{q}_i, \ p_i \to \hat{p}_i.
    \end{equation}
    The quantum observables are required to satisfy the Heisenberg commutation relation,
    \begin{equation}
        [\hat{q}_i, \hat{p}_j] = i \hbar \delta_{ij},
    \end{equation}
    instead of the Poisson bracket from \autoref{eq:poisson_bracket}. Here $\hbar$ is the 
    reduced Planck's constant. For any general variables $A$ and $B$, this 
    transition can be expressed by substitution between Poisson brackets for the 
    classical variables and commutators for the quantum observables,
    \begin{equation}
        \{A, B\} \to \frac{1}{i\hbar}[\hat{A},\hat{B}].
    \end{equation}

    This correspondence between the classical and quantum dynamical equations is 
    directly related to \emph{Ehrenfest's therom}, which states that the classical 
    dynamical equations keep their validity also in the quantum theory, with the 
    classical variables replaced by their corresponding quantum expectation
    values\cite{ehrenfest1927bemerkung}.

\subsection{The Dirac-von Neumann Postulates}

    The following postulates, or axioms, provide a precise and concise description 
    of quantum mechanics in terms of operators on the Hilbert space. There are
    many variations of these postulates, two of which were introduced by the
    namesakes of the postulates, Paul Adriene Maurice Dirac \cite{dirac1930principles}
    and John von Neumann \cite{vonNeumann1932foundations}.

    \subsubsection{Hilbert Space}
    A quantum state of an isolated physical system is described by a vector
    with unit norm in a Hilbert space $\mathcal{H}$, 
    a complex vector space equipped with an inner product. The inner product 
    associates a scalar value, which may be either real or complex, with any pair
    of state vectors.
    
    The inner product can be defined as 
    \begin{equation}
        \braket{\psi_\alpha}{\psi_\beta} = \int \psi_\alpha^*(x) \psi_\beta(x)dx,
    \end{equation}
    where $\psi^*$ is the complex conjugate of $\psi$. Here we have introduced 
    Dirac notation, which is very common when describing quantum states. For each 
    quantum state $\ket{\psi_\alpha}$ there exists a dual state $\bra{\psi_\alpha}$.
    We refer to these two vectors as \emph{bra} and \emph{ket} vectors, respectively.
    Some properties of an inner product, written in Dirac's style, read
    \begin{align}
        \braket{\psi_\alpha}{\psi_\beta} =& \braket{\psi_\beta}{\psi_\alpha}^* \\
        \bra{\psi_\alpha}\left(z_1\ket{\psi_\beta} + z_2 \ket{\psi_\beta}\right)
            =& z_1 \braket{\psi_\alpha}{\psi_\beta} + z_2\braket{\psi_\alpha}{\psi_\beta} \\
        \braket{\psi_\alpha} \geq& 0,
    \end{align}
    where $z_n = a_n + ib_n$ is a complex number. Notice that in these properties, a 
    superposition of a state wavefunction in the form of linear combination of two other 
    states have appeared. Such superpositions are generally written
    \begin{equation}
        \label{eq:superposition}
        \ket{\psi_\gamma} = z_1\ket{\psi_\alpha} + z_2\ket{\psi_\beta},
    \end{equation}
    where we have produced a new quantum state from two other states. Any two or more
    states may be superposed to produce a new state in this manner. Superposition is of fundamental 
    importance to quantum mechanics, and even though the concept is similar to the 
    classical superposition principle for waves in classical physics, "the superposition 
    that occurs in quantum mechanics is of an essentially different nature from any 
    occuring in the classical theory" according 
    to \citeauthor{dirac1930principles} \cite{dirac1930principles}.

    To conclude the description of quantum states for now; a state function $\psi$ is said 
    to be \emph{normal} if its innerproduct with itself is one, $\braket{\psi}{\psi} = 1$.
    Two different state functions are \emph{orthogonal} if their inner product is zero. 
    We have orthogonal functions if $\braket{\psi_\alpha}{\psi_\beta} = \delta_{\alpha\beta}$,
    where $\delta_{\alpha\beta}$ is the Kronecker delta.


    \subsubsection{Observables}
    Each physical observable of a system is associated with a \emph{Hermitian}
    operator acting on the Hilbert space. The eigenstates of each such
    operator define a \emph{complete}, \emph{orthonormal} basis set of vectors
    $\mathcal{B}$ for the $d$-dimensional Hilbert space,
    \begin{equation}
        \mathcal{B} = \{\ket{i}\}_{i=1}^d.
    \end{equation}
    Completeness of the basis set $\mathcal{B}$ means,
    \begin{equation}
        \sum_{i=1}^d \ket{i}\bra{i} = \mathds{1}.
    \end{equation}

    With $\hat{O}$ an operator, \emph{hermiticity} means,
    \begin{equation}
        \bra{\phi}\hat{O}\psi\rangle = \langle\hat{O}\phi\ket{\psi} \equiv \bra{\phi} \hat{O} \ket{\psi}.
    \end{equation}
    This implies that the operator $\hat{O}$ must be its own Hermitian conjugate,
    \begin{equation}
        \hat{O}^\dagger = \hat{O}.
    \end{equation}
    Some properties of the Hermitian conjugate read,
    \begin{align}
       (z\hat{O})^\dagger =& z^*\hat{O}^\dagger \\
       (\hat{O}_1 + \hat{O}_2)^\dagger =& \hat{O}_1^\dagger + \hat{O}_2^\dagger \\
       (\hat{O}_1\hat{O_2})^\dagger =& \hat{O}_2^\dagger \hat{O}_1^\dagger.
    \end{align}

    \subsubsection{Measurements}
    Physically measurable values, associated with an observable $\hat{O}$ are defined by the 
    eigenvalues $o_n$ of the observable,
    \begin{equation}
        \hat{O}\ket{n} = o_n\ket{n},
    \end{equation}
    where $\ket{n}$ are the eigenvectors of the same observable $\hat{O}$.
    The probability for finding a particular eigenvalue in the measurement is
    \begin{equation}
        p_n = \abs{\braket{n}{\psi}}^2,
    \end{equation}
    with the system in state $\ket{\psi}$ before the measurement, and $\ket{n}$ as the 
    eigenstate corresponding to the eigenvalue $o_n$.
    If the observable $\hat{O}$ is Hermitian, we can write the operators as a spectral 
    decomposition 
    \begin{equation}
        \hat{O} = \sum_{n=1}^d o_n \dyad{n},
    \end{equation}
    where $d$ is the dimensionality of the Hilbert space.

    \subsubsection{Time Evolution}
    In the \emph{Schrödinger picture}
    time evolution of the state vector, $\ket{\psi} = \ket{\psi (t)}$, is given by the Schrödinger
    equation,
    \begin{equation}
        i\hbar \frac{d}{dt} \ket{\psi (t)} = \hat{H} \ket{\psi (t)}.
    \end{equation}
    Note that any superposed state, as described by \ref{eq:superposition}, will also be 
    a solution to the Schrödinger equation due to its linearity.

    The Schrödinger equation is first order in the time derivative, meaning that the time 
    evolution $\ket{\psi} = \ket{\psi(t)}$ is uniquely determined by some initial 
    condition $\ket{\psi_0} = \ket{\psi(t_0)}$. $\hat{H}$ is the Hamiltonian of the system,
    which is a \emph{linear}, \emph{hermitian} operator. The Hamiltonian gives rise to the 
    time evolution, which is a \emph{unitary} mapping between quantum states in time,
    \begin{equation}
        \ket{\psi(t)} =  \hat{\mathcal{U}}(t, t_0) \ket{t_0}.
    \end{equation}
    The time evolution operator $\hat{\mathcal{U}}$ is determined by the Hamiltonian 
    through the equation 
    \begin{equation}
        i\hbar \frac{\partial}{\partial t} \hat{\mathcal{U}}(t, t_0)
            = \hat{H}\hat{\mathcal{U}}(t, t_0),
    \end{equation}
    which follows from the Schrödinger equation. For a time-independent 
    Hamiltonian it is given by 
    \begin{equation}
        \hat{\mathcal{U}}(t, t_0) = e^{-i \hat{H}(t - t_0/\hbar)}.
    \end{equation}
    We see that this time-propagator is Hermitian 
    \begin{equation}
        \hat{\mathcal{U}}\hat{\mathcal{U}}^\dagger 
        = \hat{\mathcal{U}}^\dagger \hat{\mathcal{U}} = 1.
    \end{equation}
    If however $\hat{H}$ is time-dependent, so that the operator at different times do 
    not commute, we may use a more general integral expression for the time-propagator,
    \begin{equation}
        \hat{\mathcal{U}}(t, t_0)
         = \sum_{n=0}^\infty (-\frac{i}{\hbar})^n
            \int_{t_0}^t \int_{t_0}^{t_1}dt_1 \int_{t_0}^{t_1}dt_2 \dots \int_{t_0}^{t_{n-1}}
                dt_n \hat{H}(t_1)\hat{H}(t_2) \dots \hat{H}(t_n).
    \end{equation}

    A unitary transformation of states and observables
    \begin{equation}
        \ket{\psi} \to \ket{\psi'} = \hat{U}\ket{\psi}, \ 
        \hat{O} \to \hat{O}' = \hat{U}\hat{O}\hat{U}^\dagger, \ 
        \hat{U}^\dagger\hat{U} = 1.
    \end{equation}
    leads to a different, but equivalent representation of a quantum system.
    The transition to the \emph{Heisenberg picture} is defined by a special
    time-dependent unitary transformation,
    \begin{equation}
        \hat{U} = \hat{\mathcal{U}}^\dagger(t, t_0),
    \end{equation}
    which is the inverse of the time-evolution operator. When applied to the 
    time-dependent state vector of the Schrödinger picture it will cancel 
    the time-dependence
    \begin{equation}
        \ket{\psi}_\text{H} 
        = \hat{\mathcal{U}}^\dagger(t, t_0) \ket{\psi(t)}_\text{S}
        = \ket{\psi(t_0)}_\text{S}.
    \end{equation}
    The time-dependence is now carried by the observables, rather than the state 
    vectors,
    \begin{equation}
        \hat{O}_\text{H} 
        = \hat{\mathcal{U}}^\dagger(t, t_0)
        \hat{O}_\text{S}
        \hat{\mathcal{U}}(t, t_0),
    \end{equation}
    and the Schrödinger equation is replaced by the Heisenberg equation
    \begin{equation}
        \frac{d}{dt} 
        = \frac{i}{\hbar}[\hat{H}, \hat{O}_\text{H}]
        + \frac{\partial}{\partial t} \hat{O}_\text{H}.
    \end{equation}

    There is a third representation called the \emph{interaction picture}, but we will 
    remain firmly rooted in the Schrödinger picture and halt this general introduction 
    to time-development here. 

\section{The Quantum Hamiltonian}

    The full Hamiltionian for a quantum many-body system can be a large 
    and unwieldy thing. In this study we will constrain ourselves to the study of
    electronic systems. On a phenomenological basis, one would include nuclear
    terms in the Hamiltonian as well. In this study however, we will stay within the
    Born-Oppenheimer approximation and treat the nuclei as stationary, thereby
    refraining from introducing terms that involve the motion of nuclei.

    The full  molecular electronic Breit-Pauli Hamiltionian, thoroughly described in 
    Helgaker et al.\cite{helgaker2012recent}, contains the following types of terms
    \begin{equation}
        \label{eq:breit_pauli_hamiltonian}
        \hat{H}^{\text{BP}}_{\text{mol}}
        = \begin{cases}
            \hat{H}_{\text{kin}} \quad &\leftarrow \text{kinetic energy} \\
            + \hat{H}_{\text{cou}} \quad &\leftarrow \text{Coulomb interactions} \\
            + \hat{H}_{\text{ee}} \quad &\leftarrow \text{external electric field interaction} \\
            + \hat{H}_Z \quad &\leftarrow \text{Zeeman interactions} \\
            + \hat{H}_{\text{so}} &\leftarrow \text{spin-orbit interactions} \\
            + \hat{H}_{\text{ss}} &\leftarrow \text{spin-spin interactions} \\
            + \hat{H}_{\text{oo}} &\leftarrow \text{orbit-orbit interactions} \\
            + \hat{H}_{\text{dia}} &\leftarrow \alpha^4 \text{ diamagnetic interactions}
        \end{cases}
    \end{equation}
    We will not be working the full Breit-Pauli Hamiltionian, but we will go into some of the
    most import terms that an electronic Hamiltonian can constitute. 
    
    \paragraph{Kinetic energy}

    The genereal kinetic energy operator is given by
    \begin{equation}
        \label{eq:kinetic_energy_term}
        \hat{H}_{\text{kin}}
            = - \frac{\hbar}{2m}\sum_i \nabla^2_i,
    \end{equation}
    where $\nabla$ is the differential operator and the sum is over all electrons in 
    the system. This term is an example of what we call a one-particle operator as
    it remains the 
    same for all electrons and contains no terms that would represent interactions 
    between particles. For a free particle or a gas of non-interacting particles, 
    \autoref{eq:kinetic_energy_term} is sufficient to describe the entire system. 

    \paragraph{Potential terms}

    Adding a confining potential to the Hamiltonian in addition to the kinetic 
    energy term in \autoref{eq:kinetic_energy_term},
    \begin{equation}
        \hat{H} \supset \hat{H}_\text{kin} + \hat{V}
    \end{equation}
    gives rise to much more interesting systems and is the beginning of an approximation 
    of reality. Perhaps the most common is the harmonic oscillator potential, which 
    in one dimension reads 
    \begin{equation}
        \hat{V}(x) = \frac{1}{2}m\omega^2x^2,
    \end{equation}
    where $m$ is the mass of the particles and $\omega$ is the (angular) frequency 
    of oscillation.

    This is a very popular confining potential because virtually any oscillatory motion can 
    be approximated by it, if the amplitude of the oscillations is sufficiently small.
    This parabolic, harmonic
    potential is the basis of a quantum dot which is a central part of this study. 
    In quantum chemistry we only consider potentials derived from particle-particle
    interactions, and not such external potentials.

    \paragraph{Coulomb interactions}

    The electrostatic interaction between particles in a molecule or atom is 
    modelled by Coulomb terms in the Hamiltonian. 
    \begin{equation}
            \hat{H}_{\text{cou}} 
                = -\sum_{iK}\frac{k_eZ_Ke^2}{r_{iK}} 
                    + \frac{1}{2}\sum_{i \neq j} \frac{k_e e^2}{r_{ij}} 
                    + \frac{1}{2}\sum_{K \neq L} \frac{k_eZ_K Z_Le^2}{R_{KL}},
    \end{equation}
    where $e$ is the elementary particle charge and $k_e = 1 / 4\pi\epsilon_0$ is the 
    Coulomb constant.
    The first term is the potential between nuclei and electrons, the second term is
    the potential between electrons and the last term is the potential between 
    nuclei.

    \paragraph{External electric field interactions}
   
    Now comes the time to go through, in broadest of strokes, a quantisation of an electromagnetic 
    field. For a thorough derivation see for instance 
    \citeauthor{joachain2012atoms}\cite{joachain2012atoms}.
    When we include an electromagnetic field in the model it is necessary to include terms in
    the Hamiltonian that 
    model the effects of an externally applied scalar potential $\phi = \phi(\vb{r}, t)$
    and a vector potential $\vb{A} = \vb{A}(\vb{r}, t)$. This will also affect the 
    kinetic energy of the particles, which we therefore include at first,
    \begin{equation}
        \hat{H}_\text{ef} 
            \subset \frac{1}{2m}( e \vb{A} - \vb{p})^2 + e\phi
            = \frac{\hat{\vb{p}^2}}{2m} - \frac{e}{2m}(\vb{A}\cdot\hat{\vb{p}} + \hat{\vb{p}}\cdot\vb{A})
                + \frac{e^2}{2m}\vb{A}^2 + e\phi.
    \end{equation}
    This Hamiltonian now describes a free particle subject to an external 
    electric field. Since we have already included a term for kinetic energy we 
    now wish to remove it, keeping only the new terms,
    \begin{equation}
        \hat{H}_\text{ef} 
            = - \frac{e}{2m}(\vb{A}\cdot\hat{\vb{p}} + \hat{\vb{p}}\cdot\vb{A})
                + \frac{e^2}{2m}\vb{A}^2 + e\phi.
    \end{equation}

    We assume that the external field has sufficiently large wavelength compared to the 
    system, making the vector potential uniform in space $\vb{A}(\vb{r}, t) = \vb{A}(t)$.
    This approximation is very reasonable, considering visible light has a wavelength of 
    $\lambda \sim 5000 \text{Å}$ and the diameter of an atom is around $1 \text{Å}$. In the dipole 
    approximation we can write the vector potential as 
    \begin{equation}
        \label{eq:photon_kill_create}
        \vb{A}(t) = \vb{A}_0e^{-i \omega_k t} + \vb{A}_0^*e^{i\omega_k t},
    \end{equation}
    where $\vb{A}_0$ and $\vb{A}_0^*$ are photon creation and annihilation operators,
    and $\omega_k$ is the angular frequency of the field.
    The photon creation and annihilation operators allow for spontaneous emission and absorption of photons without the presence 
    of a field, but we will disregard such phenomena and stick to a semi-classical 
    description. We therefore rewrite the vector potential as 
    \begin{equation}
        \vb{A}(t) = \boldsymbol{\epsilon} A_0\sin(\omega_k t),
    \end{equation}
    where $\boldsymbol{\epsilon}$ is the polarisation vector.
    This expression is the same as \autoref{eq:photon_kill_create} up to a phase. In the Coulomb gauge we have 
    \begin{equation}
        \vb{E} = - \frac{d}{dt}\vb{A},
    \end{equation}
    which gives us an expression for the E-field 
    \begin{equation}
        \vb{E}(t) = \boldsymbol{\epsilon} \vb{E}_0 \cos(\omega_k t),
    \end{equation}
    where $\vb{E}_0 = -\omega_k A_0$ and we can approximate the external
    field time-dependent electric field by 
    \begin{equation}
        \hat{H}_\text{ef} = - \hat{\vb{d}} \cdot \boldsymbol{\epsilon} \vb{E}_0 \cos{\omega_k t},
    \end{equation}
    where $\hat{\vb{d}} = q\hat{\vb{r}}$, is the dipole operator, dictating the 
    allowed transitions.

    \subsection{Angular Momentum and Intrinsic Spin}

    In general, modelling of interactions with magnetic fields necessitates the 
    use of operators for intrinsic angular momentum ($\hat{\vb{S}}$) and extrinsic 
    angular momentum ($\hat{\vb{L}}$). These are often referred to as \emph{spin} 
    and \emph{angular momentum}, respectively.
    We will spend some time here elaborating on such terms.

    \subsubsection{Angular momentum}
    In a classical system, the angular momentum of a particle with 
    respect to the origin is given as
    \begin{equation}
        \vb{L} = \vb{r} \times \vb{p},
    \end{equation}
    which broken down into components becomes,
    \begin{equation}
        L_x = y p_z - z p_y,\ 
        L_y = y p_x - x p_z,\ 
        L_z = x p_y - y p_x.
    \end{equation}
    From this we can obtain the quantum mechanical description by promotion to operators,
    and inserting the representation for the momentum operator in position space,
    $p_x \to i\hbar \partial / \partial x$.

    The commutators of the angular momentum operators obey the following cyclic
    permutation of indices rule,
    \begin{equation}
        [\hat{L}_x, \hat{L}_y] = i\hbar \hat{L}_z, \  
        [\hat{L}_y, \hat{L}_z] = i\hbar \hat{L}_x, \ 
        [\hat{L}_z, \hat{L}_x] = i\hbar \hat{L}_y.
    \end{equation}
    We see that they do not commute with each other, but the square of the total 
    angular momentum, defined by 
    \begin{equation}
        \hat{L}^2 \equiv \hat{L}_x^2 + \hat{L}_y^2 + \hat{L}_z^2,
    \end{equation}
    does
    \begin{equation}
        [\hat{L}^2, \hat{L}_x] = 0, \ 
        [\hat{L}^2, \hat{L}_y] = 0, \ 
        [\hat{L}^2, \hat{L}_z] = 0.
    \end{equation}

    In spherical coordinates, which is better suited for our needs, the angular 
    momentum operator is given by
    \begin{equation}
        \hat{\vb{L}} = -i\hbar \hat{\vb{r}} \times \nabla.
    \end{equation}
    Given the gradient in spherical coordinates,
    \begin{equation}
        \nabla = 
          e_r \frac{\partial}{\partial} 
        + e_\theta \frac{1}{r} \frac{\partial}{\partial \theta}
        + e_\phi \frac{1}{r\sin \theta} \frac{\partial}{\partial \phi}
    \end{equation}
    and $\vb{r} = re_r$ we get 
    \begin{equation}
        \begin{aligned}
        \hat{\vb{L}}
        &= 
        -i\hbar\left[ 
              r(e_r \times e_r)\frac{\partial}{\partial r} 
            + (e_r \times e_\theta)\frac{\partial}{\partial \theta}
            + (e_r \times e_\phi) \frac{1}{\sin\theta} \frac{\partial}{\partial \phi}  
        \right] \\
        &=
        -i\hbar \left( 
        e_\phi \frac{\partial}{\partial \theta}
        - e_\theta \frac{1}{\sin\theta}\frac{\partial}{\partial \phi}     
        \right).
        \end{aligned}
    \end{equation}
    Now, with a bit of algebra we get
    \begin{align}
        \hat{L}_x &= i\hbar \left(
            \sin\phi \frac{\partial}{\partial\theta}
            +\cos\phi\cot\theta\frac{\partial}{\partial \phi}
        \right) \\
        \hat{L}_y &= i\hbar \left(
            \cos\phi \frac{\partial}{\partial\theta}
            +\sin\phi\cot\theta\frac{\partial}{\partial \phi}
        \right) \\
        \hat{L}_z &= -i\hbar \frac{\partial}{\partial \phi}
    \end{align}
    The squared operator becomes
    \begin{equation}
        \hat{L}^2 = -\hbar^2 \left[
        \frac{1}{\sin\theta}\frac{\partial}{\partial\theta}
            \left(\sin\theta\frac{\partial}{\partial\theta} \right)
        + \frac{1}{\sin^2\theta}\frac{\partial^2}{\partial\phi^2}
        \right].
    \end{equation}
    The eigenvalue equations of $\hat{L}^2$ and $\hat{L}_z$ are
    \begin{equation}
        \hat{L}^2\psi = \hbar^2l(l + 1)\psi, \quad \hat{L}_z\psi = \hbar m \psi,
    \end{equation}
    where $\psi = Y^m_l(\theta, \phi)$ are the spherical harmonics,
    \begin{equation}
        Y^m_l (\theta, \phi) 
            = 
            \epsilon\sqrt{\frac{(2l + 1)}{4\pi}\frac{(l - |m|)!}{(1 + |m|)!}}
            e^{im\theta} P^m_l(cos\theta),
    \end{equation}
    and $P^m_l$ are the associated Legendre polynomials.

    \subsubsection{Spin}

    In classical mechanics, intrinsic spin ($\vb{S} = I\omega$) is associated with
    an object's motion about its centre of mass. A similar thing goes on in quantum 
    mechanics, but it has nothing to do with motion in space. In quantum mechanics spin 
    is seen as a property that particles can carry, but is analoguous with its classical 
    counterpart only in name.

    Algebraically, \emph{spin} is the same as \emph{angular momentum}, beginning with 
    the commutator relations,
    \begin{equation}
        [\hat{S}_x, \hat{S}_y] = i\hbar \hat{S}_z, \ 
        [\hat{S}_y, \hat{S}_z] = i\hbar \hat{S}_x, \ 
        [\hat{S}_z, \hat{S}_x] = i\hbar \hat{S}_y,
    \end{equation}
    while the eigenvectors of $\hat{S}^2$ and $S_z$ satisfy 
    \begin{equation}
        \hat{S}^2\ket{s\ m_s} = \hbar^2 s(s + 1)\ket{s\ m_s}, \quad
        \hat{S}_z\ket{s\ m_s} = \hbar m \ket{s\ m_s}.
    \end{equation}
    Here,$\ket{s\ m_s}$ is an eigenstate determined by the quantum numbers $s$ and $m_s$.
   
    Because the quantum mechanical spin has nothing to do with motion in space and 
    is independent of any coordinates $r$, $\theta$ or $\phi$, there is no reason to 
    exclude half-integer values of $s$ and $m_s$,
    \begin{equation}
        s=0,\frac{1}{2},1,\frac{3}{2},\dots, \quad m_s = -s, s + 1, \dots s - 1, s.
    \end{equation}
    As it would turn out, every elemental particle has a specific and immutable value of 
    $s$, and the most important one is $s=\frac{1}{2}$ (!) as it is the spin-value for 
    all leptons, including the electron, all quarks as well as protons and neutrons. Since 
    our particle of scrutiny is the electron our interest lies in spin-half systems, i.e. 
    \begin{equation}
        s = \frac{1}{2}, \quad m_s = \pm \frac{1}{2}.
    \end{equation}
    In such a system there are only two spin eigenstates,
    \begin{equation}
        \ket{\frac{1}{2},+\frac{1}{2}} = \ket{\uparrow} = \ket{+}, \quad
        \ket{\frac{1}{2},-\frac{1}{2}} = \ket{\downarrow} = \ket{-}.
    \end{equation}
    This means that the Hilbert space $\mathcal{H}$ for spin-half particles has 
    two dimensions, and that any state can be expressed as a two-dimensional 
    vector called a spinor,
    \begin{equation}
        \ket{\chi} = \begin{pmatrix}
            a \\ b
        \end{pmatrix}
        = a\ket{\uparrow} + b\ket{\downarrow}
        = a\begin{pmatrix}
            1 \\ 0 
        \end{pmatrix}
        + b\begin{pmatrix}
            0 \\ 1
        \end{pmatrix}.
    \end{equation}
    The probability of finding a particle represented by the state vector $\ket{\chi}$
    in the spin up or spin down state is $|a|^2$ and $|b|^2$, respectively. This requires 
    a normalisation, such that $|a|^2 + |b|^2 = 1$.

    In the basis of $\ket{\uparrow}$ and $\ket{\downarrow}$ the operators $\hat{S}^2$, 
    $\hat{S}_x$, $\hat{S}_y$ and $\hat{S}_z$ are represented by two-dimensional matrices,
    \begin{equation}
        \hat{S}^2 = \frac{3}{4}\hbar^2\begin{pmatrix}
            1 & 0 \\ 0 & 1 
        \end{pmatrix}
    \end{equation}
    \begin{equation}
        \hat{S}_x = \frac{\hbar}{2}\sigma_x, \quad 
        \hat{S}_y = \frac{\hbar}{2}\sigma_y, \quad 
        \hat{S}_z = \frac{\hbar}{2}\sigma_z,
    \end{equation}
    where the Pauli matrices are given by, 
    \begin{equation}
        \sigma_x = \begin{pmatrix} 0 &  1 \\ 1 & 0 \end{pmatrix} \quad 
        \sigma_y = \begin{pmatrix} 0 & -i \\ i & 0 \end{pmatrix} \quad 
        \sigma_z = \begin{pmatrix} 1 &  0 \\ 0 & -1 \end{pmatrix}.
    \end{equation}

    \subsection{Atomic Units}
    \label{sec:atomic_units}

        It is common practice to switch to a set of units that are easier to work with,
        in essence setting $\hbar = m_e = e = \dots = 1$. In this study we use atomic units, a
        form of such dimensionless units. To see how these units arise, consider
        the time-independent Schrödinger equation for a Hydrogen atom,
        \begin{equation}
            \left(-\frac{\hbar^2}{2m_e}\nabla^2 - \frac{e^2}{4\pi\epsilon_0 r} \right)
            \phi = E\phi,
        \end{equation}
        where $\hbar$ is the reduced Planck constant, equal to Planck's constant divided by 
        $2\pi$; $m_e$ is the mass of the electron, $-e$ is the charge of the electron, 
        $\epsilon_0$ is the permitivity of free space and $\nabla$ is the many-dimensional 
        differential operator. 
        We make this equation dimensionless by 
        letting $r \to \lambda r'$,
        \begin{equation}
            \left(-\frac{\hbar^2}{2m_e\lambda^2}\nabla'^2 - \frac{e^2}{4\pi\epsilon_0\lambda r'} \right)
            \phi' = E\phi'.
        \end{equation}
        We can factor out the constants in front of the operators, if we choose $\lambda$ so that,
        \begin{equation}
            \frac{\hbar^2}{m_e\lambda^2} = \frac{e^2}{4\pi \epsilon_0 \lambda} = E_a
            \to \lambda \frac{4\pi\epsilon_0\hbar^2}{m_e e^2} = a_0
        \end{equation}
        where $E_a$ is the atomic unit of energy that chemists call Hartree. Incidentally,
        we see that $\lambda$ is just the Bohr radius, $a_0$. If we let $E' = E/E_a$, we 
        obtain the dimensionless Schrödinger equation,
        \begin{equation}
            \left(-\frac{1}{2}\nabla'^2 - \frac{1}{r'} \right) \phi' = E'\phi'.
        \end{equation}
        Some conversion factors between atomic units and SI units can be found in
        \autoref{tab:atomic_units_conversion}.

        \begin{table}
            \centering
            \caption{Conversion of atomic units to SI units.}
            \begin{tabular}{ccc} \hline
                Physical quantity & Conversion factor & Value \\ \hline
                Length  & $a_0$ & $5.2918 \times 10^{-11} m$ \\
                Mass    & $m_e$ & $9.1095 \times 10^{-31} kg$ \\
                Time    & $\hbar/E_a$ & $2.4189 \times 10^{-17} s$ \\
                Charge  & $e$   & $1.6022 \times 10^{-19} C$ \\
                Energy  & $E_a$ & $4.3598 \times 10^{-18} J$ \\
                Velocity& $a_0E_a/\hbar$ & $2.1877 \times 10^{6} ms^{-1}$ \\
                Angular momentum & $\hbar$ &  $1.0546 \times 10^{-34} Js$ \\
                Electric dipole moment & $ea_0$& $8.4784 \times 10^{-30} Cm$ \\
                Electric polarizability & $e^2a_0^2/E_a$ & $1.6488 \times 10^{-41} C^2m^2J^{-1}$ \\
                Electric field & $E_a/(ea_0)$ & $5.1423 \times 10^{11} Vm^{-1} $ \\
                Wave function & $a_0^{-3/2}$ & $2.5978 \times 10^{15} m^{-3/2}$ \\ \hline
            \end{tabular}
            \label{tab:atomic_units_conversion}
        \end{table}


\section{Indistinguishable Particles}

    In classical mechanics, although particles are indistinguishable, one typically
    regards particles as individuals because a permutation of particles is counted as
    a new arrangement and something different than the initial configuration. 
    This was called ``Transcedental
    Individuality'' by Heinz Post\cite{post1963individuality}. In quantum mechanics, on 
    the other hand, a permutation is not regarded as giving rise to a new 
    arrangement. It follows that quantum objects are very different from anything else we
    know from everyday life, and must be considered ``non-individual''. This 
    idea has its origin from the uncertainty principle, stating that no sharply defined 
    particle exist. If we take this idea
    to its extreme one may postulate that all particles of a given type are one and the
    same. Here from Richard Feynman's Nobel lecture\cite{feynman1965nobel}:
    \say{I received a telephone call one day at the graduate college at Princeton from 
    Professor Wheeler, in which he said, \say{Feynman, I know why all electrons have
    the same charge and the same mass} \say{Why?} \say{Because, they are all the same 
    electron!} }

    Following the brief discussion above one may begin to postulate that, the probability density
    for the location of particles in a system must be permutation invariant,
    \begin{equation}
        \label{eq:square_wavefunction_pauli}
        \abs{\Psi(x_1, x_2, \dots, x_i, x_j, \dots, x_N)}^2 
        = 
        \abs{\Psi(x_1, x_2, \dots, x_j, x_i, \dots, x_N)}^2,
    \end{equation}
    where $\Psi$ represents a wavefunction description of $N$ particles 
    For any arbitrary permutation, this is equivalent to 
    \begin{equation}
        \Psi(x_1, \dots x_N) 
        =
        e^{i\alpha(\sigma)}\Psi(x_{\sigma(1)}, x_{\sigma(2)}, \dots, x_{\sigma(N)}),
    \end{equation}
    where $\sigma \in S_N$ is some permutation of $N$ indices and $\alpha$ is 
    some real number that may be dependent on $\sigma$.
    The same relation can be written by way of a linear permutation operator,
    \begin{equation}
        (\hat{P_\sigma}\Psi)(x_1, \dots, x_N)
        =
        \Psi(x_{\sigma(1)}, x_{\sigma(2)}, \dots, x_{\sigma(N)}).
    \end{equation}
    The `indistinguishability postulate'' states that if a permutation $P$ is applied 
    to a state representing an assembly of particles, there is no way of distinguishing
    between the permuted state and the original, by means of an observation at any time.

    One can show that the resulting wavefunction that has undergone a permutation operation 
    falls into two categories,
    \begin{equation}
        \hat{P}_\sigma\Psi = 
        \begin{cases}
            \Psi &\\
            (-1)^{\abs{\sigma}}\Psi& 
        \end{cases} \forall \sigma \in S_N,
    \end{equation}
    where $\abs{\sigma}$ is the number of transpositions in $\sigma$ and the sign 
    will be $(-1)^{\abs{\sigma}} = \pm 1$. In the former case, when the sign is $+$,
    the wavefunction is ``totally symmetric with respect to permutations''; while in 
    the latter case, when the sign is $-$, the wavefunction is ``totally anti-symmetric.''
    We show this result with a simple permutation operator $\hat{P}_{ij}$ that exchanges 
    coordinates of particle $i$ and $j$, i.e.
    \begin{equation}
        \hat{P}_{ij}\Psi(x_1, x_2, \dots, x_i, x_j, \dots, x_N)
        =  \Psi(x_1, x_2, \dots, x_j, x_i, \dots, x_N).
    \end{equation}
    Applying this permutation operator twice will return us to the initial wavefunction,
    \begin{equation}
        \hat{P}_{ij} \hat{P}_{ij} = 1,
    \end{equation}
    which implies that the permutation operator is Hermitian and unitary. Morover, the 
    Permutation operator must commute with any operator $\hat{O}$,
    \begin{equation}
        [\hat{P}_{ij}, \hat{O}] = 0.
    \end{equation}
    Consider an eigenvalue equation for the permutation operator $\hat{P}_{ij}$,
    \begin{equation}
        \hat{P}_{ij}\Psi = \lambda_{ij}\Psi,
    \end{equation}
    from which it follows that 
    \begin{equation}
        \Psi = \hat{P}_{ij}^2 \Psi = \lambda_{ij}^2\Psi\ \to \  \lambda_{ij}^2 = 1.
    \end{equation}

    This leads us to another postulate in quantum theory, that we have only two types of 
    basic particles. \emph{Bosons} have totally symmetric wavefunctions only, while 
    \emph{fermions} have totally anti-symmetric wavefunctions only. ``The physical 
    consequences of this postulate seems to be in good agreement with experimental data''
    \cite{leinaas1977theory}. Moreover, all particles with integer spin are bosons, 
    and all particles with half-integer spin are fermions
    \cite{fierz1939relativistische,pauli1940connection}. This can be proved in relativistic
    quantum mechanics, but must be accepted as an axiom in nonrelativistic 
    theory \cite{hilborn1995atoms}. Generally, the degeneracy of a state for a given 
    energy $\epsilon$ is divided into three categories,
    \begin{equation}
        n(\epsilon) = \begin{cases}
            e^{-(\epsilon - \mu)/k_BT} &\quad \text{Maxwell-Boltzmann} \\
            \frac{1}{e^{-(\epsilon - \mu)/k_BT} + 1} &\quad \text{Fermi-Dirac} \\
            \frac{1}{e^{-(\epsilon - \mu)/k_BT} - 1} &\quad \text{Bose-Einstein}.
        \end{cases}
    \end{equation}
    The Maxwell-Boltzmann distribution is the classical result, for \emph{distinguishable}
    particles; the Fermi-Dirac distribution applies to \emph{identical fermions}, and 
    the Bose-Einstein distribution is for \emph{identical bosons}. Here, $T$ is the 
    temperature, $k_B$ is Boltzmann's constant and $\mu$ is the chemical potential.

    To this day, particles with no other spin has been found, but the norwegian physicists
    Jon Magne Leinaas and Jan Myrheim discovered that in one- and two dimensions, more 
    general permutation symmetries are possible. They dubbed this third class of
    fundamental particles "anyons"\cite{leinaas1977theory}.

\section{Density Operators}

    Density operators will become very useful, especially later when we use them to compute 
    expectation values. A formal introduction to this concept is therefor warranted.
    Consider a system that is described not by 
    a single state vector, but by an ensemble of state vectors
    $\{\ket{\psi}_1, \ket{\psi}_2, \dots ,\ket{\psi}_n\}$ with a probability distribution 
    $\{p_1,p_2,\dots p_m \}$ defined over the ensemble. We may consider this ensemble to contain
    \emph{quantum probabilites} carried by the state vectors $\{\ket{\psi}_k \}$ and classical 
    propabilities carried by the distribution $\{p_k\}$. A system described by an ensemble 
    state is said to be in a \emph{mixed} state.

    The expectation value of a quantum observable in a state described by an ensemble of state
    vectors is 
    \begin{equation}
        \ev{\hat{O}} = \sum_{k=1}^n p_k \ev{\hat{O}}_k = \sum_{k=1}^n p_k \mel{\psi_k}{\hat{O}}{\psi_k}.
    \end{equation}
    This expression motivates the introduction of the \emph{density operator} associated with 
    the mixed state,
    \begin{equation}
        \hat{\rho} = \sum_{k=1}^n p_k \dyad{\psi_k}.
    \end{equation}
    The corresponding matrix, defined by reference to an orthogonal basis $\{\ket{\phi_i}\}$,
    is called the \emph{density matrix},
    \begin{equation}
        \rho_{ij} = \sum_{i=1}^n p_k \bra{\phi_i}\dyad{\psi_k}\ket{\phi_j}.
    \end{equation}
    An important note is that all measurable information about the system is contained in its 
    density operator. We can for instance compute expectation values using the density operator,
    \begin{equation}
        \begin{aligned}
        \ev{\hat{O}}
            =& \sum_{k=1}^n p_k \sum_i \mel{\psi_k}{\hat{O}}{\phi_i} \braket{\phi_i}{\psi_k} \\
            =& \sum_i \sum_{k=1}^n p_k \braket{\phi_i}{\psi_k} \mel{\psi_k}{\hat{O}}{\phi_i}
            = \tr{\hat{\rho}\hat{O}}.
        \end{aligned}
    \end{equation}
    
    There are certain general properties that any density operator has to satisfy,
    \begin{equation}
        \begin{aligned}
        p_k = p_k^*\ \to\ \hat{\rho}=\hat{\rho}^\dagger &\quad
            \text{Hermiticity} \\
        p_k \geq 0\ \to\ \mel{\chi}{\hat{\rho}}{\chi} \forall \chi &\quad 
            \text{Positive semi-definite} \\
        \sum_k p_k = 1\ \to \ \tr{\hat{\rho}} = 1 &\quad 
            \text{Normalisation}.
        \end{aligned}
    \end{equation}     

    We also note that
    \begin{equation}
        \tr \hat{\rho}^2 = \sum p_k^2 \ \to \ \tr \hat{\rho}^2 \leq 1,
    \end{equation}
    because all eigenvalues are $p_k \leq 1$, which means that $\tr \hat{\rho}^2 \leq \tr \hat{\rho}$.
    The pure state is a special case where one of the probabilities $p_k$ is equal to one, and the others 
    are $0$. In this case, the density operator will be equivalent to a projection operator onto this single 
    state. Moreover, $\tr\hat{\rho}^2 = 1$ for a pure state, while for a mixed state we have 
    $\tr\hat{\rho}^2 <1$.