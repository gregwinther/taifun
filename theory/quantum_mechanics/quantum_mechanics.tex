\chapter{Quantum Mechanics}
    \epigraph{Hierzu ist es notwendig, die Energy nicht als eine stetige
    unbeschränkt teilbare, sondern als eine discrete, ause einer ganzen
    Zahl von endlichen gleichen Teilen zusammengesetzte Grösse 
    aufzufassen.}
    {--- Max Planck}


\section{Classical Mechanics}

    The formalism used in quantum mechanics largely stems from William Rowan Hamilton's 
    formulation of classical mechanics. Through the process of canonical
    quantisation any classical model of a physical system is turned into
    a quantum mechanical model.

    In Hamilton's formulation of classical mechanics, a complete description of a system
    of N particles is described by a set of canonical coordinates 
    $q = (\vec{q}_1, \dots, \vec{p}_N)$ and corresponding conjugate momenta
    $p = (\vec{p}_1, \dots \vec{p}_N)$. Together, each pair of coordinate and momentum
    form a point $\xi = (q, p)$ in phase space, which is a space of all possible states
    of the system. Moreover, pairs of generalised coordinates and conjugate
    momenta are canonical if they satisfy the Poisson brackets so that 
    $\{q_i, p_k\} = \delta_{ij}$. The Poisson bracket of two functions is defined as 

    \begin{equation}
        \{f, g\} = \frac{\partial f}{\partial q} \frac{\partial g}{\partial g}
        - \frac{\partial f}{\partial p} \frac{\partial g}{\partial q}.
    \end{equation}

    The governing equations of motion in a classical system is Hamilton's equations,

    \begin{align}
        \dot{q} &= \frac{\partial}{\partial p} \mathscr{H}(q, p) \\
        \dot{p} &= -\frac{\partial}{\partial q} \mathscr{H}(q, p)
    \end{align}

    where $\mathscr{H}(q, p)$ is the Hamiltonian, a function for the total energy of the
    system. Hamilton's equations may also be stated in terms of the Poisson brackets,

    \begin{equation}
        \frac{dp_i}{dt} = \{p_i, \mathscr{H}\}, \ \frac{dq_i}{dt} = \{q_i, \mathscr{H}\}.
    \end{equation}

    A system consisting of $N$ of equal mass $m$, subject forces caused by an external
    potential, as well as acting on each other with forces stemming from a central
    potetntial $w(q_ij)$ has the following Hamiltonian,

    \begin{equation}
        \mathscr{H}(q, p) = \mathscr{T}(q) + \mathscr{V}(p) + \mathscr{W}(p) 
            = \frac{1}{2m}\sum_{i} \abs{\vec{p}_i}^2 + \sum_{i} v(\vec{r}_i)
                + \frac{1}{2}\sum_{i<j} w(\vec{r_{ij}}).
    \end{equation}

    This Hamiltonian conveniently contains several parts - the kinetic energy, the
    external potential energy and the interaction energy; denoted by $\mathscr{T}$,
    $\mathscr{V}$ and $\mathscr{W}$ respectively.

\section{Canonical Quantisation}

    In order to transition from a classical system to a quantum system, we move from 
    the classucal phase space to the Hilbert space, through the procedure known as 
    canonical, or first\footnote{Second quantisation comes later.}-, quantisation.
    Whilst the state of a classical system is a point in phase space, a quantum state
    is a complex-valued state vector in discrete, infinite dimensional, Hilbert space,
    that is a complete vector space equipped with an inner product. This space
    is most commonly chosen to be the space of square-integrable functions $\Psi$,
    dependent on all coordinates
    
    \begin{equation}
        \Psi = \Psi(x_1, x_2, \dots, x_N).
    \end{equation}
    
    These functions
    are dubbed wavefunctions and are maps from a point $(x_1, \dots, x_N)$ in
    configuration space to the complex vector space,

    \begin{equation}
        \Psi: X^N \to \mathds{C}.
    \end{equation}

    It has been widely discussed how such an object can represent the state of a 
    particle. The answer is provided by Max Born's probabilistic interpretation,
    which says that $\abs{\Psi(x_1, \dots, x_N)}^2$, gives the probability of finding
    the particle at a certain position. For a situation with one particle in one
    dimension we have,

    \begin{equation}
        \int_a^b \abs{\Psi(x)}^2 dx = 
        \left\{\begin{aligned}
            \text{probability of finding the} \\
            \text{particle between $a$ and $b$}
        \end{aligned}\right\}
    \end{equation}

    while $\abs{\Psi(x_1, x_2, \dots, x_N)}^2$ is the probability density for locating
    all particles at the point $(x_1, \dots x_N) \in X^N$. Since the total probability
    must be 1, we are provided with a normalisation condition for the wavefunction,
    
    \begin{equation}
        \int_{X^N} \abs{\Psi(x_1, x_2, \dots, x_N)}^2 dx_1 dx_2\dots dx_N = \mathds{1}.
    \end{equation}
    
\subsection{The Dirac-von Neumann Postulates}

    The following postulates, or axioms, provide a precise and concise description 
    of quantum mechanics in terms of operators on the Hilbert space. There are
    many variations of these postulates, introduced both by their namesakes 
    Paul Adriene Maurice Dirac\cite{dirac1930principles} and John von 
    Neumann\cite{vonNeumann1932foundations}.

    \paragraph{Hilbert Space}
    A quantum state of an isolated physical system is described by a vector
    with unit norm in a Hilbert space, a complex vector space quipped with a
    scalar product. 

    \paragraph{Observables}
    Each physical observable of a system is accociated with a \emph{hermitian}
    operator acting on the Hilbert space. The eigenstates of each such
    operator define a \emph{complete}, \emph{orthonormal} set of vectors.

    With $\hat{O}$ an operator, hermiticity means,
    \begin{equation}
        \bra{\phi}\hat{O}\psi\rangle = \langle\hat{O}\phi\ket{\psi} \equiv \bra{\phi} \hat{O} \ket{\psi}.
    \end{equation}

    Completeness means,
    \begin{equation}
        \sum_i \ket{i}\bra{i} = \mathds{1}.
    \end{equation}

    Orthonormal means,
    \begin{equation}
        \braket{i}{j} = \delta_{ij}.
    \end{equation}

    \paragraph{Time Evolution}
    The time evolution of the state vector, $\ket{\psi} = \ket{\psi (t)}$, is given by the Schrödinger
    equation\footnote{In the Schrödinger picture.}.
    \begin{equation}
        i\hbar \frac{d}{dt} \ket{\psi (t)} = \hat{H} \ket{\psi (t)}.
    \end{equation}

    \paragraph{Measurements}
    Physically measurable values, associated with an obeservable $\hat{O}$ are defined by the 
    eigenvalues $o_n$ of the observable,
    \begin{equation}
        \hat{O}\ket{n} = o_n\ket{n}.
    \end{equation}
    The probability for finding a particular eigenvalue in the measurement is
    \begin{equation}
        p_n = \abs{\braket{n}{\psi}}^2,
    \end{equation}
    with the system in state $\ket{\psi}$ before the measurement, and $\ket{n}$ as the 
    eigenstate corresponding to the eigenvalue $o_n$.

\section{The Many-Body Quantum Hamiltonian}

    The full Hamiltionian for a quantum many-body system can be a large 
    and unwieldy thing. In this study we will constrain ourselves to the study of
    electronic systems. Electronic systems are 

    Free particle in one dimension,
    \begin{equation}
        \hat{H} = -\frac{\hbar^2}{2m} \frac{\partial^2}{\partial x^2}
    \end{equation}

    Introduce potential.

    Many particles in many dimenions.

    Interaction.

    Nuclear interaction (molecule).

    There is other stuff as well.

    In atomic units, the Hamiltonian for $N$ electrons and $M$ nuclei is
    \footnote{Often referred to as the electronic Hamiltonian},

    \begin{equation}
        \label{eq:electronic_hamiltonian}
        \hat{H} = 
            - \sum_{i=1}^N \frac{1}{2} \nabla _i^2 
            - \sum_{A=1}^M \frac{1}{2M_A} \nabla_A^2
            - \sum_{i=1}^N\sum_{A=1}^M \frac{Z_A}{r_iA}
            + \sum_{i=1}^N\sum_{j>i}^N \frac{1}{r_{ij}}
            + \sum_{A=1}^M\sum_{B>A}^M \frac{Z_A Z_B}{R_{AB}},
    \end{equation}

    where $M_A$ si the ratio of the mass of nucleus $A$ to the mass of an electron,
    and $Z_A$ is the atomic number of nuclus $A$. The first term in 
    \autoref{eq:electronic_hamiltonian} is the operator for the kinetic energy of 
    the electrons, the second term is the operator for the kinetic energi of the 
    nuclei, the third term represents the Coulomb attraction between the electrons
    and the nuclei, the fourth term is the Coulomb repulsion between electrons 
    and the fifth and last term is the Coulomb repulsion between nuclei.

    PERHAPS FIGURE OF GENERAL SYSTEM HERE?

    \subsection{Atomic Units}

        In the Hamiltionian above (\autoref{eq:electronic_hamiltonian}), we have grown up
        and set $\hbar = m_e = e = \dots = 1$. This is a result of using atomic units, a
        form of commonly used dimensionless units. To see how these units arise, consider
        the time-independent Schrödinger equation for a Hydrogen atom,

        \begin{equation}
            \left(-\frac{\hbar^2}{2m_e}\nabla^2 - \frac{e^2}{4\pi\epsilon_0 r} \right)
            \phi = E\phi,
        \end{equation}

        where $\hbar$ is the reduced Planck constant, equal to Planck's constant divided by 
        $2\pi$; $m_e$ is the mass of the electron, $-e$ is the charge of the electron and 
        $\epsilon_0$ is the permitivity of free space. We make this equation dimensionless by 
        letting $r \to \lambda r'$,

        \begin{equation}
            \left(-\frac{\hbar^2}{2m_e\lambda^2}\nabla'^2 - \frac{e^2}{4\pi\epsilon_0\lambda r'} \right)
            \phi' = E\phi'.
        \end{equation}

        We can factor out the constants in front of the operators, if we choose $\lambda$ so that,

        \begin{equation}
            \frac{\hbar^2}{m_e\lambda^2} = \frac{e^2}{4\pi \epsilon_0 \lambda} = E_a
            \to \lambda \frac{4\pi\epsilon_0\hbar^2}{m_e e^2} = a_0
        \end{equation}

        where $E_a$ is the atomic unit of energy that chemists call Hartree. Incidently,
        we see that $\lambda$ is just the Bohr radius, $a_0$. If we let $E' = E/E_a$, we 
        obtain the dimensionless Schrödinger equation,

        \begin{equation}
            \left(-\frac{1}{2}\nabla'^2 - \frac{1}{r'} \right) \phi' = E'\phi'.
        \end{equation}

        Some conversion factors between atomic units and SI units can be found in
        \autoref{tab:atomic_units_conversion}.

        \begin{table}
            \centering
            \caption{Conversion of atomic units to SI units THIS IS FROM SZABO AND OSTLUND.}
            \begin{tabular}{ccc} \hline
                Physical quantity & Conversion factor & Value \\ \hline
                Length  & $a_0$ & $5.2918 \times 10^{-11} m$ \\
                Mass    & $m_e$ & $9.1095 \times 10^{-31} kg$ \\
                Charge  & $e$   & $1.6022 \times 10^{-19} C$ \\
                Energy  & $E_a$ & $4.3598 \times 10^{-18} J$ \\
                Angular momentum & $\hbar$ &  $1.0546 \times 10^{-34} Js$ \\
                Electric dipole moment & $ea_0$& $8.4784 \times 10^{-30} Cm$ \\
                Electric polarizability & $e^2a_0^2/E_a$ & $1.6488 \times 10^{-41} C^2m^2J^{-1}$ \\
                Electric field & $E_a/(ea_0)$ & $5.1423 \times 10^{11} Vm^{-1} $ \\
                Wave function & $a_0^{-3/2}$ & $2.5978 \times 10^{15} m^{-3/2}$ \\ \hline
            \end{tabular}
            \label{tab:atomic_units_conversion}
        \end{table}

    \subsection{The Born-Oppenheimer Approximation}



\section{Indistinguishable Particles}

    In a quantum system the particles are identical and impossible to tell
    apart, as indicated by several studies (KILDE!!). 
    Feynman: In fact, all electrons are one and the same.
    The probability density
    for the location of particles in a system must therefore be permutation
    invariant,

    \begin{equation}
        \abs{\Psi(x_1, x_2, \dots, x_i, x_j x_N)}^2 
        = 
        \abs{\Psi(x_1, x_2, \dots, x_j, x_i x_N)}^2.
    \end{equation}

    For any arbitrary permutation, this is equivalent to 
    
    \begin{equation}
        \Psi(x_1, \dots x_N) 
        =
        e^{i\alpha(\sigma)}\Psi(x_{\sigma(1)}, x_{\sigma(2)}, \dots, x_{\sigma(N)}),
    \end{equation}

    where $\sigma \in S_N$ is some permutation of $N$ indices and $\alpha$ is 
    some real number that may be dependent on $\sigma$.

    The same relation can be written by way of a linear permutation operator,

    \begin{equation}
        (\hat{P_\sigma}\Psi)(x_1, \dots, x_N)
        =
        \Psi(x_{\sigma(1)}, x_{\sigma(2)}, \dots, x_{\sigma(N)}).
    \end{equation}
    
    \subsection{Slater Determinants}