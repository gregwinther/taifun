\chapter{Quantum Mechanics}
    \epigraph{Hierzu ist es notwendig, die Energy nicht als eine stetige
    unbeschränkt teilbare, sondern als eine discrete, ause einer ganzen
    Zahl von endlichen gleichen Teilen zusammengesetzte Grösse 
    aufzufassen.}
    {--- Max Planck}


\section{Classical Mechanics}

    The formalism used in quantum mechanics largely stems from William Rowan Hamilton's 
    formulation of classical mechanics. Through the process of canonical
    quantisation any classical model of a physical system is turned into
    a quantum mechanical model.

    In Hamilton's formulation of classical mechanics, a complete description of a system
    of N particles is described by a set of canonical coordinates 
    $q = (\vec{q}_1, \dots, \vec{p}_N)$ and corresponding conjugate momenta
    $p = (\vec{p}_1, \dots \vec{p}_N)$. Together, each pair of coordinate and momentum
    form a point $\xi = (q, p)$ in phase space, which is a space of all possible states
    of the system. Moreover, pairs of generalised coordinates and conjugate
    momenta are canonical if they satisfy the Poisson brackets so that 
    $\{q_i, p_k\} = \delta_{ij}$. The Poisson bracket of two functions is defined as 

    \begin{equation}
        \{f, g\} = \frac{\partial f}{\partial q} \frac{\partial g}{\partial g}
        - \frac{\partial f}{\partial p} \frac{\partial g}{\partial q}.
    \end{equation}

    The governing equations of motion in a classical system is Hamilton's equations,

    \begin{align}
        \dot{q} &= \frac{\partial}{\partial p} \mathscr{H}(q, p) \\
        \dot{p} &= -\frac{\partial}{\partial q} \mathscr{H}(q, p)
    \end{align}

    where $\mathscr{H}(q, p)$ is the Hamiltonian, a function for the total energy of the
    system. Hamilton's equations may also be stated in terms of the Poisson brackets,

    \begin{equation}
        \frac{dp_i}{dt} = \{p_i, \mathscr{H}\}, \ \frac{dq_i}{dt} = \{q_i, \mathscr{H}\}.
    \end{equation}

    A system consisting of $N$ of equal mass $m$, subject forces caused by an external
    potential, as well as acting on each other with forces stemming from a central
    potetntial $w(q_ij)$ has the following Hamiltonian,

    \begin{equation}
        \mathscr{H}(q, p) = \mathscr{T}(q) + \mathscr{V}(p) + \mathscr{W}(p) 
            = \frac{1}{2m}\sum_{i} \abs{\vec{p}_i}^2 + \sum_{i} v(\vec{r}_i)
                + \frac{1}{2}\sum_{i<j} w(\vec{r_{ij}}).
    \end{equation}

    This Hamiltonian conveniently contains several parts - the kinetic energy, the
    external potential energy and the interaction energy; denoted by $\mathscr{T}$,
    $\mathscr{V}$ and $\mathscr{W}$ respectively.

\section{Canonical Quantisation}

    In order to transition from a classical system to a quantum system, we move from 
    the classucal phase space to the Hilbert space, through the procedure known as 
    canonical, or first\footnote{Second quantisation comes later.}-, quantisation.
    Whilst the state of a classical system is a point in phase space, a quantum state
    is a complex-valued state vector in discrete, infinite dimensional, Hilbert space,
    that is a complete vector space equipped with an inner product. This space
    is most commonly chosen to be the space of square-integrable functions $\Psi$,
    dependent on all coordinates
    
    \begin{equation}
        \Psi = \Psi(x_1, x_2, \dots, x_N).
    \end{equation}
    
    These functions
    are dubbed wavefunctions and are maps from a point $(x_1, \dots, x_N)$ in
    configuration space to the complex vector space,

    \begin{equation}
        \Psi: X^N \to \mathds{C}.
    \end{equation}

    It has been widely discussed how such an object can represent the state of a 
    particle. The answer is provided by Max Born's probabilistic interpretation,
    which says that $\abs{\Psi(x_1, \dots, x_N)}^2$, gives the probability of finding
    the particle at a certain position. For a situation with one particle in one
    dimension we have,

    \begin{equation}
        \int_a^b \abs{\Psi(x)}^2 dx = 
        \left\{\begin{aligned}
            \text{probability of finding the} \\
            \text{particle between $a$ and $b$}
        \end{aligned}\right\}
    \end{equation}

    while $\abs{\Psi(x_1, x_2, \dots, x_N)}^2$ is the probability density for locating
    all particles at the point $(x_1, \dots x_N) \in X^N$. Since the total probability
    must be 1, we are provided with a normalisation condition for the wavefunction,
    
    \begin{equation}
        \int_{X^N} \abs{\Psi(x_1, x_2, \dots, x_N)}^2 dx_1 dx_2\dots dx_N = \mathds{1}.
    \end{equation}
    
\subsection{The Dirac-von Neumann Postulates}

    The following postulates, or axioms, provide a precise and concise description 
    of quantum mechanics in terms of operators on the Hilbert space. There are
    many variations of these postulates, introduced both by their namesakes 
    Paul Adriene Maurice Dirac\cite{dirac1930principles} and John von 
    Neumann\cite{vonNeumann1932foundations}.

    \paragraph{Hilbert Space}
    A quantum state of an isolated physical system is described by a vector
    with unit norm in a Hilbert space, a complex vector space quipped with a
    scalar product. 

    \paragraph{Observables}
    Each physical observable of a system is accociated with a \emph{hermitian}
    operator acting on the Hilbert space. The eigenstates of each such
    operator define a \emph{complete}, \emph{orthonormal} set of vectors.

    With $\hat{O}$ an operator, hermiticity means,
    \begin{equation}
        \bra{\phi}\hat{O}\psi\rangle = \langle\hat{O}\phi\ket{\psi} \equiv \bra{\phi} \hat{O} \ket{\psi}.
    \end{equation}

    Completeness means,
    \begin{equation}
        \sum_i \ket{i}\bra{i} = \mathds{1}.
    \end{equation}

    Orthonormal means,
    \begin{equation}
        \braket{i}{j} = \delta_{ij}.
    \end{equation}

    \paragraph{Time Evolution}
    The time evolution of the state vector, $\ket{\psi} = \ket{\psi (t)}$, is given by the Schrödinger
    equation\footnote{In the Schrödinger picture.}.
    \begin{equation}
        i\hbar \frac{d}{dt} \ket{\psi (t)} = \hat{H} \ket{\psi (t)}.
    \end{equation}

    \paragraph{Measurements}
    Physically measurable values, associated with an obeservable $\hat{O}$ are defined by the 
    eigenvalues $o_n$ of the observable,
    \begin{equation}
        \hat{O}\ket{n} = o_n\ket{n}.
    \end{equation}
    The probability for finding a particular eigenvalue in the measurement is
    \begin{equation}
        p_n = \abs{\braket{n}{\psi}}^2,
    \end{equation}
    with the system in state $\ket{\psi}$ before the measurement, and $\ket{n}$ as the 
    eigenstate corresponding to the eigenvalue $o_n$.

\section{The Many-Body Quantum Hamiltonian}

    The full Hamiltionian for a quantum many-body system can be a large 
    and unwieldy thing. In this study we will constrain ourselves to the study of
    electronic systems. Purely on a phenomenological basis, one would include nuclear
    terms in the Hamiltonian as well. In this study however, we will stay within the
    Born-Oppenheimer approximation and treat the nuclei as stationary particles, thereby
    refraining from introducing terms that involve the motion of nuclei. Here we
    introduce the molecular electronic Breit-Pauli Hamiltionian, throrougly described in 
    Helgaker et al.\cite{helgaker2012recent},
    \begin{equation}
        \label{eq:breit_pauli_hamiltonian}
        \hat{H}^{\text{BP}}_{\text{mol}}
        = \begin{cases}
            \hat{H}_{\text{kin}} \quad &\leftarrow \text{kinetic energy} \\
            + \hat{H}_{\text{cou}} \quad &\leftarrow \text{Coulomb interactions} \\
            + \hat{H}_{\text{ee}} \quad &\leftarrow \text{external electric field interaction} \\
            + \hat{H}_Z \quad &\leftarrow \text{Zeeman intercations} \\
            + \hat{H}_{\text{so}} &\leftarrow \text{spin-orbit interactions} \\
            + \hat{H}_{\text{ss}} &\leftarrow \text{spin-spin interactions} \\
            + \hat{H}_{\text{oo}} &\leftarrow \text{spin-spin interactions} \\
            + \hat{H}_{\text{dia}} &\leftarrow \alpha^4 \text{diamagnetic interactions}
        \end{cases}
    \end{equation}
    
    \paragraph{Kinetic energy}
    The Breit-Pauli kinetic energy term in \autoref{eq:breit_pauli_hamiltonian} is
    \begin{equation}
        \hat{H}_{\text{kin}}
            = - \frac{1}{2}\sum_i \nabla^2_i - \frac{\alpha^2}{8}\sum_{i} \nabla^4_i,
    \end{equation}
    where the first term is the common classical kinetic energy operator and the second
    term is the relativistic mass-velocity term. This term arises because of the
    dependence of mass on velocity. This is one of the larger relativistic corrections
    for slow electrons. The mass-velocity term is unbounded from below and should not
    be included in variational calculations\cite{kutzelnigg1990perturbation}.


    \paragraph{Coulomb interactions}

    Coulomb interaction terms in the Breit-Pauli Hamiltonian (\autoref{eq:breit_pauli_hamiltonian})
    are the following,
    \begin{equation}
        \begin{aligned}
            \hat{H}_{\text{cou}} 
                =& -\sum_{iK}\frac{Z_K}{r_{iK}} 
                    + \frac{1}{2}\sum_{i \neq j} \frac{1}{r_{ij}} 
                    + \frac{1}{2}\sum_{K \neq L} \frac{Z_k Z_L}{R_{KL}} \\
                &+ \frac{\alpha^2\pi}{2} \sum_{iK} Z_{K}\delta (\vb{r}_{iK})
                - \frac{\alpha^2\pi}{2} \sum_{i \neq j} \delta (\vb{r}_{ij}) \\
                &+ \frac{2\pi}{3} \sum_{iK}Z_K R^2_K \delta (\vb{r}_{iK})
                    - \frac{1}{3} \sum_{iK}
                        \frac{\tr \Theta_K (3\vb{r}_{iK} \vb{r}^T_{iK} - r^2_{iK}I_3)}
                        {r^5_{ik}}.
        \end{aligned}
    \end{equation}
    The first three terms are the nonrelativistic Coulomb point-charge interactions between
    nucles and electron, electron and electron and nucleus and nucleus, respectively.
    The fourth and fifth terms are the Darwin corrections caused by the Zitterbewegung
    of the electrons. Because the nuclear point-charge model is not always adequate, the
    second-to-last and last term correct errors from this approximation. Here, $R_K$ is 
    the nuclear extent and $\Theta_K$ is the nuclear quadropole moment. These terms are
    important in nuclear resonance studies\cite{kutzelnigg1990perturbation} and nucllear quadropole resonance
    studies\cite{abragam1961principles}.

    \paragraph{External electric field interactions}
    The Breit-Pauli Hamiltonian includes terms that model the effects of an externally
    applied scalar potential $\phi(\vb{r})$,
    \begin{equation}
        \label{eq:ef_hamiltonian}
        \hat{H}_{\text{ef}}
            = - \sum_i \phi_i
              + \sum_K Z_K\phi_K
              + \frac{\alpha^2}{\phi_K} \sum_i (\nabla_i \cdot \vb{E}_i).
    \end{equation}
    It is often safe to assume that the applied field are quite uniform on the
    molecular scale, and one therefore often expands \autoref{eq:ef_hamiltonian}
    in multipoles,
    \begin{equation}
        \hat{H}_{\text{ef}}
            = Q_{\text{tot}}\phi_0 
            - \vb{\mu}_{\text{tot}} \cdot \vb{E}_0
            - \frac{1}{2}\tr \vb{Q}_{\text{tot}}\vb{V}_0
            + \dots,
    \end{equation}
    where $q_{\text{tot}}$ is the total charge of the molecule, $\vb{\mu}_0$ is the 
    dipole moment, $\vb{Q}_{\text{tot}}$ is the second moment, and $\vb{V}_0$ is the
    electric field gradient. Higher-order terms are only necessary for fields that vary
    greatly in time.

    \paragraph{Zeeman interactions}
    Paramagnetic interactions of the molecule with an externally applied magnetic field
    $\vb{B}$ are described by the Zeeman term in the Breit-Pauli Hamiltonian
    (\autoref{eq:breit_pauli_hamiltonian}),
    \begin{equation}
        \label{eq:zeeman_hamiltonian}
        \hat{H}_Z
            = - \vb{B} \cdot \sum_i \left(
                - \frac{1}{2}\vb{l}_{iO} - \vb{s}_i
                + \frac{1}{2}\alpha^2\vb{s}_i\nabla_i^2
            \right)
            - \vb{B} \cdot \sum_K \vb{M}_K.
    \end{equation} 
    The nuclear part, here represented by the last term in \autoref{eq:zeeman_hamiltonian},
    are on the order of $10^{-3}$ in atomic units. This is much smaller than the electronic
    part, given by the first sum in \autoref{eq:zeeman_hamiltonian}, but the nuclear part is 
    very important in nuclear magnetic resonance (NMR) computations, where it determines the
    unshielded resonance lines in the spectra. 

    The first term inside the parenthesis in \autoref{eq:zeeman_hamiltonian} corresponds to
    Zeeman interaction with the magnetic moment generated by the orbital angular momentum of
    the electrons, $\vb{l}_{iO} = \vb{r}_{iO} \times \vb{p}_i$. The second and third terms 
    in the paranthesis are electronic contributions to Zeeman effect from the spin of the 
    electrons. The relativistic correction constituted in the third term is important in
    electron paramegnetic resonance (EPR) spectroscopy.

    \paragraph{Spin-orbit interactions}
    Up to second order in the fine-structure constant, the terms that couple motion of eletrons 
    to particle spins in \autoref{eq:breit_pauli_hamiltonian} are
    \begin{equation}
        \begin{aligned}
            \label{eq:spin_orbit_hamiltonian}
            \hat{H}_{\text{so}}
                =& \frac{\alpha^2}{2} \sum_{iK} \frac{Z_K\vb{s}_i \cdot \vb{l}_{iK}}
                        {r^3_{iK}}
                    - \frac{\alpha^2}{2}\sum_{i \neq j} \frac{\vb{s}_i \vb{l}_{ij}}
                        {r^3_{ij}}
                    - \alpha^2 \sum_{i \neq j} \frac{\vb{s}_j \vb{l}_{ij}}
                        {r^3_{ij}} \\
                &+ \alpha^2 \sum_{iK} \frac{\vb{M}_K \cdot \vb{l}_{iK}}{r^3_{iK}}
                    + \frac{\alpha^2}{4} \sum_i (\vb{E}_i \times \vb{p}_i - \vb{p}_i\times\vb{E}_i)
        \end{aligned}
    \end{equation}
    When electron spin coupled to magnetic filed induced by other charges i motion we have spin-orbit 
    interaction. The first term in \autoref{eq:spin_orbit_hamiltonian} models nuclear spin-orbit effect, 
    the second term models interaction between the spin of a particle with its own orbit, 
    the third is interaction with other orbits. The fourth term is known as the orbital hyperfin
    operator and couples magnetic moments to the 
    orbital motion of electrons, while the fifth and last term is modles coupling of electric fields 
    and orbits.

    \paragraph{Spin-psin interaction}
    All terms that arises in the Breit-Pauli Hamiltionian (\autoref{eq:breit_pauli_hamiltonian}) due to 
    coupling between magnetic momenta or spin of two particles are
    \begin{equation}
        \begin{aligned}
            \hat{H}_{\text{ss}}
                &= \frac{\alpha^2}{2} \sum_{i \neq j}
                \left[ 
                    \frac{r^2_{ij}\vb{s}_i \cdot \vb{s}_j - 3\vb{s}_i \cdot \vb{r}_{ij}\vb{r}_{ij} \cdot \vb{s}_j}
                            {r^5_{ij}}
                    - \frac{8\pi}{3} \delta(\vb{r}_{ij}) \vb{s}_i \cdot \vb{s}_j
                \right] \\
                &= \alpha^2 \sum_{iK}
                \left[ 
                    \frac{r^2_{iK}\vb{s}_i \cdot \vb{M}_K - 3\vb{s}_i \cdot \vb{r}_{iK}\vb{r}_{iK} \cdot \vb{M}_K}
                            {r^5_{iK}}
                    - \frac{8\pi}{3} \delta(\vb{r}_{iK}) \vb{s}_i \cdot \vb{M}_k
                \right] \\
                &= \frac{\alpha^2}{2} \sum_{K \neq L}
                \left[ 
                    \frac{r^2_{KL} (\vb{M}_K \vb{M}_L - 3(\vb{M}_K \cdot \vb{R}_{KL}) (\vb{R}_{KL} \cdot \vb{M}_L)}
                            {R^5_{KL}}
                \right]
        \end{aligned}
    \end{equation}

    \paragraph{Diamagnetic Interactions}
    The magnitude of effects from diamagnetic interaction in the Breit-Pauli Hamiltonian 
    (\autoref{eq:breit_pauli_hamiltonian}) are terms of order $\alpha^4$ or smaller.
    Most of these effects are only important in some cases where strong external magnetic
    fields are applied (NMR, EPR).

    \subsection{Atomic Units}

        In the Hamiltionian above (\autoref{eq:breit_pauli_hamiltonian}), we have grown up
        and set $\hbar = m_e = e = \dots = 1$. This is a result of using atomic units, a
        form of commonly used dimensionless units. To see how these units arise, consider
        the time-independent Schrödinger equation for a Hydrogen atom,

        \begin{equation}
            \left(-\frac{\hbar^2}{2m_e}\nabla^2 - \frac{e^2}{4\pi\epsilon_0 r} \right)
            \phi = E\phi,
        \end{equation}

        where $\hbar$ is the reduced Planck constant, equal to Planck's constant divided by 
        $2\pi$; $m_e$ is the mass of the electron, $-e$ is the charge of the electron and 
        $\epsilon_0$ is the permitivity of free space. We make this equation dimensionless by 
        letting $r \to \lambda r'$,

        \begin{equation}
            \left(-\frac{\hbar^2}{2m_e\lambda^2}\nabla'^2 - \frac{e^2}{4\pi\epsilon_0\lambda r'} \right)
            \phi' = E\phi'.
        \end{equation}

        We can factor out the constants in front of the operators, if we choose $\lambda$ so that,

        \begin{equation}
            \frac{\hbar^2}{m_e\lambda^2} = \frac{e^2}{4\pi \epsilon_0 \lambda} = E_a
            \to \lambda \frac{4\pi\epsilon_0\hbar^2}{m_e e^2} = a_0
        \end{equation}

        where $E_a$ is the atomic unit of energy that chemists call Hartree. Incidently,
        we see that $\lambda$ is just the Bohr radius, $a_0$. If we let $E' = E/E_a$, we 
        obtain the dimensionless Schrödinger equation,

        \begin{equation}
            \left(-\frac{1}{2}\nabla'^2 - \frac{1}{r'} \right) \phi' = E'\phi'.
        \end{equation}

        Some conversion factors between atomic units and SI units can be found in
        \autoref{tab:atomic_units_conversion}.

        \begin{table}
            \centering
            \caption{Conversion of atomic units to SI units THIS IS FROM SZABO AND OSTLUND.}
            \begin{tabular}{ccc} \hline
                Physical quantity & Conversion factor & Value \\ \hline
                Length  & $a_0$ & $5.2918 \times 10^{-11} m$ \\
                Mass    & $m_e$ & $9.1095 \times 10^{-31} kg$ \\
                Charge  & $e$   & $1.6022 \times 10^{-19} C$ \\
                Energy  & $E_a$ & $4.3598 \times 10^{-18} J$ \\
                Angular momentum & $\hbar$ &  $1.0546 \times 10^{-34} Js$ \\
                Electric dipole moment & $ea_0$& $8.4784 \times 10^{-30} Cm$ \\
                Electric polarizability & $e^2a_0^2/E_a$ & $1.6488 \times 10^{-41} C^2m^2J^{-1}$ \\
                Electric field & $E_a/(ea_0)$ & $5.1423 \times 10^{11} Vm^{-1} $ \\
                Wave function & $a_0^{-3/2}$ & $2.5978 \times 10^{15} m^{-3/2}$ \\ \hline
            \end{tabular}
            \label{tab:atomic_units_conversion}
        \end{table}


    \section{Indistinguishable Particles}

        In classical mechanics, although particles are indistinguishable, one typically
        regards particles as individuals because a permutation of particles is counted as
        a new arrangement and something different than the initial configuration. 
        This was called ``Transcedental
        Individuality'' by Heinz Post\cite{post1963individuality}. In quantum mechanics, on 
        the other hand, a permutation is not regarded as giving rise to a new 
        arrangement. It follows that quantum objects are very different from anything else we
        know from everyday life, and must be considered ``non-individual''. By taken this idea
        to it's extreme one may postulate that all particles of a given type are one and the
        same. Here from a telephone call betwwen John Wheeler and Richard
        Feynman\cite{feynman1965nobel},
        \begin{displayquote}
             I received a telephone call one day at the graduate college at Princeton from 
            Professor Wheeler, in which he said, ``Feynman, I know why all electrons have
            the same charge and the same mass'' ``Why?'' ``Because, they are all the same 
            electron!''           
        \end{displayquote}

        Following the brief discussion above one may conclude that, the probability density
        for the location of particles in a system must be permutation invariant,
        \begin{equation}
            \label{eq:square_wavefunction_pauli}
            \abs{\Psi(x_1, x_2, \dots, x_i, x_j, \dots, x_N)}^2 
            = 
            \abs{\Psi(x_1, x_2, \dots, x_j, x_i, \dots, x_N)}^2.
        \end{equation}
        For any arbitrary permutation, this is equivalent to 
        \begin{equation}
            \Psi(x_1, \dots x_N) 
            =
            e^{i\alpha(\sigma)}\Psi(x_{\sigma(1)}, x_{\sigma(2)}, \dots, x_{\sigma(N)}),
        \end{equation}
        where $\sigma \in S_N$ is some permutation of $N$ indices and $\alpha$ is 
        some real number that may be dependent on $\sigma$.
        The same relation can be written by way of a linear permutation operator,
        \begin{equation}
            (\hat{P_\sigma}\Psi)(x_1, \dots, x_N)
            =
            \Psi(x_{\sigma(1)}, x_{\sigma(2)}, \dots, x_{\sigma(N)}).
        \end{equation}
        The `indistinguishability postulate'' states that if a permutation $P$ is applied 
        to a state representing an assembly of particles, there is no way of distinguishing
        between the permuted state and the original, by means of an observation at any time.
   
        One can show (Difficult to show? exercise 2.2 in FYS-KJM4480) that 
        \begin{equation}
            \hat{P}_\sigma\Psi = 
            \begin{cases}
                \Psi &\\
                (-1)^{\abs{\sigma}}\Psi& 
            \end{cases} \forall \sigma \in S_N
        \end{equation}
        where $\abs{\sigma}$ is the number of transpositions in $\sigma$ and the sign 
        will be $(-1)^{\abs{\sigma}} = \pm 1$. In the former case, when the sign is $+$,
        the wavefunction is ``totally symmetric with respect to permutations''; while in 
        the latter case, when the sign is $-$, the wavefunction is ``totally anti-symmetric.''

        This leads us to another postulate in quantum theory that we have only two types of 
        basic particles, \emph{bosons} have totally symmetric wavefunctions only, while 
        \emph{fermions} have totally anti-symmetric wavefunctions only. ``The physical 
        consequences of this postulate seems to be in good agreement with experimental data''
        \cite{leinaas1977theory}. Moreover, all particles with integer spin are bosons, 
        and all particles with half-integer spin are fermions
        \cite{fierz1939relativistische,pauli1940connection}. This can be proved in relativistic
        quantum mechanics, but must be accepted as an axiom in nonrelativistic 
        theory\cite{hilborn1995atoms}. Boson follow Bose-Einsten statistics and fermions
        follow Fermi-Dirac statistics.

        To this day, particles with no other spin has been found, but norwegian physicists
        Jon Magne Leinaas and Jan Myrheim discovered that in one- and two dimensions, more 
        general permutations symmetries are possible. The dubbed this third class of
        fundamental particles "anyons"\cite{leinaas1977theory}.
        
    \section{Representation of the Wavefunction}

        We have already invested some time in what the wave-function is, but some more time 
        is necessary in order to build a nomenclature for writing down wavefunctions that
        actually describe many-electron systems with which we are concerned. For some smaller
        systems it can be satisfactory or even provident to use a single, special function 
        to describe the entire system. Here however, we introduce the Slater determinant as 
        we will only consider many-electron wavefunctions that can be written as a single 
        Slater determinant or as a linear combination of several Slater determinants.

        We define an \emph{orbital}\footnote{Sometimes also called a single-particle function,
        a single-particle orbital, a single-electron orbital or similar. There is a chance that 
        these terms will be used interchangably throughout this text without warning.} which is the
        wavefunction for a single particle, or more precicely a single electron. The wavefunction
        a larger group of electrons, for instance those electrons surround an atom or molecule,
        we call the \emph{molecular orbital}. We also discriminate between 
        spatial orbitals which are functions of spatial coordinates; and spinorbitals, which 
        are functions of the space and spin coordinates (typically a product of a spatial orbital
        and a spin function). A very complete description and thorough discussion of all things 
        concerning electronic sructure wavefunctions is given by Szabo and Ostlund\cite{szabo2012modern}. 

        \subsection{Slater Determinants}

        The best description for a multiple-electron wavefunction, given by the independent-particle
        approximation is the Slater determinant,
        \begin{equation}
            \label{eq:general_slater_determinant}
            \Phi = \frac{1}{\sqrt{N!}} \begin{vmatrix}
                \psi_1(1) & \psi_2(1) & \dots & \psi_N(1) \\
                \psi_1(2) & \psi_2(2) & \dots & \psi_N(2) \\
                \vdots & \vdots & \ddots & \vdots \\
                \psi_1(N) & \psi_2(N) & \dots & \psi_N(N)
            \end{vmatrix}
            = \mathscr{A}\psi_1 \psi_2 \dots \psi_N,
        \end{equation}
        where $\psi_i(\mu)$ is a spinorbital and $\mathscr{A}$ is the antisymmetriser. The
        spinorbitals, are single-particle functions in $L^2(X)$, not necessarily orthonormal.

        To illustrate why this is a good approximation of the electronic wave function,
        condiser first the two-electron case,
        \begin{equation}
            \Phi_{N=2} = \frac{1}{\sqrt(2)}(\psi_1(1)\psi_2(2) - \psi_1(2)\psi_2(1)).
        \end{equation}
        We see from this relatively simple expression that if the electrons where to occupy
        the same state. This ensures that the Pauli exclusion principle for 
        fermions\cite{pauli1925zusammenhang}. Moreover, if we switch coordinates of any two 
        single-particle functions (spinorbitals), corresponding to the interchange of rows in
        \autoref{eq:general_slater_determinant}, the result is a change of sign. This attribute
        accomodates the total anti-symmetry necessary for a fermionic wavefunction.       