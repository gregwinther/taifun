\chapter{Quantum Mechanics}
    \epigraph{Hierzu ist es notwendig, die Energy nicht als eine stetige
    unbeschränkt teilbare, sondern als eine discrete, ause einer ganzen
    Zahl von endlichen gleichen Teilen zusammengesetzte Grösse 
    aufzufassen.}
    {--- Max Planck}


\section{The Dirac-von Neumann Postulates}

    This is the TL;DR version of Quantum Mechanics.

    \paragraph{Hilbert Space}
    A quantum state of an isolated physical system is described by a vector
    with unit norm in a Hilbert space, a complex vector space quipped with a
    scalar product. 

    \paragraph{Observables}
    Each physical observable of a system is accociated with a \emph{hermitian}
    operator acting on the Hilbert space. The eigenstates of each such
    operator define a \emph{complete}, \emph{orthonormal} set of vectors.

    With $\hat{O}$ an operator, hermiticity means,
    \begin{equation}
        \bra{\phi}\hat{O}\psi\rangle = \langle\hat{O}\phi\ket{\psi} \equiv \bra{\phi} \hat{O} \ket{\psi}.
    \end{equation}

    Completness means,
    \begin{equation}
        \sum_i \ket{i}\bra{i} = \mathds{1}.
    \end{equation}

    Orthonormal means,
    \begin{equation}
        \braket{i}{j} = \delta_{ij}.
    \end{equation}

    \paragraph{Time Evolution}
    The time evolution of the state vector, $\ket{\psi} = \ket{\psi (t)}$, is given by the Schrödinger
    equation\footnote{In the Schrödinger picture.}.
    \begin{equation}
        i\hbar \frac{d}{dt} \ket{\psi (t)} = \hat{H} \ket{\psi (t)}.
    \end{equation}

    \paragraph{Measurments}
    Physically measurable values, associated with an obeservable $\hat{O}$ are defined by the 
    eigenvalues $o_n$ of the observable,
    \begin{equation}
        \hat{O}\ket{n} = o_n\ket{n}.
    \end{equation}
    The probability for finding a particular eigenvalue in the measurement is
    \begin{equation}
        p_n = \abs{\braket{n}{\psi}}^2,
    \end{equation}
    with the system in state $\ket{\psi}$ before the measurement, and $\ket{n}$ as the 
    eigenstate corresponding to the eigenvalue $o_n$.