\chapter{Quantum Mechanics}
    \epigraph{Hierzu ist es notwendig, die Energy nicht als eine stetige
    unbeschränkt teilbare, sondern als eine discrete, ause einer ganzen
    Zahl von endlichen gleichen Teilen zusammengesetzte Grösse 
    aufzufassen.}
    {--- Max Planck}


\section{Classical Mechanics}

    The formalism used in quantum mechanics largely stems from William Rowan Hamilton's 
    formulation of classical mechanics. Through the process of canonical
    quantisation any classical model of a physical system is turned into
    a quantum mechanical model.

    In Hamilton's formulation of classical mechanics, a complete description of a system
    of N particles is described by a set of canonical coordinates 
    $q = (\vec{q}_1, \dots, \vec{p}_N)$ and corresponding conjugate momenta
    $p = (\vec{p}_1, \dots \vec{p}_N)$. Together, each pair of coordinate and momentum
    form a point $\xi = (q, p)$ in phase space, which is a space of all possible states
    of the system. Moreover, pairs of generalised coordinates and conjugate
    momenta are canonical if they satisfy the Poisson brackets so that 
    $\{q_i, p_k\} = \delta_{ij}$. The Poisson bracket of two functions is defined as 

    \begin{equation}
        \label{eq:poisson_bracket}
        \{f, g\} = \frac{\partial f}{\partial q} \frac{\partial g}{\partial g}
        - \frac{\partial f}{\partial p} \frac{\partial g}{\partial q}.
    \end{equation}

    The governing equations of motion in a classical system is Hamilton's equations,

    \begin{align}
        \dot{q} &= \frac{\partial}{\partial p} \mathscr{H}(q, p) \\
        \dot{p} &= -\frac{\partial}{\partial q} \mathscr{H}(q, p)
    \end{align}

    where $\mathscr{H}(q, p)$ is the Hamiltonian, a function for the total energy of the
    system. Hamilton's equations may also be stated in terms of the Poisson brackets,

    \begin{equation}
        \frac{dp_i}{dt} = \{p_i, \mathscr{H}\}, \ \frac{dq_i}{dt} = \{q_i, \mathscr{H}\}.
    \end{equation}

    A system consisting of $N$ of equal mass $m$, subject forces caused by an external
    potential, as well as acting on each other with forces stemming from a central
    potetntial $w(q_ij)$ has the following Hamiltonian,

    \begin{equation}
        \mathscr{H}(q, p) = \mathscr{T}(q) + \mathscr{V}(p) + \mathscr{W}(p) 
            = \frac{1}{2m}\sum_{i} \abs{\vec{p}_i}^2 + \sum_{i} v(\vec{r}_i)
                + \frac{1}{2}\sum_{i<j} w(\vec{r_{ij}}).
    \end{equation}

    This Hamiltonian conveniently contains several parts - the kinetic energy, the
    external potential energy and the interaction energy; denoted by $\mathscr{T}$,
    $\mathscr{V}$ and $\mathscr{W}$ respectively.

\section{Canonical Quantisation}

    In order to transition from a classical system to a quantum system, we move from 
    the classucal phase space to the Hilbert space, through the procedure known as 
    canonical, or first\footnote{Second quantisation comes later.}-, quantisation.
    Whilst the state of a classical system is a point in phase space, a quantum state
    is a complex-valued state vector in discrete, infinite-dimensional, Hilbert space,
    that is a complete vector space equipped with an inner product. This space
    is most commonly chosen to be the space of square-integrable functions $\Psi$,
    dependent on all coordinates
    
    \begin{equation}
        \Psi = \Psi(x_1, x_2, \dots, x_N).
    \end{equation}
    
    These functions
    are dubbed wavefunctions and are maps from a point $(x_1, \dots, x_N)$ in
    configuration space to the complex vector space,

    \begin{equation}
        \Psi: X^N \to \mathds{C}.
    \end{equation}

    It has been widely discussed how such an object can represent the state of a 
    particle. The answer is provided by Max Born's probabilistic interpretation,
    which says that $\abs{\Psi(x_1, \dots, x_N)}^2$, gives the probability of finding
    the particle at a certain position. For a situation with one particle in one
    dimension we have,

    \begin{equation}
        \int_a^b \abs{\Psi(x)}^2 dx = 
        \left\{\begin{aligned}
            \text{probability of finding the} \\
            \text{particle between $a$ and $b$}
        \end{aligned}\right\}
    \end{equation}

    while $\abs{\Psi(x_1, x_2, \dots, x_N)}^2$ is the probability density for locating
    all particles at the point $(x_1, \dots x_N) \in X^N$. Since the total probability
    must be 1, we are provided with a normalisation condition for the wavefunction,
    
    \begin{equation}
        \int_{X^N} \abs{\Psi(x_1, x_2, \dots, x_N)}^2 dx_1 dx_2\dots dx_N = \mathds{1}.
    \end{equation}

    The relation between a classical- and quantum decsription of a mechanical system 
    is most clearly seen when the two descriptions are expressed in terms of the same 
    variables. In fact, we may apply \emph{quantisation} of the classical variables to 
    produce the quantum equivalent of the system. The classical phase space variables 
    are changed into quantum observables,
    \begin{equation}
        q_i \to \hat{q}_i, \ p_i \to \hat{p}_i.
    \end{equation}
    The quantum observables are required to satisfy the Heisenberg commutation relation,
    \begin{equation}
        [\hat{q}_i, \hat{p}_j] = i \hbar \delta_{ij},
    \end{equation}
    instead of the Poisson bracket (\autoref{eq:poisson_bracket}). Here $\hbar$ is the 
    reduced Planck's constant. For any general variables $A$ and $B$, this 
    transition can be expressed by subsition between Poission brackets for the 
    classical variables and commutators for the quantum observables,
    \begin{equation}
        \{A, B\} \to \frac{1}{i\hbar}[\hat{A},\hat{B}].
    \end{equation}

    This correspondence between the classical and quantum dynamical equations is 
    directly related to \emph{Ehrenfest's therom}, which states that the classical 
    dynamical equations keep their validity also in the quantum theory, with the 
    classical variables repaced by their corresponding quantum expectation
    values\cite{ehrenfest1927bemerkung}.

\subsection{The Dirac-von Neumann Postulates}

    The following postulates, or axioms, provide a precise and concise description 
    of quantum mechanics in terms of operators on the Hilbert space. There are
    many variations of these postulates, introduced both by their namesakes 
    Paul Adriene Maurice Dirac\cite{dirac1930principles} and John von 
    Neumann\cite{vonNeumann1932foundations}.

    \subsubsection{Hilbert Space}
    A quantum state of an isolated physical system is described by a vector
    with unit norm in a Hilbert space $\mathcal{H}$, 
    a complex vector space equipped with an inner product. The inner product 
    associates a scalar value, which may be either real or complex, with any pair
    of state vectors.
    
    The inner product can be defined as 
    \begin{equation}
        \braket{\psi_\alpha}{\psi_\beta} = \int \psi_\alpha^*(x) \psi_\beta(x)dx,
    \end{equation}
    where $\psi^*$ is the complex conjugate of $\psi$. Here we have introduced 
    Dirac notation, which is very common when describing quantum states. For each 
    quantum state $\ket{\psi_\alpha}$ there exists a dual state $\bra{\psi_\alpha}$.
    We refers to these two vectors as \emph{bra} and \emph{ket} vectors, respectably.
    Some properties of a Dirac-style inner product read
    \begin{align}
        \braket{\psi_\alpha}{\psi_\beta} =& \braket{\psi_\beta}{\psi_\alpha}^* \\
        \bra{\psi_\alpha}\left(z_1\ket{\psi_\beta} + z_2 \ket{\psi_\beta}\right)
            =& z_1 \braket{\psi_\alpha}{\psi_\beta} + z_2\braket{\psi_\alpha}{\psi_\beta} \\
        \braket{\psi_\alpha} \geq& 0,
    \end{align}
    where $z_n = a_n + ib_n$ is some complex number. Notice that in these properties, a 
    superposition of a state wavefunction in the form of linear combination of two other 
    states have appeared. Such superpositions are generally written
    \begin{equation}
        \label{eq:superposition}
        \ket{\psi_\gamma} = z_1\ket{\psi_\alpha} + z_2\ket{\psi_\beta},
    \end{equation}
    where we have produced a new quantum state from two other states. Any two or more
    states may be superposed to produce a new state. Superposition is of fundamental 
    importance to quantum mechanics, and even though the concept is similar to the 
    classical superposition principle for waves in classical physics, "the superposition 
    that occurs in quantum mechanics is of an essentially different nature from any 
    occuring in the classical theory" according 
    to \citeauthor{dirac1930principles}\cite{dirac1930principles}.

    Proceeding with the description of quantum states, a state function $\psi$ is said 
    to be \emph{normal} if its innerproduct with itself is one, $\braket{\psi}{\psi} = 1$.
    Two different state functions are \emph{orthogonal} if their inner product is zero. 
    We have orthogonal functions if $\braket{\psi_\alpha}{\psi_\beta} = \delta_{\alpha\beta}$,
    where $\delta_{\alpha\beta}$ is the Kronecker delta.


    \subsubsection{Observables}
    Each physical observable of a system is accociated with a \emph{hermitian}
    operator acting on the Hilbert space. The eigenstates of each such
    an operator define a \emph{complete}, \emph{orthonormal} basis set of vectors
    $\mathcal{B}$ for the $d$-dimensional Hilbert space,
    \begin{equation}
        \mathcal{B} = \{\ket{i}\}_{i=1}^d
    \end{equation}
    Completeness of the basis set $\mathcal{B}$ means,
    \begin{equation}
        \sum_{i=1}^d \ket{i}\bra{i} = \mathds{1}.
    \end{equation}

    With $\hat{O}$ an operator, \emph{hermiticity} means,
    \begin{equation}
        \bra{\phi}\hat{O}\psi\rangle = \langle\hat{O}\phi\ket{\psi} \equiv \bra{\phi} \hat{O} \ket{\psi}.
    \end{equation}
    This means that the operator $\hat{O}$ is its own Hermitian conjugate,
    \begin{equation}
        \hat{O}^\dagger = \hat{O}.
    \end{equation}
    Some properties of the Hermitian conjugate reads,
    \begin{align}
       (z\hat{O})^\dagger =& z^*\hat{O}^\dagger \\
       (\hat{O}_1 + \hat{O}_2)^\dagger =& \hat{O}_1^\dagger \hat{O}_2^\dagger \\
       (\hat{O}_1\hat{O_2})^\dagger =& \hat{O}_2^\dagger \hat{O}_1^\dagger.
    \end{align}

    \subsubsection{Measurements}
    Physically measurable values, associated with an observable $\hat{O}$ are defined by the 
    eigenvalues $o_n$ of the observable,
    \begin{equation}
        \hat{O}\ket{n} = o_n\ket{n}.
    \end{equation}
    The probability for finding a particular eigenvalue in the measurement is
    \begin{equation}
        p_n = \abs{\braket{n}{\psi}}^2,
    \end{equation}
    with the system in state $\ket{\psi}$ before the measurement, and $\ket{n}$ as the 
    eigenstate corresponding to the eigenvalue $o_n$.
    If the observable $\hat{O}$ is Hermitian, we can write the operators as a spectral 
    decomposition 
    \begin{equation}
        \hat{O} = \sum_{n=1}^d o_n \dyad{n},
    \end{equation}
    where $d$ is the dimensionality of the Hilbert space.

    \subsubsection{Time Evolution}
    In the \emph{Schrödinger picture}
    time evolution of the state vector, $\ket{\psi} = \ket{\psi (t)}$, is given by the Schrödinger
    equation\footnote{In the Schrödinger picture.}.
    \begin{equation}
        i\hbar \frac{d}{dt} \ket{\psi (t)} = \hat{H} \ket{\psi (t)}.
    \end{equation}
    Note that any superposed state, as described by \ref{eq:superposition}, will also be 
    a solution to the Schrödinger equation due to its linearity.

    The Schrödinger equation is first order in the time derivate, meaning that the time 
    evolution $\ket{\psi} = \ket{\psi(t)}$ is uniquely determined by some initial 
    condition $\ket{\psi_0} = \ket{\psi(t_0)}$. $\hat{H}$ is the Hamiltonian of the system,
    which is a \emph{linear}, \emph{hermitian} operator. The Hamiltonian gives rise to the 
    time evolution, which is a \emph{unitary} mapping between quantum states in time,
    \begin{equation}
        \ket{\psi(t)} =  \hat{\mathcal{U}}(t, t_0) \ket{t_0}.
    \end{equation}
    The time evolution operator $\hat{\mathcal{U}}$ is determined by the Hamiltonian 
    through the equaton 
    \begin{equation}
        i\hbar \frac{\partial}{\partial t} \hat{\mathcal{U}}(t, t_0)
            = \hat{H}\hat{\mathcal{U}}(t, t_0),
    \end{equation}
    which follows from the Schrödinger equation. For a time-independent 
    Hamiltonian it is given by 
    \begin{equation}
        \hat{\mathcal{U}}(t, t_0) = e^{-i \hat{H}(t - t_0/\hbar)}.
    \end{equation}
    We see that this time-propagater is Hermitian 
    \begin{equation}
        \hat{\mathcal{U}}\hat{\mathcal{U}}^\dagger 
        = \hat{\mathcal{U}}^\dagger \hat{\mathcal{U}} = 1.
    \end{equation}
    If however $\hat{H}$ is time-dependent, so that the operator at different times do 
    not commute, we may use a mroe general integral expression for the time-propagator,
    \begin{equation}
        \hat{\mathcal{U}}(t, t_0)
         = \sum_{n=0}^\infty (-\frac{i}{\hbar})^n
            \int_{t_0}^t \int_{t_0}^{t_1}dt_1 \int_{t_0}^{t_1}dt_2 \dots \int_{t_0}^{t_{n-1}}
                dt_n \hat{H}(t_1)\hat{H}(t_2) \dots \hat{H}(t_n).
    \end{equation}

    A unitary transformation of states and observables
    \begin{equation}
        \ket{\psi} \to \ket{\psi'} = \hat{U}\ket{\psi}, \ 
        \hat{O} \to \hat{O}' = \hat{U}\hat{O}\hat{U}^\dagger, \ 
        \hat{U}^\dagger\hat{U} = 1.
    \end{equation}
    leads to a different, but equivalent representation of a quantum system.
    The transition to the \emph{Heisenberg picture} is defined by a special
    time-dependent unitary transformation,
    \begin{equation}
        \hat{U} = \hat{\mathcal{U}}^\dagger(t, t_0),
    \end{equation}
    which is the inverse of the time-evolution operator. When applied to the 
    time-dependent state vector of the Schrödinger picture it will cancel 
    the time-dependence
    \begin{equation}
        \ket{\psi}_\text{H} 
        = \hat{\mathcal{U}}^\dagger(t, t_0) \ket{\psi(t)}_\text{S}. 
        = \ket{\psi(t_0)}_\text{S}
    \end{equation}
    The time-dependece is now carried by the observables, rather than the state 
    vectors,
    \begin{equation}
        \hat{O}_\text{H} 
        = \hat{\mathcal{U}}^\dagger(t, t_0)
        \hat{O}_\text{S}
        \hat{\mathcal{U}}(t, t_0),
    \end{equation}
    and the Schrödinger equation is replaced by the Heisenberg equation
    \begin{equation}
        \frac{d}{dt} 
        = \frac{i}{\hbar}[\hat{H}, \hat{O}_\text{H}]
        + \frac{\partial}{\partial t} \hat{O}_\text{H}.
    \end{equation}

    There is a third representation called the \emph{interaction picture}, but we will 
    remain firmly rooted in the Schrödinger picture and halt this general introduction 
    to time-development here. 

\section{The Quantum Hamiltonian}

    The full Hamiltionian for a quantum many-body system can be a large 
    and unwieldy thing. In this study we will constrain ourselves to the study of
    electronic systems. Purely on a phenomenological basis, one would include nuclear
    terms in the Hamiltonian as well. In this study however, we will stay within the
    Born-Oppenheimer approximation and treat the nuclei as stationary particles, thereby
    refraining from introducing terms that involve the motion of nuclei.

    The full  molecular electronic Breit-Pauli Hamiltionian, throrougly described in 
    Helgaker et al.\cite{helgaker2012recent}, contains the following types of terms
    \begin{equation}
        \label{eq:breit_pauli_hamiltonian}
        \hat{H}^{\text{BP}}_{\text{mol}}
        = \begin{cases}
            \hat{H}_{\text{kin}} \quad &\leftarrow \text{kinetic energy} \\
            + \hat{H}_{\text{cou}} \quad &\leftarrow \text{Coulomb interactions} \\
            + \hat{H}_{\text{ee}} \quad &\leftarrow \text{external electric field interaction} \\
            + \hat{H}_Z \quad &\leftarrow \text{Zeeman interactions} \\
            + \hat{H}_{\text{so}} &\leftarrow \text{spin-orbit interactions} \\
            + \hat{H}_{\text{ss}} &\leftarrow \text{spin-spin interactions} \\
            + \hat{H}_{\text{oo}} &\leftarrow \text{orbit-orbit interactions} \\
            + \hat{H}_{\text{dia}} &\leftarrow \alpha^4 \text{ diamagnetic interactions}
        \end{cases}
    \end{equation}
    We will not be working with this giant beast, but will go into some of the most  
    import terms that an electronic Hamiltonian can constitute. 
    
    \paragraph{Kinetic energy}

    The genereal kinetic energy operator is given by
    \begin{equation}
        \label{eq:kinetic_energy_term}
        \hat{H}_{\text{kin}}
            = - \frac{\hbar}{2m}\sum_i \nabla^2_i,
    \end{equation}
    where $\nabla$ is the differential operator and the sum is over all electrons in 
    the system. This term is an example of a one-particle operator as it remains the 
    same for all eletrons and has is contains to terms that would constitute interaction 
    between particles. For a free particle or a gas of non-interacting particles, 
    \autoref{eq:kinetic_energy_term} is sufficient to describe the system. 

    \paragraph{Potential terms}

    By adding a confining potential to the Hamiltonian in addition to the kinetic 
    energy term \autoref{eq:kinetic_energy_term},
    \begin{equation}
        \hat{H} \supset \hat{H}_\text{kin} + \hat{V}
    \end{equation}
    gives rise to much more intersting systems and are the beginning of an approximation 
    of reality. Perhaps the most common is the harmonic oscillator potential, which 
    in one dimension reads 
    \begin{equation}
        \hat{V}(x) = \frac{1}{2}m\omega^2x^2.
    \end{equation}
    This is a very popular confining potential because virtually any oscilaltory motion can 
    be approximated by it, if the amplitude of the oscillations are small. The parabolic 
    potential is the basis of a quantum dot which is a central part of this study. 
    In quantum chemistry we only consider potentials derived from particle-particle 
    interactions, and not such external potentials.

    \paragraph{Coulomb interactions}

    Electrostatic interaction between particles in a molecule or atom is 
    modelled by Coulomb terms in the Hamiltonian. 
    \begin{equation}
            \hat{H}_{\text{cou}} 
                = -\sum_{iK}\frac{k_eZ_Ke^2}{r_{iK}} 
                    + \frac{1}{2}\sum_{i \neq j} \frac{k_e e^2}{r_{ij}} 
                    + \frac{1}{2}\sum_{K \neq L} \frac{k_eZ_K Z_Le^2}{R_{KL}},
    \end{equation}
    where $e$ is the elementary particle charge and $k_e = 1 / 4\pi\epsilon_0$ is the 
    Coulomb constant.
    The first term is the potential between nuclei and electrons, the second term is
    the potential between electrons and the last term is the potential between 
    nuclei. In the Born-Oppenheimer approximaton we ignore the first and third term.

    \paragraph{External electric field interactions}
    
    Applying an electromagnetic field we need to include terms in the Hamiltonian that 
    model the effects of externally applied scalar potential $\phi = \phi(\vb{r}, t)$
    and vector potential $\vb{A} = \vb{A}(\vb{r}, t)$. This will also affect the 
    kinetic energy of the particles, which we therefore include at first,
    \begin{equation}
        \hat{H}_\text{ef} 
            \subset \frac{1}{2m}(\vb{p} + e \vb{A})^2 - e\phi
            = \frac{\hat{\vb{p}}}{2m} + \frac{e}{2m}(\vb{A}\cdot\hat{\vb{p}} + \hat{\vb{p}}\cdot\vb{A})
                + \frac{e^2}{2m}\vb{A}^2 - e\phi.
    \end{equation}
    This Hamiltonian now describes a free partice subject to an external 
    electro field. Since we have already included a term for kinetic energy we 
    now wish to remove it,
    \begin{equation}
        \hat{H}_\text{ef} 
            = - \frac{e}{2m}(\vb{A}\cdot\hat{\vb{p}} + \hat{\vb{p}}\cdot\vb{A})
                + \frac{e^2}{2m}\vb{A}^2 - e\phi.
    \end{equation}

    We assume that the external field has sufficiently large wavelenght compared to the 
    system, making the vector potential uniform in space $\vb{A}(\vb{r}, t) = \vb{A}(t)$
    This approximation is very reasonable, because visible light has a wavelength of 
    $\lambda \sim 5000 \text{Å}$ and an atom is around $1 \text{Å}$. In the dipole 
    approximation we can write the vector potential as 
    \begin{equation}
        \vb{A}(t) = \vb{A}_0e^{-i \omega_k t} + \vb{A}_0^*e^{i\omega_k t}a,
    \end{equation}
    where $\vb{A}_0$ and $\vb{A}_0^*$ are photon creation and annihilation operators. 
    These allow for spontaneous emission and absorption of photons without the presence 
    of a field, but we will stick to a semi-classical world where such things are 
    impossible. We therefore rewrite the vector potential to 
    \begin{equation}
        \vb{A}(t) = \epsilon A_0\sin(\omega_k t),
    \end{equation}
    which is the same up to a phase. In the Coulomb gauge we have 
    \begin{equation}
        \vb{E} = - \frac{d}{dt}\vb{A},
    \end{equation}
    which gives us an expression for the E-field 
    \begin{equation}
        \vb{E} = \epsilon \mathscr{E}_0 \cos(\omega_k t),
    \end{equation}
    where $\mathscr{E}_0 = -\omega_k A_0$ and we can approximate the external electric 
    field time-dependent field by 
    \begin{equation}
        \hat{H}_\text{ef} = - \hat{d} \cdot \epsilon \mathscr{E}_0 \cos{\omega_k t},
    \end{equation}
    where $\hat{\vb{d}} = q\hat{\vb{r}}$, is the dipole operator, dictating the 
    allowed transitions.

    \subsection{Angular Momentum and Intrinsic Spin}

    In general, modelling of interaction with magnetic fields necessitates the 
    use of operators for intrinsic angular momentum ($\vb{S}$) and extrinsic 
    angular momentum ($\vb{L}$). These are often referred to as \emph{spin} 
    and \emph{angular momentum}, respectively.
    We will spend some time here elaborating on such terms.

    \subsubsection{Angular momentum}
    The angular momentum in a classical system the angular momentum of a particle with 
    resepct to the orign is given as
    \begin{equation}
        \vb{L} = \vb{r} \times \vb{p},
    \end{equation}
    which broken down into components becomes,
    \begin{equation}
        L_x = y p_z - z p_y,\ 
        L_y = y p_x - x p_z,\ 
        L_z = x p_y - y p_x.
    \end{equation}
    From this we can obtain the quanutm mechanical description by promotion to operators,
    and inserting the representaion for the momentum operator in position space,
    $p_x \to i\hbar \partial / \partial x$.

    The commutators of the angular momentum operators follows a nice cyclic permutation 
    of indices rule:
    \begin{equation}
        [\hat{L}_x, \hat{L}_y] = i\hbar \hat{L}_z, \  
        [\hat{L}_y, \hat{L}_z] = i\hbar \hat{L}_x, \ 
        [\hat{L}_z, \hat{L}_x] = i\hbar \hat{L}_y.
    \end{equation}
    We see that they do not commute with each other, but the square of the total 
    angular momentum, defined by 
    \begin{equation}
        \hat{L}^2 \equiv \hat{L}_x^2 + \hat{L}_y^2 + \hat{L}_z^2,
    \end{equation}
    does
    \begin{equation}
        [\hat{L}^2, \hat{L}_x] = 0, \ 
        [\hat{L}^2, \hat{L}_y] = 0, \ 
        [\hat{L}^2, \hat{L}_z] = 0.
    \end{equation}

    In spherical coordinates, which is better suited for our needs, the angular 
    momentum operator is given by
    \begin{equation}
        \hat{\vb{L}} = -i\hbar \hat{\vb{r}} \times \nabla.
    \end{equation}
    Given the gradient in spherical coordinates,
    \begin{equation}
        \nabla = 
          e_r \frac{\partial}{\partial} 
        + e_\theta \frac{1}{r} \frac{\partial}{\partial \theta}
        + e_\phi \frac{1}{r\sin \theta} \frac{\partial}{\partial \phi}
    \end{equation}
    and $\vb{r} = re_r$ we get 
    \begin{equation}
        \begin{aligned}
        \hat{\vb{L}}
        &= 
        -i\hbar\left[ 
              r(e_r \times e_r)\frac{\partial}{\partial r} 
            + (e_r \times e_\theta)\frac{\partial}{\partial \theta}
            + (e_r \times e_\phi) \frac{1}{\sin\theta} \frac{\partial}{\partial \phi}  
        \right] \\
        &=
        -i\hbar \left( 
        e_\phi \frac{\partial}{\partial \theta}
        - e_\theta \frac{1}{\sin\theta}\frac{\partial}{\partial \phi}     
        \right).
        \end{aligned}
    \end{equation}
    Now, with an algebraic spinning backflip kick and 
    \begin{align}
        \hat{L}_x &= i\hbar \left(
            \sin\phi \frac{\partial}{\partial\theta}
            +\cos\phi\cot\theta\frac{\partial}{\partial \phi}
        \right) \\
        \hat{L}_y &= i\hbar \left(
            \cos\phi \frac{\partial}{\partial\theta}
            +\sin\phi\cot\theta\frac{\partial}{\partial \phi}
        \right) \\
        \hat{L}_z &= -i\hbar \frac{\partial}{\partial \phi}
    \end{align}
    falls out of the piñata.
    The squared operator becomes
    \begin{equation}
        \hat{L}^2 = -\hbar^2 \left[
        \frac{1}{\sin\theta}\frac{\partial}{\partial\theta}
            \left(\sin\theta\frac{\partial}{\partial\theta} \right)
        + \frac{1}{\sin^2\theta}\frac{\partial^2}{\partial\phi^2}
        \right]
    \end{equation}
    The eigenvalue equations of $\hat{L}^2$ and $\hat{L}_z$ are
    \begin{equation}
        \hat{L}^2\psi = \hbar^2l(l + 1)\psi, \quad \hat{L}_z\psi = \hbar m \psi,
    \end{equation}
    where $\psi = Y^m_l(\theta, \phi)$ are the spherical harmonics,
    \begin{equation}
        Y^m_l (\theta, \phi) 
            = 
            \epsilon\sqrt{\frac{(2l + 1)}{4\pi}\frac{(l - |m|)!}{(1 + |m|)!}}
            e^{im\theta} P^m_l(cos\theta),
    \end{equation}
    and $P^m_l$ are the associated Legendre polynomials.

    \subsubsection{Spin}

    In classical mechanics, intrinsic spin ($\vb{S} = I\omega$) is associated with
    an objects mortion aboth its centre of mass. A similar thing goes on in quantum 
    mechanics, but it has nothing to do with motion in space. In quantum mechanics spin 
    is seen as a property that particles can carry, but i analoguous with its classical 
    counterpart only in name.

    Algebraically, \emph{spin} is the same as \emph{angular momentum}, beginning with 
    the commutator relations,
    \begin{equation}
        [\hat{S}_x, \hat{S}_y] = i\hbar \hat{S}_z, \ 
        [\hat{S}_y, \hat{S}_z] = i\hbar \hat{S}_x, \ 
        [\hat{S}_z, \hat{S}_x] = i\hbar \hat{S}_y,
    \end{equation}
    while the eigenvectors of $\hat{S}^2$ and $S_z$ satisfy 
    \begin{equation}
        \hat{S}^2\ket{s\ m_s} = \hbar^2 s(s + 1)\ket{s\ m_s}, \quad
        \hat{S}_z\ket{s\ m_s} = \hbar m \ket{s\ m_s}.
    \end{equation}
   
    Because the quantum mechanical spin has nothing to do with motion in space and 
    is independent of any coordinates $r$, $\theta$ or $\phi$m there is no reason to 
    exclude half-integer values of $s$ and $m_s$,
    \begin{equation}
        s=0,\frac{1}{2},1,\frac{3}{2},\dots, \quad m_s\in[-s, s].
    \end{equation}
    As it would turn out, every elemental particle as a specific and immutable value of 
    $s$, and the most important one is $s=\frac{1}{2}$ (!) as it is the spin-value for 
    all leptons, including the electron, all quarks as well as protons and neutrons. Since 
    our particle of scrutiny is the electron our interest lies in spin-half systems, i.e. 
    \begin{equation}
        s = \frac{1}{2}, \quad m_s = \pm \frac{1}{2}.
    \end{equation}
    In such a system there are only two eigenstates,
    \begin{equation}
        \ket{\frac{1}{2},+\frac{1}{2}} = \ket{\uparrow} = \ket{+}, \quad
        \ket{\frac{1}{2},-\frac{1}{2}} = \ket{\downarrow} = \ket{-}.
    \end{equation}
    This means that the Hilbert space $\mathcal{H}$ for spin-half particles has 
    two dimensions. This means that any state can be expressed as a two-dimensional 
    vectors called spinors,
    \begin{equation}
        \ket{\chi} = \begin{pmatrix}
            a \\ b
        \end{pmatrix}
        = a\ket{\uparrow} + b\ket{\downarrow}
        = a\begin{pmatrix}
            1 \\ 0 
        \end{pmatrix}
        + b\begin{pmatrix}
            0 \\ 1
        \end{pmatrix}.
    \end{equation}
    The probability of finding a particle represented by the state vector $\ket{\chi}$
    in the spin up or spin down state is $|a|^2$ and $|b|^2$, respectively. This requires 
    a normalisation, $|a|^2 + |b|^2 = 1$.

    In the basis of $\ket{\uparrow}$ and $\ket{\downarrow}$ the operators $\hat{S}^2$, 
    $\hat{S}_x$, $\hat{S}_y$ and $\hat{S}_z$ are represented by two-dimensional matrices,
    \begin{equation}
        \hat{S}^2 = \frac{3}{4}\hbar^2\begin{pmatrix}
            1 & 0 \\ 0 & 1 
        \end{pmatrix}
    \end{equation}
    \begin{equation}
        \hat{S}_x = \frac{\hbar}{2}\sigma_x, \quad 
        \hat{S}_y = \frac{\hbar}{2}\sigma_y, \quad 
        \hat{S}_z = \frac{\hbar}{2}\sigma_z,
    \end{equation}
    where the Pauli matricies are given by, 
    \begin{equation}
        \sigma_x = \begin{pmatrix} 0 &  1 \\ 1 & 0 \end{pmatrix} \quad 
        \sigma_y = \begin{pmatrix} 0 & -i \\ i & 0 \end{pmatrix} \quad 
        \sigma_z = \begin{pmatrix} 1 &  0 \\ 0 & -1 \end{pmatrix}.
    \end{equation}

    \subsection{Atomic Units}

        It is common practice to switch to a set of units that are easier to work with,
        i.e. setting $\hbar = m_e = e = \dots = 1$. In this study we use atomic units, a
        form of such dimensionless units. To see how these units arise, consider
        the time-independent Schrödinger equation for a Hydrogen atom,

        \begin{equation}
            \left(-\frac{\hbar^2}{2m_e}\nabla^2 - \frac{e^2}{4\pi\epsilon_0 r} \right)
            \phi = E\phi,
        \end{equation}

        where $\hbar$ is the reduced Planck constant, equal to Planck's constant divided by 
        $2\pi$; $m_e$ is the mass of the electron, $-e$ is the charge of the electron and 
        $\epsilon_0$ is the permitivity of free space. We make this equation dimensionless by 
        letting $r \to \lambda r'$,

        \begin{equation}
            \left(-\frac{\hbar^2}{2m_e\lambda^2}\nabla'^2 - \frac{e^2}{4\pi\epsilon_0\lambda r'} \right)
            \phi' = E\phi'.
        \end{equation}

        We can factor out the constants in front of the operators, if we choose $\lambda$ so that,

        \begin{equation}
            \frac{\hbar^2}{m_e\lambda^2} = \frac{e^2}{4\pi \epsilon_0 \lambda} = E_a
            \to \lambda \frac{4\pi\epsilon_0\hbar^2}{m_e e^2} = a_0
        \end{equation}

        where $E_a$ is the atomic unit of energy that chemists call Hartree. Incidently,
        we see that $\lambda$ is just the Bohr radius, $a_0$. If we let $E' = E/E_a$, we 
        obtain the dimensionless Schrödinger equation,

        \begin{equation}
            \left(-\frac{1}{2}\nabla'^2 - \frac{1}{r'} \right) \phi' = E'\phi'.
        \end{equation}

        Some conversion factors between atomic units and SI units can be found in
        \autoref{tab:atomic_units_conversion}.

        \begin{table}
            \centering
            \caption{Conversion of atomic units to SI units.}
            \begin{tabular}{ccc} \hline
                Physical quantity & Conversion factor & Value \\ \hline
                Length  & $a_0$ & $5.2918 \times 10^{-11} m$ \\
                Mass    & $m_e$ & $9.1095 \times 10^{-31} kg$ \\
                Time    & $\hbar/E_a$ & $2.4189 \times 10^{-17} s$ \\
                Charge  & $e$   & $1.6022 \times 10^{-19} C$ \\
                Energy  & $E_a$ & $4.3598 \times 10^{-18} J$ \\
                Velocity& $a_0E_a/\hbar$ & $2.1877 \times 10^{6} ms^{-1}$ \\
                Angular momentum & $\hbar$ &  $1.0546 \times 10^{-34} Js$ \\
                Electric dipole moment & $ea_0$& $8.4784 \times 10^{-30} Cm$ \\
                Electric polarizability & $e^2a_0^2/E_a$ & $1.6488 \times 10^{-41} C^2m^2J^{-1}$ \\
                Electric field & $E_a/(ea_0)$ & $5.1423 \times 10^{11} Vm^{-1} $ \\
                Wave function & $a_0^{-3/2}$ & $2.5978 \times 10^{15} m^{-3/2}$ \\ \hline
            \end{tabular}
            \label{tab:atomic_units_conversion}
        \end{table}


    \section{Indistinguishable Particles}

        In classical mechanics, although particles are indistinguishable, one typically
        regards particles as individuals because a permutation of particles is counted as
        a new arrangement and something different than the initial configuration. 
        This was called ``Transcedental
        Individuality'' by Heinz Post\cite{post1963individuality}. In quantum mechanics, on 
        the other hand, a permutation is not regarded as giving rise to a new 
        arrangement. It follows that quantum objects are very different from anything else we
        know from everyday life, and must be considered ``non-individual''. By taken this idea
        to it's extreme one may postulate that all particles of a given type are one and the
        same. Here from a telephone call betwwen John Wheeler and Richard
        Feynman\cite{feynman1965nobel},
        \say{I received a telephone call one day at the graduate college at Princeton from 
        Professor Wheeler, in which he said, \say{Feynman, I know why all electrons have
        the same charge and the same mass} \say{Why?} \say{Because, they are all the same 
       electron!} }

        Following the brief discussion above one may conclude that, the probability density
        for the location of particles in a system must be permutation invariant,
        \begin{equation}
            \label{eq:square_wavefunction_pauli}
            \abs{\Psi(x_1, x_2, \dots, x_i, x_j, \dots, x_N)}^2 
            = 
            \abs{\Psi(x_1, x_2, \dots, x_j, x_i, \dots, x_N)}^2.
        \end{equation}
        For any arbitrary permutation, this is equivalent to 
        \begin{equation}
            \Psi(x_1, \dots x_N) 
            =
            e^{i\alpha(\sigma)}\Psi(x_{\sigma(1)}, x_{\sigma(2)}, \dots, x_{\sigma(N)}),
        \end{equation}
        where $\sigma \in S_N$ is some permutation of $N$ indices and $\alpha$ is 
        some real number that may be dependent on $\sigma$.
        The same relation can be written by way of a linear permutation operator,
        \begin{equation}
            (\hat{P_\sigma}\Psi)(x_1, \dots, x_N)
            =
            \Psi(x_{\sigma(1)}, x_{\sigma(2)}, \dots, x_{\sigma(N)}).
        \end{equation}
        The `indistinguishability postulate'' states that if a permutation $P$ is applied 
        to a state representing an assembly of particles, there is no way of distinguishing
        between the permuted state and the original, by means of an observation at any time.
   
        One can show (Difficult to show? exercise 2.2 in FYS-KJM4480) that 
        \begin{equation}
            \hat{P}_\sigma\Psi = 
            \begin{cases}
                \Psi &\\
                (-1)^{\abs{\sigma}}\Psi& 
            \end{cases} \forall \sigma \in S_N
        \end{equation}
        where $\abs{\sigma}$ is the number of transpositions in $\sigma$ and the sign 
        will be $(-1)^{\abs{\sigma}} = \pm 1$. In the former case, when the sign is $+$,
        the wavefunction is ``totally symmetric with respect to permutations''; while in 
        the latter case, when the sign is $-$, the wavefunction is ``totally anti-symmetric.''

        This leads us to another postulate in quantum theory that we have only two types of 
        basic particles, \emph{bosons} have totally symmetric wavefunctions only, while 
        \emph{fermions} have totally anti-symmetric wavefunctions only. ``The physical 
        consequences of this postulate seems to be in good agreement with experimental data''
        \cite{leinaas1977theory}. Moreover, all particles with integer spin are bosons, 
        and all particles with half-integer spin are fermions
        \cite{fierz1939relativistische,pauli1940connection}. This can be proved in relativistic
        quantum mechanics, but must be accepted as an axiom in nonrelativistic 
        theory\cite{hilborn1995atoms}. Boson follow Bose-Einsten statistics and fermions
        follow Fermi-Dirac statistics.

        To this day, particles with no other spin has been found, but norwegian physicists
        Jon Magne Leinaas and Jan Myrheim discovered that in one- and two dimensions, more 
        general permutations symmetries are possible. The dubbed this third class of
        fundamental particles "anyons"\cite{leinaas1977theory}.
        