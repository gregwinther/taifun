\chapter{Solving the Schrödinger Equation Analytically}
\label{app:analytical_tise}

Here we will illustrate how one would solve the time-independent Schrödinger
equation analytically in three dimensions. This is not only a nice pedagogical exercise,
but it illustrates the necessity of numerical methods in quantum mechanics. 
Not only is numerical methods necessary because the equtions become extensive and 
difficult for a human to handle, but also because of the relatively low number 
of systems that we can tackle in this manner. Here we also provide the 
solutions for the quantum harmonic oscillator, which is has been very important 
in this thesis.

\section{The Schrödinger Equation in Spherical Coordinates}

Here we will solve the time-independent Schrödinger equation,
\begin{equation}
    \label{eq:tise}
    -\frac{\hbar^2}{2m}\nabla^2\Psi + V\Psi = E\Psi,
\end{equation}
by separation of variables. In \autoref{eq:tise}, $m$ is the mass of the 
particle in the system, $\hbar$ is Planck's reduced constant, $E$ is the eigenenergy,
$\Psi$ the eigenstate of the system. The squared differential operator $\nabla^2$
becomes the following in spherical coordinates,
\begin{equation}
    \label{eq:spherical_laplacian}
    \nabla^2 = 
        \frac{1}{r^2} \frac{\partial}{\partial r} 
            \left(r^2 \frac{\partial}{\partial r}\right)
        + 
        \frac{1}{r^2\sin\theta} \frac{\partial}{\partial \theta}
            \left(\sin\theta \frac{\partial}{\partial \theta} \right)
        +
        \frac{1}{r^2\sin\theta}
            \left(\frac{\partial^2}{\partial\phi^2}\right),
\end{equation}
where we have used the coordinates $r$, $\theta$ and $\phi$ as the radius,
polar angle and azimuthal angle, respectively. Inserting
\autoref{eq:spherical_laplacian} into \autoref{eq:tise} yields,
\begin{equation}
    \label{eq:spherical_tise}
    -\frac{\hbar^2}{2m}
    \left[
        \frac{1}{r^2}\frac{\partial}{\partial r} 
            \left(r^2 \frac{\partial \Psi}{\partial r} \right)
        +
        \frac{1}{r^2\sin\theta} \frac{\partial}{\partial\theta}
            \left(\sin\theta \frac{\partial \Psi}{\partial \theta} \right)
        +
        \frac{1}{r^2\sin\theta}
            \left(\frac{\partial^2 \Psi}{\partial \phi^2} \right)
    \right]
    +
    V \Psi
    =
    E \Psi.
\end{equation}
We look solutions to this equations of the form 
\begin{equation}
    \Psi(r,\theta,\phi) = R(r) Y(\theta, \phi).
\end{equation}
Inserting this into \autoref{eq:spherical_tise} gives,
\begin{equation}
   -\frac{\hbar^2}{2m}
   \left[
        \frac{Y}{r^2} \frac{d}{dr}
            \left(r^2\frac{dR}{dr} \right)
        +
        \frac{R}{r^2\sin\theta}
            \left(\sin\theta \frac{\partial Y}{\partial\theta} \right)
        +
        \frac{R}{r^2\sin^2\theta}
            \left(\frac{\partial^2Y}{\partial\phi^2} \right)
   \right] 
    +
    VRY
    =
    ERY,
\end{equation}
which we multiply by $-\frac{2mr^2}{RY\hbar^2}$,
\begin{equation}
    \begin{aligned}
        &
        \left\{ 
            \frac{1}{R}\frac{d}{dr}
                \left(r^2\frac{dR}{dr} \right)
            -
            \frac{2mr^2}{\hbar^2}
                \left[V(r) - E \right]
        \right\}
        \\ + \frac{1}{Y} &
        \left\{
            \frac{1}{\sin\theta }\frac{\partial}{\partial\theta}
                \left(\sin\theta \frac{\partial Y}{\partial\theta} \right)
            +
            \frac{1}{\sin\theta}\frac{\partial^2Y}{\partial\phi^2}
        \right\}
        = 0,
    \end{aligned}
\end{equation}
where the first part is only dependent on $r$ and the second part is only
dependent on $\theta$ and $\phi$. Notice that we have also assumed a spherical 
symmetric potential, i.e. $V=V(r)$. Next, we introduce the cleverly chosen 
separation constant $l(l + 1)$,
\begin{align}
    \label{eq:the_radial_equation}
    \frac{1}{R}\frac{d}{dr}
        \left(r^2\frac{dR}{dr}\right)
    -
    \frac{2mr^2}{\hbar^2}\left[V(r) - E \right] 
    &= l (l + 1) \\
    \label{eq:the_angular_equation}
    \frac{1}{Y}\left\{ 
        \frac{1}{\sin\theta}\frac{\partial}{\partial\theta}
            \left(\sin\theta \frac{\partial Y}{\partial\theta} \right)
        +
        \frac{1}{\sin^2\theta}\frac{\partial^2 Y}{\partial \phi^2}
    \right\} 
    &= - l (l + 1).
\end{align}

\subsection{The Angular Equation}

Starting with \autoref{eq:the_angular_equation}, which dictates the dependence if
$\Psi$ on $\theta$ and $\phi$,
\begin{equation}
    \frac{1}{\sin\theta}\frac{\partial}{\theta} 
        \left(\sin\theta\frac{\partial Y}{\partial \theta} \right)
    +
    \frac{1}{\sin^2\theta} \frac{\partial^2}{\partial\phi^2}
    =
    -l(l + 1)\sin^2\theta Y.
\end{equation}
We wish to make a further separation, by inserting
$Y(\theta, \phi) = \Theta(\theta)\Phi(\phi)$ and dividing by $\Theta\Phi$,
\begin{equation}
    \frac{1}{\Theta}
    \left[
        \sin\theta \frac{d}{d\theta}
            \left(\sin\theta \frac{d\Theta}{d\theta} \right)
    \right]
    + l(l + 1) \sin\theta 
    +
    \frac{1}{\Phi} \frac{d^2\Phi}{d\phi} = 0,
\end{equation}
where again see that the variables have gathered in separate terms. We therefore 
introduce the separation constant $m^2$,
\begin{align}
    \label{eq:angular_1}
    \frac{1}{\Theta}
    \left[
        \sin\theta \frac{d}{d\theta}
            \left(\sin\theta \frac{d\Theta}{d\theta} \right)
    \right]
    + l(l + 1)\sin\theta &= m^2 \\
    \label{eq:angular_2}
    \frac{1}{\Phi}\frac{d^2\Phi}{d\phi^2} &= -m^2.
\end{align}
Of these two \autoref{eq:angular_2} is instantly recognisable,
\begin{equation}
    \frac{d^2\Phi}{d\phi^2} = -m^2\Phi,
\end{equation}
with the general solution
\begin{equation}
    \Phi(\phi) = C_1e^{im\phi} + C_2 e^{-im\phi},
\end{equation}
but we will choose the more simpler solution 
\begin{equation}
    \Phi(\phi) = e^{im\theta}.
\end{equation}
We require periodicity, i.e. $\Phi(\phi + 2\pi) = \Phi(\phi)$, which means that 
$m$ only takes integer values,
\begin{equation*}
    m = 0, \pm 1, \pm 2, \pm 3, \dots
\end{equation*}

The other angular equation, in \autoref{eq:angular_1}, is not so simple,
\begin{equation}
    \sin\theta\frac{d}{d\theta}\left(\sin\theta\frac{d\Theta}{d\theta} \right)
    + \left[l(l + 1)\sin\theta - m^2 \right]\Theta = 0.
\end{equation}
It has the more complicated solutions,
\begin{equation}
    \label{eq:angular_1_solution}
    \Theta(\theta) = A P^m_l(\cos\theta),
\end{equation}
where $P^m_l$ is the associated Legendre function,
\begin{equation}
    P^m_l(x) \equiv (1-x^2)^{|m|/2} \left(\frac{d}{dx}\right)^{|m|}P_l(x),
\end{equation}
and $P_l$ is the $l$th Legendre polynomial given by the Rodrigues formula,
\begin{equation}
    P_l(x) \equiv \frac{1}{2^l l!}\left(\frac{d}{dx} \right)^l(x^2 - l)^l.
\end{equation}
We see that for this function to be take a value, we must have $l > 0$. Moreover, 
we see that if $|m| > l$ then $P^m_l = 0$. For a given $l$, there must be 
$2l + 1$ possible values for $m$,
\begin{equation*}
    l = 0, 1, 2, \dots \quad m = l, -1 + 1, \dots - 1, 0, 1, \dots l-1, l.
\end{equation*}
A mathematician would at this point argue that because \autoref{eq:angular_1} is 
a second-order differential equation it should have two sets of solutions, not 
just the solution in \autoref{eq:angular_1_solution}. Another set of solutions 
exist but these are not sensible in the physical sense as they have singularities 
at $\theta=0, \pi$.

We would want to normalise the solutions we have found thus far. In spherical 
coordinates the volume element is 
\begin{equation}
    d^3\vb{r} = r^2 \sin\theta dr d\theta d\phi,
\end{equation}
this gives,
\begin{equation}
    \int\int\int |\Psi|^2 r^2\sin\theta dr d\theta d\phi
    = \int |R|^2r^2 dr \int\int|Y|^2\sin\theta dr d\theta d\phi = 1.
\end{equation}
We can normalise these part separately,
\begin{equation}
    \int_0^\infty |R|^2 r^2 dr = 1, \quad 
    \int_0^{2\pi}\int_0^\pi |Y|^2 \sin\theta d\theta d\phi = 1.
\end{equation}
It so happens that the normalised angular wave functions are the 
spherical harmonics,
\begin{equation}
    Y^m_l(\theta, \phi) 
    =
    \epsilon\sqrt{\frac{2l + 1}{4\pi}\frac{(l - |m|)!}{(l + |m|)!}}
    e^{im\phi} P^m_l(\cos\theta).
\end{equation}
A nice attribute of the spherical harmonics is that they are orthogonal.

\subsection{The Radial Equation}

For the type of spherical symmetric system we are dealing with, we have 
saved the best part for last, namely \autoref{eq:the_radial_equation},
which contains the potential $V$,
\begin{equation}
    \label{eq:radial_equation2}
    \frac{d}{dr}\left(r^2\frac{dR}{dr} \right)
    -
    \frac{2mr^2}{\hbar^2} [V(r) - E]R
    =
    l(l + 1)R.
\end{equation}
To make our efforts less effortfull we make a variable change, $u(r) \equiv = r R(r)$,
such that 
\begin{equation*}
    R = \frac{u}{r}, \quad
    \frac{dR}{dr} = \frac{1}{r}\frac{du}{dr} - \frac{u}{r^2}, \quad 
    \frac{d}{dr}\left(r^2 \frac{dR}{dr} \right)
        = \frac{d^2u}{dr^2},
\end{equation*}
making \autoref{eq:radial_equation2},
\begin{equation}
    - \frac{\hbar^2}{2m}\frac{d^2u}{dr^2}
    +
    \left[
        V + \frac{\hbar^2}{2m}\frac{l(l + 1)}{r^2} 
    \right]
    u = Eu,
\end{equation}
which is identical to one-dimensional Schrödinger equation, except for 
a new \emph{effective} potential,
\begin{equation}
    V_\text{eff} = v + \frac{\hbar}{2m}\frac{l(l + 1)}{r^2}.
\end{equation}
For the radial equation with the substituted variable, we also require normalisation,
\begin{equation} 
    \int_0^\infty |u|^2 dr = 1.
\end{equation}

This is as far we get without insertion of a potential. There are a few potentials 
to choose from that are analytically pleasant to deal with. First among them is the 
harmonic oscillator potential.