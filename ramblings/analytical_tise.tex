\chapter{Solving the Schrödinger Equation Analytically}
\label{app:analytical_tise}

Here we will illustrate how one can solve the time-independent Schrödinger
equation analytically in three dimensions. By doing this exercise, we also see
where the quantum numbers associated with the extra degree of freedom the
introduction of an additional dimension entails. Here we also provide some 
motivation for the necessity of numerical approximations, as we introduce a 
few simple systems. Lastly, we give the solutions to the quantum harmonic 
oscillator, an importan system in this thesis.

\section{The Schrödinger Equation in Spherical Coordinates}

Here we will solve the time-independent Schrödinger equation,
\begin{equation}
    \label{eq:tise}
    -\frac{\hbar^2}{2m}\nabla^2\Psi + V\Psi = E\Psi,
\end{equation}
by separation of variables. In \autoref{eq:tise}, $m$ is the mass of the 
particle in the system, $\hbar$ is Planck's reduced constant, $E$ is the eigenenergy,
$\Psi$ the eigenstate of the system. The squared differential operator $\nabla^2$
becomes the following in spherical coordinates,
\begin{equation}
    \label{eq:spherical_laplacian}
    \nabla^2 = 
        \frac{1}{r^2} \frac{\partial}{\partial r} 
            \left(r^2 \frac{\partial}{\partial r}\right)
        + 
        \frac{1}{r^2\sin\theta} \frac{\partial}{\partial \theta}
            \left(\sin\theta \frac{\partial}{\partial \theta} \right)
        +
        \frac{1}{r^2\sin\theta}
            \left(\frac{\partial^2}{\partial\phi^2}\right),
\end{equation}
where we have used the coordinates $r$, $\theta$ and $\phi$ as the radius,
polar angle and azimuthal angle, respectively. Inserting
\autoref{eq:spherical_laplacian} into \autoref{eq:tise} yields,
\begin{equation}
    \label{eq:spherical_tise}
    -\frac{\hbar^2}{2m}
    \left[
        \frac{1}{r^2}\frac{\partial}{\partial r} 
            \left(r^2 \frac{\partial \Psi}{\partial r} \right)
        +
        \frac{1}{r^2\sin\theta} \frac{\partial}{\partial\theta}
            \left(\sin\theta \frac{\partial \Psi}{\partial \theta} \right)
        +
        \frac{1}{r^2\sin\theta}
            \left(\frac{\partial^2 \Psi}{\partial \phi^2} \right)
    \right]
    +
    V \Psi
    =
    E \Psi.
\end{equation}
We look solutions to this equations of the form 
\begin{equation}
    \Psi(r,\theta,\phi) = R(r) Y(\theta, \phi).
\end{equation}
Inserting this into \autoref{eq:spherical_tise} gives,
\begin{equation}
   -\frac{\hbar^2}{2m}
   \left[
        \frac{Y}{r^2} \frac{d}{dr}
            \left(r^2\frac{dR}{dr} \right)
        +
        \frac{R}{r^2\sin\theta}
            \left(\sin\theta \frac{\partial Y}{\partial\theta} \right)
        +
        \frac{R}{r^2\sin^2\theta}
            \left(\frac{\partial^2Y}{\partial\phi^2} \right)
   \right] 
    +
    VRY
    =
    ERY,
\end{equation}
which we multiply by $-\frac{2mr^2}{RY\hbar^2}$,
\begin{equation}
    \begin{aligned}
        &
        \left\{ 
            \frac{1}{R}\frac{d}{dr}
                \left(r^2\frac{dR}{dr} \right)
            -
            \frac{2mr^2}{\hbar^2}
                \left[V(r) - E \right]
        \right\}
        \\ + \frac{1}{Y} &
        \left\{
            \frac{1}{\sin\theta }\frac{\partial}{\partial\theta}
                \left(\sin\theta \frac{\partial Y}{\partial\theta} \right)
            +
            \frac{1}{\sin\theta}\frac{\partial^2Y}{\partial\phi^2}
        \right\}
        = 0,
    \end{aligned}
\end{equation}
where the first part is only dependent on $r$ and the second part is only
dependent on $\theta$ and $\phi$. Notice that we have also assumed a spherical 
symmetric potential, i.e. $V=V(r)$. Next, we introduce the cleverly chosen 
separation constant $l(l + 1)$,
\begin{align}
    \label{eq:the_radial_equation}
    \frac{1}{R}\frac{d}{dr}
        \left(r^2\frac{dR}{dr}\right)
    -
    \frac{2mr^2}{\hbar^2}\left[V(r) - E \right] 
    &= l (l + 1) \\
    \label{eq:the_angular_equation}
    \frac{1}{Y}\left\{ 
        \frac{1}{\sin\theta}\frac{\partial}{\partial\theta}
            \left(\sin\theta \frac{\partial Y}{\partial\theta} \right)
        +
        \frac{1}{\sin^2\theta}\frac{\partial^2 Y}{\partial \phi^2}
    \right\} 
    &= - l (l + 1).
\end{align}

\subsection{The Angular Equation}

Starting with \autoref{eq:the_angular_equation}, which dictates the dependence if
$\Psi$ on $\theta$ and $\phi$,
\begin{equation}
    \frac{1}{\sin\theta}\frac{\partial}{\theta} 
        \left(\sin\theta\frac{\partial Y}{\partial \theta} \right)
    +
    \frac{1}{\sin^2\theta} \frac{\partial^2}{\partial\phi^2}
    =
    -l(l + 1)\sin^2\theta Y.
\end{equation}
We wish to make a further separation, by inserting
$Y(\theta, \phi) = \Theta(\theta)\Phi(\phi)$ and dividing by $\Theta\Phi$,
\begin{equation}
    \frac{1}{\Theta}
    \left[
        \sin\theta \frac{d}{d\theta}
            \left(\sin\theta \frac{d\Theta}{d\theta} \right)
    \right]
    + l(l + 1) \sin\theta 
    +
    \frac{1}{\Phi} \frac{d^2\Phi}{d\phi} = 0,
\end{equation}
where again see that the variables have gathered in separate terms. We therefore 
introduce the separation constant $m^2$,
\begin{align}
    \label{eq:angular_1}
    \frac{1}{\Theta}
    \left[
        \sin\theta \frac{d}{d\theta}
            \left(\sin\theta \frac{d\Theta}{d\theta} \right)
    \right]
    + l(l + 1)\sin\theta &= m^2 \\
    \label{eq:angular_2}
    \frac{1}{\Phi}\frac{d^2\Phi}{d\phi^2} &= -m^2.
\end{align}
Of these two \autoref{eq:angular_2} is instantly recognisable,
\begin{equation}
    \frac{d^2\Phi}{d\phi^2} = -m^2\Phi,
\end{equation}
with the general solution
\begin{equation}
    \Phi(\phi) = C_1e^{im\phi} + C_2 e^{-im\phi},
\end{equation}
but we will choose the more simpler solution 
\begin{equation}
    \Phi(\phi) = e^{im\theta}.
\end{equation}
We require periodicity, i.e. $\Phi(\phi + 2\pi) = \Phi(\phi)$, which means that 
$m$ only takes integer values,
\begin{equation*}
    m = 0, \pm 1, \pm 2, \pm 3, \dots
\end{equation*}

The other angular equation, in \autoref{eq:angular_1}, is not so simple,
\begin{equation}
    \sin\theta\frac{d}{d\theta}\left(\sin\theta\frac{d\Theta}{d\theta} \right)
    + \left[l(l + 1)\sin\theta - m^2 \right]\Theta = 0.
\end{equation}
It has the more complicated solutions,
\begin{equation}
    \label{eq:angular_1_solution}
    \Theta(\theta) = A P^m_l(\cos\theta),
\end{equation}
where $P^m_l$ is the associated Legendre function,
\begin{equation}
    P^m_l(x) \equiv (1-x^2)^{|m|/2} \left(\frac{d}{dx}\right)^{|m|}P_l(x),
\end{equation}
and $P_l$ is the $l$th Legendre polynomial given by the Rodrigues formula,
\begin{equation}
    P_l(x) \equiv \frac{1}{2^l l!}\left(\frac{d}{dx} \right)^l(x^2 - l)^l.
\end{equation}
We see that for this function to be take a value, we must have $l > 0$. Moreover, 
we see that if $|m| > l$ then $P^m_l = 0$. For a given $l$, there must be 
$2l + 1$ possible values for $m$,
\begin{equation*}
    l = 0, 1, 2, \dots \quad m = l, -1 + 1, \dots - 1, 0, 1, \dots l-1, l.
\end{equation*}
A mathematician would at this point argue that because \autoref{eq:angular_1} is 
a second-order differential equation it should have two sets of solutions, not 
just the solution in \autoref{eq:angular_1_solution}. Another set of solutions 
exist but these are not sensible in the physical sense as they have singularities 
at $\theta=0, \pi$.

We would want to normalise the solutions we have found thus far. In spherical 
coordinates the volume element is 
\begin{equation}
    d^3\vb{r} = r^2 \sin\theta dr d\theta d\phi,
\end{equation}
this gives,
\begin{equation}
    \int\int\int |\Psi|^2 r^2\sin\theta dr d\theta d\phi
    = \int |R|^2r^2 dr \int\int|Y|^2\sin\theta dr d\theta d\phi = 1.
\end{equation}
We can normalise these part separately,
\begin{equation}
    \int_0^\infty |R|^2 r^2 dr = 1, \quad 
    \int_0^{2\pi}\int_0^\pi |Y|^2 \sin\theta d\theta d\phi = 1.
\end{equation}
It so happens that the normalised angular wave functions are the 
spherical harmonics,
\begin{equation}
    \label{eq:spherical_harmonics_app}
    Y^m_l(\theta, \phi) 
    =
    \epsilon\sqrt{\frac{2l + 1}{4\pi}\frac{(l - |m|)!}{(l + |m|)!}}
    e^{im\phi} P^m_l(\cos\theta).
\end{equation}
A nice attribute of the spherical harmonics is that they are orthogonal.

\subsection{The Radial Equation}

For the type of spherical symmetric system we are dealing with, we have 
saved the best part for last, namely \autoref{eq:the_radial_equation},
which contains the potential $V$,
\begin{equation}
    \label{eq:radial_equation2}
    \frac{d}{dr}\left(r^2\frac{dR}{dr} \right)
    -
    \frac{2mr^2}{\hbar^2} [V(r) - E]R
    =
    l(l + 1)R.
\end{equation}
To make our efforts less effortfull we make a variable change, $u(r) \equiv = r R(r)$,
such that 
\begin{equation*}
    R = \frac{u}{r}, \quad
    \frac{dR}{dr} = \frac{1}{r}\frac{du}{dr} - \frac{u}{r^2}, \quad 
    \frac{d}{dr}\left(r^2 \frac{dR}{dr} \right)
        = \frac{d^2u}{dr^2},
\end{equation*}
making \autoref{eq:radial_equation2},
\begin{equation}
    - \frac{\hbar^2}{2m}\frac{d^2u}{dr^2}
    +
    \left[
        V + \frac{\hbar^2}{2m}\frac{l(l + 1)}{r^2} 
    \right]
    u = Eu,
\end{equation}
which is identical to one-dimensional Schrödinger equation, except for 
a new \emph{effective} potential,
\begin{equation}
    V_\text{eff} = V + \frac{\hbar}{2m}\frac{l(l + 1)}{r^2}.
\end{equation}
For the radial equation with the substituted variable, we also require normalisation,
\begin{equation} 
    \int_0^\infty |u|^2 dr = 1.
\end{equation}

This is as far we get without insertion of a potential. 

\section{Quantum Systems}

There are several such potentials, some that constitute mere toy models, like 
the ``particle-in-a-box'' potential, also called the infinite square well,
\begin{equation}
    V(r) = \begin{cases}
        0,      &\text{ if } r < a;\\
        \infty  &\text{ if } r > a.
    \end{cases} 
 \end{equation}
Other potentials are models nature more properly, like the potential that models
the hydrogen atom.

The hydrogen atom consists of a heavy proton of charge $e$, together 
with a much lighter electron of charge $-e$, that orbits around it. The proton is 
essentially motionless, while the electron 
in bound by the mutual attraction between opposite charges. The potential energy 
follows from Coulomb's law,
\begin{equation}
    V(r) = - \frac{e^2}{4\pi \epsilon_0}\frac{1}{r}.
\end{equation}
This gives a relatively simple radial equation,
\begin{equation}
    -\frac{\hbar^2}{2m}\frac{d^2u}{dr^2}
    + \left[
        - \frac{e^2}{4\pi\epsilon_0}\frac{1}{r} 
        + \frac{\hbar}{2m}\frac{l(l + 1)}{r^2}
    \right]
    u = E_u.
\end{equation}

After Hydrogen, the simples atom is helium. The Hamiltonian,
\begin{equation}
    \hat{H} =
    \left\{
        -\frac{\hbar^2}{2m}\nabla^2_i
        -\frac{1}{4\pi\epsilon_0} \frac{2e^2}{r_1}
    \right\}
    +
    \left\{
        -\frac{\hbar^2}{2m}\nabla^2_i
        -\frac{1}{4\pi\epsilon_0} \frac{2e^2}{r_2}
    \right\}
    +
    \frac{1}{4\pi\epsilon_0}\frac{e^2}{|\vb{r}_1 - \vb{r}_2|},
\end{equation}
consists of two hydrogen-like Hamiltonians, one for electron $1$ and one for 
electron $2$, inside curly braces. The last term describes the repulsion between 
the two electron, and causes trouble. If we ignore the last term, the Schrödinger
equation separates, and we would get a ground state energy of $-109 \text{ eV}$,
but the experimentally measured value of $-78.975 \text{ eV}$s.

\subsection{The Quantum Harmonic Oscillator}

The quantum harmonic oscillator is the quantum mechanical analogue to the 
classical harmonic oscillator. Any arbitrary potential can be closely 
approximated as a harmonic potential about its equilibrium point, making the 
quantum harmonic oscillator one of the most important systems in quantum mechanics.

\subsubsection{One Dimension}

The hamiltionian of a one-dimensional harmonic oscillator is
\begin{equation}
    \hat{H} = -\frac{\hbar^2}{2m} \nabla^2 + \frac{1}{2}m\omega^2\hat{x}^2,
\end{equation}
where the potential function is a rewritten form of Hooke's law,
\begin{equation}
    V(x) = \frac{1}{2}k \hat{x}^2 = \frac{1}{2}m\omega^2\hat{x}^2,
\end{equation}
such that 
\begin{equation*}
    \omega = \sqrt{\frac{k}{m}}.
\end{equation*}
The eigenstates of the one-dimensional harmonic oscillator is given by,
\begin{equation}
    \Psi_n(x) =
        \frac{1}{\sqrt{2^n n!}} 
        \left(\frac{m\omega}{\pi\hbar} \right) 
        e^{-\frac{m\omega\hat{x}^2}{2\hbar}}
        H_n\left(\sqrt{\frac{m\omega}{\hbar}} x \right),
        \quad n=0,1,2,3,\dots,
\end{equation}
where the functions $H_n$ are the Hermite polynomials,
\begin{equation}
    H_n(z) = 
    (-1)^n e^{x^2}
    \left(\frac{d}{dz} \right)^n
    \left(e^{-z^2} \right).
\end{equation}
The corresponding eigenenergies are,
\begin{equation}
    E_n = \hbar\omega\left(n + \frac{1}{2} \right).
\end{equation}

\subsubsection{Higher Dimensions}

The three-dimensional harmonic oscillator can be solved in cartesian 
coordinates by separation of variables. The time-dependent Schrödinger
equation for this system is,
\begin{equation}
    \label{eq:3d_ho_tise}
    -\frac{\hbar^2}{2m} \nabla^2\Psi
    +\frac{1}{2}m\omega^2\hat{r}^2\Psi = E\Psi,
\end{equation}
where $\hat{r} = \sqrt{\hat{x}^2 + \hat{y}^2 + \hat{z}^2}$. We insert
\begin{equation*}
    \Psi(x, y, z) = \xi(x)\eta(y)\zeta(z)
\end{equation*}
into \autoref{eq:3d_ho_tise} and divide by $\xi\eta\zeta$,
\begin{equation}
    -\frac{\hbar^2}{2m}\frac{1}{\xi}\frac{\partial \xi}{x}
    +\frac{1}{2}m\omega^2\hat{x}^2
    -\frac{\hbar^2}{2m}\frac{1}{\eta}\frac{\partial \eta}{x}
    +\frac{1}{2}m\omega^2\hat{y}^2
    -\frac{\hbar^2}{2m}\frac{1}{\zeta}\frac{\partial \zeta}{x}
    +\frac{1}{2}m\omega^2\hat{z}^2 = E.
\end{equation}
In this expressions we have two terms each of which depends on 
only one of the variables $x$, $y$ and $z$, the sum of which is 
the constant $E$. Each of the three groups of terms must be constant 
on its own, and we can conclude that the three-dimensional harmonic 
oscillator is equivalent to the sum of three one-dimensional 
oscillators.

The general energy expression of for a higher-dimensional 
quantum harmonic oscillator reads,
\begin{equation}
    E_n = \left(n + \frac{d}{2} \right)\hbar\omega,
\end{equation}
where $d$ is the number of dimensions.

It is also possible to solve the harmonic-oscillator system in 
spherical coordinates, by inserting the potential
\begin{equation*}    
    V(r) = \frac{1}{2}m \omega^2\hat{r}^2,
\end{equation*}
into the radial equation, specified in \autoref{eq:the_radial_equation},
\begin{equation}
    -\frac{\hbar^2}{2m}\frac{d^2u}{dr^2}
    + \left(
        \frac{1}{2}m\omega^2\hat{r}^2 
        + \frac{\hbar^2}{2m}\frac{l(l + 1)}{r^2} 
    \right)u = Eu.
\end{equation}

The solutions will be different, depending on the number of dimensions.
In two dimensions we have eigenfunctions given by,
\begin{equation}
    \Psi_{nm} (r,\theta) 
    = N_{nm}ae^{im\theta}(ar)^{|m|}L_n^{|m|}(a^2r^2)e^{-a^2r^2/2},
\end{equation}
where 
\begin{equation*}
    a \equiv \sqrt{\frac{m\omega}{\hbar}},
\end{equation*}
$N_{nm}$ is a normalisation contant given by,
\begin{equation*}
    N_{nm} \sqrt{\frac{n!}{\pi(n + |m|)!}},
\end{equation*}
and
\begin{equation}
    \ref{eq:assoc_laguerre_app}
    L^p_{q-p}(x) \equiv (-1)^p \left(\frac{d}{dx} \right)^p L_q(x),
\end{equation}
is an associated Laguerre polynomial, where 
\begin{equation}
    L_q(x) \equiv e^x \left(\frac{d}{dx} \right)^q (e^{-x}x^q),
\end{equation}
is the $q$th Laguerre polynomial.

In three dimensions the eigenfunctions are given by,
\begin{equation}
    \Psi_{nlm}(r,\theta,\phi)
    =
    N_{nl}r^l e^{-a^2r^2/2}L_n^{l + \frac{1}{2}}(a^2r^2)Y^m_l(\theta,\phi),
\end{equation}
where $N_{nl}$ is a normalisation constant, $L$ are the associated Laguerre 
polynomials given by \autoref{eq:assoc_laguerre_app} and 
$Y$ are the spherical harmonics given by \autoref{eq:spherical_harmonics_app}.
