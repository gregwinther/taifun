\chapter{2D Coulomb elements}
\label{app:anisimovas}
Implementation of two-body matrix elements for the two-dimensional
quantum dots\cite{anisimovas1998energy}. Note that Anisimovas and Matulis uses the chemist's convention 
$\bra{ij} \hat{u} \ket{lk}$ which is 
$\bra{ij} \hat{u} \ket{kl}$ in the physcist's notation.
That is, the last two indices are interchanged.

\begin{python}
def coulomb_ho(n_i, m_i, n_j, m_j, n_l, m_l, n_k, m_k):
    element = 0

    if m_i + m_j != m_k + m_l:
        return 0

    M_i = 0.5 * (abs(m_i) + m_i)
    dm_i = 0.5 * (abs(m_i) - m_i)

    M_j = 0.5 * (abs(m_j) + m_j)
    dm_j = 0.5 * (abs(m_j) - m_j)

    M_k = 0.5 * (abs(m_k) + m_k)
    dm_k = 0.5 * (abs(m_k) - m_k)

    M_l = 0.5 * (abs(m_l) + m_l)
    dm_l = 0.5 * (abs(m_l) - m_l)

    n = np.array([n_i, n_j, n_k, n_l], dtype=np.int64)
    m = np.array([m_i, m_j, m_k, m_l], dtype=np.int64)
    j = np.array([0, 0, 0, 0], dtype=np.int64)
    l = np.array([0, 0, 0, 0], dtype=np.int64)
    g = np.array([0, 0, 0, 0], dtype=np.int64)

    for j_1 in range(n_i + 1):
        j[0] = j_1
        for j_2 in range(n_j + 1):
            j[1] = j_2
            for j_3 in range(n_k + 1):
                j[2] = j_3
                for j_4 in range(n_l + 1):
                    j[3] = j_4

                    g[0] = j_1 + j_4 + M_i + dm_l
                    g[1] = j_2 + j_3 + M_j + dm_k
                    g[2] = j_3 + j_2 + M_k + dm_j
                    g[3] = j_4 + j_1 + M_l + dm_i

                    G = np.sum(g)
                    ratio_1 = log_ratio_1(j)
                    prod_2 = log_product_2(n, m, j)
                    ratio_2 = log_ratio_2(G)

                    temp = 0
                    for l_1 in range(g[0] + 1):
                        l[0] = l_1
                        for l_2 in range(g[1] + 1):
                            l[1] = l_2
                            for l_3 in range(g[2] + 1):
                                l[2] = l_3
                                for l_4 in range(g[3] + 1):
                                    l[3] = l_4

                                    if l_1 + l_2 != l_3 + l_4:
                                        continue

                                    L = np.sum(l)

                                    temp += (
                                        -2
                                        * (int(g[1]+g[2]-l[1]-l[2]) & 0x1)
                                        + 1) 
                                        * np.exp(
                                        log_product_3(l, g)
                                        + math.lgamma(1.0 + 0.5 * L)
                                        + math.lgamma(0.5 * (G - L + 1.0)))
                    element += (
                        (-2 * (int(np.sum(j)) & 0x1) + 1)
                        * np.exp(ratio_1 + prod_2 + ratio_2)
                        * temp)

    element *= log_product_1(n, m)
    return element
\end{python}