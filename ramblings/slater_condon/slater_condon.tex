\chapter{Slater-Condon Rules}

% This stuff is mostly from page 60 and onwards in Shavitt and Bartlett

The Slater-Condon rules are ways to express integrals over operators in
terms of single-particle orbitals. Here is an outline of a proof for these
rules.

Consider first some Slater determinants,
\begin{gather}
    \ket{I} = \ket{i_1 i_2 \dots i_N} = 
        \hat{i}^\dagger_1 \hat{i}^\dagger_2 \dots \hat{i}^\dagger_N \ket{ } \\
    \ket{J} = \ket{j_1 j_2 \dots j_N} = 
        \hat{j}^\dagger_1 \hat{j}^\dagger_2 \dots \hat{j}^\dagger_N \ket{ }.
\end{gather}
To get started, we want to compute the inner product $\braket{I}{J}$ of these two Slater
determinants,
\begin{equation}
    \braket{I}{J} = \bra{ }\hat{i}_N \dots \hat{i}_2 \hat{i}_1 
    \hat{j}^\dagger_1 \hat{j}^\dagger_2 \dots \hat{j}^\dagger_N\ket{ }.
\end{equation}
In order to evaluate this expression, we move every annihilation operater $\hat{i}_p$
to the right. Starting with $\hat{i}_1$, for instance, we have two possible outcomes.
If there is no $\hat{j}_q$ that is the same as $\hat{i}_1$ we get
\begin{equation}
    \braket{I}{J} = \bra{ }\hat{i}_N \dots \hat{i}_2  
    \hat{j}^\dagger_1 \hat{j}^\dagger_2 \dots \hat{j}^\dagger_N
    \hat{i}_1 \ket{ }(-1)^N = 0,
\end{equation}
because $\hat{i}_1\ket{ } = 0$. The other possibility that may arise is that
$\hat{i}_1 = \hat{j}_q$, so that
\begin{equation}
    \hat{i}_1 \hat{j}^\dagger_q 
    = \{ \hat{i}_1, \hat{j}^\dagger_q \} - \hat{j}^\dagger_q \hat{i}_1
    = \hat{\delta}_{i_1k_q} - \hat{j}^\dagger_p\hat{i}_1 
    = \hat{1} - \hat{j}^\dagger_q \hat{i}_1,
\end{equation}
and
\begin{equation}
    \braket{I}{J} = \bra{ }\hat{i}_N \dots \hat{i}_2  
    \hat{j}^\dagger_1 \hat{j}^\dagger_2 \dots 
    \hat{j}^\dagger_{p-1} \hat{j}^\dagger_{p+1} \dots
    \hat{j}^\dagger_N
    \hat{i}_1 \ket{ }(-1)^{p-1} - 0.
\end{equation}
We continue in this manner, moving all $\hat{i}$ to the right and the final result
will be zero if there are any $\hat{i}_p$ without a matching $\hat{j}_q$ or $(-1)^\tau$
if the two operator strings are identical to a permutation $\tau$.

Next, consider a symmetric one-body operator
\begin{equation}
    \hat{F} = \sum_{\mu = 1}^N \hat{f}_\mu,
\end{equation}
where $\mu$ is the identity of the electron on which the identical $\hat{f}_\mu$ operate.
Computing a matrix element of this one-body operator between two Slater determinants
will yield three possible results,
\begin{equation}
    \begin{aligned}
        \bra{I} \hat{F}& \ket{J} 
            = \bra{i_1 i_2 \dots i_N} \hat{F} \ket{j_1 j_2 \dots j_N} \\
        =& \sum_\mu \bra{i_1 i_2 \dots i_N} \hat{f}_\mu \ket{j_1 j_2 \dots j_N} \\
        =& \sum_\mu \bra{\phi_{i_1} \phi_{i_2} \dots \phi_{i_N}}
            \hat{f}_\mu \sum_{\hat{P}} (-1)^{\sigma(\hat{P})}
            \ket{\hat{P} \phi_{j_1} \phi_{j_2} \dots \phi_{j_N}}
        =
        \begin{cases}
           \sum_k \bra{i_k} \hat{f} \ket{i_k}(-1)^{\sigma(\hat{P})} &\ \text{I} \\
           \bra{i_k} \hat{f} \ket{i'_k}(-1)^{\sigma(\hat{P})} &\ \text{II} \\
           0 &\ \text{III}
        \end{cases}
    \end{aligned}
\end{equation}
In the last line, the integral is written with spinorbitals instead of Slater determinants.
The result will be the first case (I), if the operators needed to construct the Slater 
determinants are the same, up to a permutation with permutation parity $\sigma$ associated
with the permutation operator $\hat{P}$ needed to permute the product of spinorbitals. If there exists
excactly one noncoincidence in the string of operators so that 
$\hat{P} j_1 j_2 \dots j_N = i_1 i_2 \dots i'_k \dots i_N$ where $i_k \neq i'_k$, we get
the result in the second case (II). If there are two or more noncoincidences, the result
is zero (III).

With second quantisation we might write a one-electron operators differently,
\begin{equation}
    \sum_{kl} \bra{k} \hat{f} \ket{l} \hat{a}^\dagger_k \hat{a}_l
    = \sum_{kl} f_{kl} \hat{a}^\dagger_k \hat{a}_l.
\end{equation}
It is possible to show that the results are the same in this representation. First,
consider the case where the two Slater determinants are equal,
\begin{equation}
    \begin{aligned}
        \bra{I}& \sum_{kl} f_{kl} \hat{a}^\dagger_k \hat{a}_l \ket{I}
        = \sum_{kl} f_{kl} \bra{I} \hat{a}^\dagger_k \hat{a}_l \ket{I} \\
        =& \sum_{kl} f_{kl} \delta_{kl} n_l(I) = \sum_{k \in I} f_{kk}
        = \sum_{k=1}^N \bra{i_k} \hat{f} \ket{i_k}.
    \end{aligned}
\end{equation}
Second, we look at the case where we have one noncoincidence, $i_p \neq j_p$,
\begin{equation}
    \begin{aligned}
        \bra{I}& \sum_{kl}f_{kl} \bra{I} \hat{a}^\dagger_k \hat{a}_l \ket{J}
        = \sum_{kl} f_{kl} \bra{I} \hat{a}^\dagger_k \hat{a}_l\ket{J} \\
        =& \sum_{kl \neq p} f_{kl} \bra{I} \hat{a}^\dagger_k \hat{a}_l \ket{J}
        + f_{i_p j_p} \bra{I} \hat{i}^\dagger_p \hat{j}_p \ket{J} \\
        =& 0 + f_{i_p j_p} \braket{I'} = \bra{\hat{i}_p}\hat{f}\ket{\hat{i}_p}.
    \end{aligned}
\end{equation}
Lastly, there is no pair of operators $\hat{a}^\dagger_k \hat{a}_l$ that will give
a non-zero result. Consequently, we see that the second-quantised form of the
one-body operator gives the same result.

Similarly, consider a symmetric two-body operator,
\begin{equation}
    \hat{G} = \sum_{\mu < \nu}^N \hat{g}_{\mu\nu} 
        = \frac{1}{2} \sum_{\mu \neq \nu}^N \hat{g}_{\mu\nu} \\
        = \frac{1}{2} \sum_{ijkl} \bra{ij} \hat{g} \ket{kl} 
        \hat{a}^\dagger_i \hat{a}^\dagger_j \hat{a}_l \hat{a}_k.
\end{equation}

We would like to show that the second-quantized form is correct, and therefore
firstly consider the case where the two Slater determinants are equal, i.e. zero
noncoincidences;
\begin{equation}
    \bra{I} \hat{G} \ket{I} 
    = \frac{1}{2}\sum_{ijkl} \bra{ij}\hat{G} \ket{kl} 
        \bra{I} \hat{a}^\dagger_i \hat{a}^\dagger_j \hat{a}_l \hat{a}_k \ket{I}.
\end{equation}
We must have $k = i_p$ and $l = i_q$ appear in $\ket{I}$, so that
\begin{equation}
    \begin{aligned}
    \bra{I} \hat{G} \ket{I}
    =& \frac{1}{2}\sum_{ij} \bra{ij}\hat{g}\ket{i_p i_q} 
        \bra{I} \hat{a}^\dagger \hat{a}^\dagger \hat{a}_{i_p} \hat{a}_{i_q}
        \ket{i_1 i_2 \dots i_p \dots i_q \dots} \\
    =& \frac{1}{2}\sum_{ij} \bra{ij} \hat{g}\ket{i_p i_q} 
        \bra{I} \hat{a}^\dagger_i \hat{a}^\dagger_j
        \ket{i_1 i_2 \dots} (-1)^{(p-1) + (q-2)}.
    \end{aligned}
\end{equation}
From this point we have two possibilities for the values of $i$ and $j$, because
the creation operators must put the same values back into the ket,
\begin{align}
    &\begin{aligned}
    \bra{i_p i_q} \hat{g} \ket{i_p i_q} 
        &\braket{I}{i_1 i_2 \dots i_p \dots i_q \dots}
        (-1)^{(p-1) + (q-2)}(-1)^{(p-1) + (q-2)} \\
    &= \bra{i_p i_q} \hat{g} \ket{i_p i_q}
    \end{aligned}
    \quad (i = i_p,\ j = i_q); \\
    &\begin{aligned}
    \bra{i_q i_p} \hat{g} \ket{i_p i_q} 
        &\braket{I}{i_1 i_2 \dots i_p \dots i_q \dots}
        (-1)^{(p-1) + (q-2)}(-1)^{(p-1) + (q-1)} \\
    &= -\bra{i_q i_p} \hat{g} \ket{i_p i_q}
        = -\bra{i_p i_q} \hat{g} \ket{i_q i_p}
    \end{aligned}
    \quad (i = i_q,\ j = i_p).
\end{align}
By starting in the reverse order, we obtain the same contributions. The total matrix
element is therefore,
\begin{equation}
    \bra{I} \hat{G} \ket{I} 
    = \frac{1}{2}\sum_{i \in I} \sum_{j \in J}
        (\bra{ij} \hat{g} \ket{ij} - \bra{ij} \hat{g} \ket{ji})
    = \sum_{\substack{i < j \\ i,j \in I}} \bra{ij} \hat{g} \ket{ij}_{\text{AS}}.
\end{equation}

Next, we consider a single noncoincidence in $\ket{I}$, $i_p \neq i'_p$,
\begin{align}
    \ket{I} &= \ket{i_1 i_2 \dots i_p \dots}, \\
    \ket{I'} &= \ket{i_1 i_2 \dots i'_p \dots}.
\end{align}
We get contributions to $\bra{I} \hat{G} \ket{I'}$ from the operator string
$\hat{a}^\dagger_i \hat{a}^\dagger_j \hat{a}_l \hat{a}_k$ in the following
cases,
\begin{align}
   i &= i'_p, \ k = i_p, \ j = l = i_q \to \braket{i'_p i_q}{i_p i_q} \\  
   i &= i'_p, \ l = i_p, \ j = k = i_q \to -\braket{i'_p i_q}{i_q i_p} \\
   j &= i'_p, \ l = i_p, \ i = k = i_q \to \braket{i_q i'_p}{i_q i_q} \\
   j &= i'_p, \ k = i_p, \ i = l = i_q \to -\braket{i_q i'_p}{i_p i_q},
\end{align}
where the two first terms are egual to the last terms, respectively. This leaves
us with,
\begin{equation}
    \bra{I'}\hat{G}\ket{I} 
    = 2 \times \frac{1}{2} 
    (\bra{i'_p j}\hat{g}\ket{i_p j} - \bra{i'_p j}\hat{g}\ket{j i_p})
    = \sum_{j\in I} \bra{i'_p j} \hat{g} \ket{i_p j}_{\text{AS}}.
\end{equation}

After a while we see a pattern emerges. For two noncoincidences 
($i_p \neq i'_p$, $i_q \neq i'_q)$ we have,
\begin{equation}
    \bra{I'} \hat{G} \ket{I} = \bra{i'_p i'_q} \hat{g} \ket{i_p i_q},
\end{equation}
while for three or more noncoincidences,
\begin{equation}
    \bra{I'} \hat{G} \ket{I} = 0. 
\end{equation}
