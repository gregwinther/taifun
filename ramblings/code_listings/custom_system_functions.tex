\section{Function for constructing system from Psi4}
\label{sec:custom_system_psi4}

This function extracts a basis set from Psi4 \cite{parrish2017psi4}, specified 
by the \lstinline{molecule} argument. Psi4 also allows for several \lstinline{options}
in the construction of a basis set. Please consult the Psi4 documentation on 
\url{https://psicode.org}, for a guide on how to create such a specification.
The return object of this function is an instance of the
\lstinline{QuantumSystem} class, specified in
\autoref{ch:impl_quantum_systems}.

\begin{python}
def construct_psi4_system(
    molecule, options, np=None, add_spin=True, anti_symmetrize=True
):
    import psi4

    if np is None:
        import numpy as np

    psi4.core.be_quiet()
    psi4.set_options(options)

    mol = psi4.geometry(molecule)
    nuclear_repulsion_energy = mol.nuclear_repulsion_energy()

    wavefunction = psi4.core.Wavefunction.build(
        mol, psi4.core.get_global_option("BASIS")
    )

    molecular_integrals = psi4.core.MintsHelper(wavefunction.basisset())

    kinetic = np.asarray(molecular_integrals.ao_kinetic())
    potential = np.asarray(molecular_integrals.ao_potential())
    h = kinetic + potential

    u = np.asarray(molecular_integrals.ao_eri()).transpose(0, 2, 1, 3)
    overlap = np.asarray(molecular_integrals.ao_overlap())

    n_up = wavefunction.nalpha()
    n_down = wavefunction.nbeta()
    n = n_up + n_down
    l = 2 * wavefunction.nmo()

    dipole_integrals = [
        np.asarray(mu) for mu in molecular_integrals.ao_dipole()
    ]
    dipole_integrals = np.stack(dipole_integrals)

    system = CustomSystem(n, l, n_up=n_up, np=np)
    system.set_h(h, add_spin=add_spin)
    system.set_u(u, add_spin=add_spin, anti_symmetrize=anti_symmetrize)
    system.set_s(overlap, add_spin=add_spin)
    system.set_dipole_moment(dipole_integrals, add_spin=add_spin)
    system.set_nuclear_repulsion_energy(nuclear_repulsion_energy)

    return system
\end{python}

\section{Function for constructing system from PySCF}
\label{sec:custom_system_pyscf}

This function extracts a basis set from PySCF \cite{PYSCF}, specified 
by the \lstinline{molecule} and \lstinline{basis} arguments.
Please consult the PySCF documentation on 
\url{https://pyscf.github.io}, for a guide on how to create such a specification.
The return object of this function is an instance of the
\lstinline{QuantumSystem} class, specified in
\autoref{ch:impl_quantum_systems}.

\begin{python}
def construct_pyscf_system(molecule, basis="cc-pvdz", np, verbose=False):
    import pyscf

    if np is None:
        import numpy as np

    # Build molecule in AO-basis
    mol = pyscf.gto.Mole()
    mol.unit = "bohr"
    mol.build(atom=molecule, basis=basis, symmetry=False)
    mol.set_common_origin(np.array([0.0, 0.0, 0.0]))
    nuclear_repulsion_energy = mol.energy_nuc()

    # Perform UHF-calculations to create the MO-basis
    hf = pyscf.scf.UHF(mol)
    ehf = hf.kernel()

    if not hf.converged:
        warnings.warn("UHF calculation did not converge")

    if verbose:
        print(f"UHF energy: {hf.e_tot}")

    # Build the coefficient matrix from the occupied and virtual integrals.
    # As we have done a UHF-calculation, we stack the two spin-directions 
    # on top of each other. That is, instead of using odd or even indices
    # for each spin direction, we set up two separate blocks.
    C_o = np.hstack(
        (
            # Fetch occupied coefficients for both spin-directions
            hf.mo_coeff[0][:, hf.mo_occ[0] > 0],
            hf.mo_coeff[1][:, hf.mo_occ[1] > 0],
        )
    )

    C_v = np.hstack(
        (
            # Fetch virtual coefficients for both spin-directions
            hf.mo_coeff[0][:, hf.mo_occ[0] == 0],
            hf.mo_coeff[1][:, hf.mo_occ[1] == 0],
        )
    )

    # Build full coefficient matrix.
    C = np.hstack((C_o, C_v))

    # Get the number of occupied molecular orbitals
    n = C_o.shape[1]
    # Fetch the number of molecular orbitals
    l = C.shape[1]

    # Check that the number of occupied molecular orbitals is correct
    assert n == sum(hf.mo_occ[0] > 0) + sum(hf.mo_occ[1] > 0)
    # Check that the number of molecular orbitals is twice of that of
    # the number of atomic orbitals.
    assert l == C.shape[0] * 2

    # Note: Should the dipole moments have a negative sign?
    dipole_moment = [
        -transform_one_body_elements(dm, C, np)
        for dm in mol.intor("int1e_r").reshape(3, mol.nao, mol.nao)
    ]
    dipole_moment = np.asarray(dipole_moment)

    # Create a tuple with the shape of the AO two-body elements
    u_shape = (mol.nao for i in range(4))

    h = transform_one_body_elements(hf.get_hcore(), C, np)
    u = transform_two_body_elements(
        mol.intor("int2e").reshape(*u_shape),
        C,
        np,
    )

    noa = sum(hf.mo_occ[0] > 0)
    nva = sum(hf.mo_occ[0] == 0)
    nob = sum(hf.mo_occ[1] > 0)
    nvb = sum(hf.mo_occ[1] == 0)
    no = noa + nob
    nv = nva + nvb

    oa = slice(0, noa)
    ob = slice(noa, no)
    va = slice(no, no + nva)
    vb = slice(no + nva, no + nv)

    a_slices = [oa, va]
    b_slices = [ob, vb]

    # Create a combination of slices that should be zero in
    # all matrix elements due to unequal spin-direction.
    zero_slices = [(a, b) for a in a_slices for b in b_slices]
    zero_slices += [(b, a) for a in a_slices for b in b_slices]

    # Create a slice object for all indices, i.e., the ":" syntax in NumPy.
    all_slice = slice(None, None)

    # Explicitly set all cross-spin terms to zero
    for s in zero_slices:
        h[s] = 0
        dipole_moment[(all_slice,) + s] = 0
        u[s + (all_slice, all_slice)] = 0
        u[(all_slice, all_slice) + s] = 0

    # Convert to physicist's notation, from Mulliken notation
    u = u.transpose(0, 2, 1, 3)

    # Build a custom system from the integral elements
    system = CustomSystem(n, l, np=np)
    system.set_h(h)
    system.set_u(u, anti_symmetrize=True)
    system.set_dipole_moment(dipole_moment)
    system.set_nuclear_repulsion_energy(nuclear_repulsion_energy)
    system.cast_to_complex()

    return system
\end{python}